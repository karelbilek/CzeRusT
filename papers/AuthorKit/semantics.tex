i%File: formatting-instruction.tex
\documentclass[letterpaper]{article}
\usepackage{aaai}
\usepackage{times}
\usepackage{helvet}
\usepackage{courier}
\frenchspacing
\pdfinfo{
/Title (Using maching learning for automatic semantic feature assignment) %
%/Subject (AAAI Publications)
%/Author (AAAI Press)}
\setcounter{secnumdepth}{0}  
 \begin{document}
% The file aaai.sty is the style file for AAAI Press 
% proceedings, working notes, and technical reports.
%
\title{}
%\author{}
\maketitle
\begin{abstract}
\begin{quote}
\end{quote}
\end{abstract}


\section{Introduction}
Semantic category assignemnt or classification of words into semantic fields 
is exploited in development of ontologies or various NLP applications such as 
word sense disambiguation or question-answering. In this paper we present a model that predicts 
semantic category of a word. The model is trained i na supervised manner
using a small training set of nouns that have semantic categories assigned. 
As features of machine learning \footnote{Initially, in this context of a Machine Learning the notion 'feature'
is used, but we will use 'atribute' instead so that it does not interfere with 'semantic feature' concept}.
we have chosen both morphological and syntactic properties of a noun.


\section{Data sources}

\subsection{Dictionary with semantic features - Ruslan}
\subsection{Monolingual corpus}


\section{Machine Learning}

This problem can be viewed as a feature assignment task or a 
classification task. %(this was a hint from Martin Holub)
When we assign a semntic feature to a noun, this produces a classification
of nouns into semantic fields.


\subsection{Logistic Regression}

\subsection{Experiment 1 - context}

\subsection{Experiment 2 - morphological characteristics of words}

\section{Evaluation}

\section{Conclusion}

\end{document}

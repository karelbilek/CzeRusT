%
% File naaclhlt2012.tex
%

\documentclass[11pt,letterpaper]{article}
\usepackage{vzor/naaclhlt2012}
\usepackage{times}
\usepackage{latexsym}
\setlength\titlebox{6.5cm}    % Expanding the titlebox
\usepackage[utf8]{inputenc}
\usepackage{float}

\title{Unknown words in Statistical Machine Translation between morphologically
rich and poor languages}

%\author{Author 1\\
%	    XYZ Company\\
%	    111 Anywhere Street\\
%	    Mytown, NY 10000, USA\\
%	    {\tt author1@xyz.org}
%	  \And
%	Author 2\\
% 	ABC University\\
% 	900 Main Street\\
% 	Ourcity, PQ, Canada A1A 1T2\\
%  {\tt author2@abc.ca}}

\date{}

\begin{document}
\maketitle
\begin{abstract}
In this paper we address the problem of unknown words in Statistical
Machine Translation (SMT) with respect to the morphological complexity of languages. 
We trained the Statistical Machine Translation system Moses
for Russian-to-English - translating from the morphologically rich to the
morphologically poor language - and Russian-to-Czech - the translation
between two morphologically rich related languages. After the analysis of
out-of-vocabulary word types, we show the ways to reduce the rate of out-of-vocabulary words (OOV), exploiting morphological analyzers
and stemming techniques, and discuss the relation of OOV and other metrics.
\end{abstract}

\section{Introduction}
The most frequent SMT errors lowering the translation quality
are untranslated words, called out-of-vocabulary words (or OOV) in this paper. Other errors (wrong morphological form of a word, syntax errors) make the translated text inconvenient to read, but still somehow understandable. But the unknown words are just kept as-is in the target language, thus giving the reader no information at all. Therefore, it is crucial to return at least some translation, even in a wrong word form.

Why can a word be OOV? In SMT, the word is kept untranslated if its form has not been seen in the training data. It might be the case of a completely out-of-domain word, but it might also be the case of another morphemic form of a word, already present in the training data. 

The latter presents a challenge for morphologically richer languages, such as the whole family of Slavic languages, where one word can have tens of morphemic forms.

Researchers have been improving the OOV rate, both disregarding and taking morphological richness of languages into account.

Some authors \cite{habash}, \cite{turchi}, \cite{bojartamchyna} address the problem of how to reduce the OOV 
rate suggesting
various techniques, as, for example, introducing morphological information or additional dictionary
resources. 

Exploiting the surface form of a word - division into morphemes, stemming - brought
positive results in terms of increasing the percentage of translated words 
especially when building a translation  model from and to morphologically rich languages \cite{popovic},
\cite{oflazer}, \cite{gispert}.  
%Very similar goals were 
%As our training and test data are of the same domain (news) the 
%http://cl.naist.jp/~kevinduh/papers/duh10analysis.pdf
Our approach mainly follows the line of research described above - making use of 
morphological resources and exploiting simple stemming techniques.

This paper also discusses the question of relation between language similarity and translation quality.

In the past, when the statistical models were not prevalent in machine translation and the main trend were rule-based systems, it was assumed that translation between related, but morphologically rich languages will be easier than between less related, but morphologically poorer one; for example, it was assumed that the translation between Czech and Russian will be easier, than between Czech and English.

Czech and Russian are both Slavic languages. They share a very similar morphological and syntactic
structure (declension types, word order) and the surface form of morphemes. These properties 
might have been useful for the rule-based machine translation. 

However, as we found out, this similarity surprisingly
 plays no role in the SMT, and, as we will show further,
the translation between Czech and Russian demonstrates lower quality output than 
between English and any of the two languages.

\section{Characterstics of Slavic languages}
Slavic languages are mostly inflecting languages characterizing by free word order and rich
inflectional paradigms. %(two facts depending on each other ). 
The table \ref{tab:slon} shows the exposion of word forms in Slavic languages on
the example of a noun phrase.
%Czech and Russian are very similar in 

\begin{table}
\begin{center}
\begin{tabular}{|l|l|l|}
\hline
English & Czech & Russian \\ \hline
\textit{jolly elephant} & veselý & veselyj slon \\
 & veselé slona & veselogo slona \\
 & veselé slonu & veselomu slonu \\
 & veselé slona & veselogo slona \\ 
 & veselé slonu & veselom slone\\
 & veselýonem& veselym slonom\\
 & veselýe &\\
\hline
\end{tabular}
\end{center}
\caption{Declension of a noun phrase.}
\label{tab:slon}
\end{table}

The above example of declension demonstrates the morphological complexity of Czech and Russian.

\section{Out-of-vocabulary words}
In statistical machine translation from and especially to the morphologically rich languages,
high out-of-vocabulary rate and mistakes in morphological forms are typical.
%Almost all works cited in the introduction 
%presented a research on a MT where one language of the translation pair was morphologically rich.
%We will focus our attention on languages

As demonstrated in the previous section, Slavic languages are very morphologically rich. This creates a problem of data sparseness that increase the number of out-of-vocabulary words(forms).

\subsection{Calculating OOV rate}
In some cases, it's hard to specifically count the OOV rate.

However, we specifically chose language pair, when one language is written in cyrillic, while the other is written in latin alphabet. This allows us to count OOV rate in a very efficient way, since cyrillic characters in non-cyrillic text almost surely mean untranslated cyrillic word and vice versa.

\subsection{Relation between BLEU and OOV rates}
We must always be reminded that decreasing OOV rate is not a goal in itself; we want to decrease it only to increase the translation quality.

For a trivial example - if we translate all words of the source language to one specific word in target language, we would have low OOV rate but also low BLEU score.

For this, we have to measure both the OOV rate and the translation quality, in our case by counting BLEU score.


\section{Decreasing the OOV rate with factors}
The basic idea for decreasing OOV rate was: we want to eliminate the cases, where we don't see a given word in training data in the source language, because it was there in a different form.

Therefore, we should take all the morphological forms of the given word on the source side and transform them to the same word (how exactly is described further), but we let the words on the target side be.

Because of that, we don't have to do any word generation (which is a positive), but on the other hand, the machine translation will never return a word, which don't appear on the target side in the training data (which is negative). We decided for this factored model because the generation models for Czech are not that well developed.

We then did two different experiments - first, we used just phrase tables created as described above, and then we used them only as a so-called backoff model (as defined in ???).

For better results, we also added tags to the target side and used them to build a language model.We describe this further.

We used two ways of transformation of the various forms of the same word to the same form - the lemmatization and stemming. We will describe them in the following subsections.

The whole factored set-up can be seen in the picture ???.

\subsection{Lemmatization, tagging}

%One of the ways to improve the out-of-vocabulary rate is using additional morphological
%information, the method that was successfully implemented for example by \cite{turchi},
%bringing a decrease of a OOV words without introducing more parallel data.

%Ja si nejsem jisty, jestli to, co delal Turchi je totez, co delame my.

We are exploiting the morphological features in two ways. First, we use lemmatization of source language for making OOV rates smaller. Also, we use tagging of the target languages for better language models.

For Czech and English, we used Morče - morphological perceptron-based tagger, developed at Charles' University in Prague. For Russian, we used TreeTagger - modular tagger, developed at University of Stuttgart, with the Russian parameter file developed by Serge Sharoff.

Because the Russian is always the source language in this case, the tagset is not relevant. For Czech, Morče uses the tagset, described in ???. For English, Morče uses ????.

\subsection{"Brute force" stemming}
In general, stemming is deriving a stem (root) for a word form in languages, where it makes sense.

In Slavic languages the inflection is done on the end of the word, so the stem is usually some beginning of the word form, unchanged.

There are some stemmers for all three languages that are trying to do this "properly". However, in our case, what we mean by stemming is just taking first $n$ letters of each word, with varying $n$ from 3 to 6 for different experiments.

We can argue that this stemming is really just "brute force" and should not give any good results. Surprisingly, the results are actually better than with the lemmatization.

%Karel: tohle je divne napsane, mazu to
%exploiting a stem(root) of a word is a primitive thus efficient technique to support OOV words guessing.
%Especially for the agglutinative languages stemming can bring some fruit because
%each morphemic category is related one-to-one to its surface form%(is it really so??)
%Czech and Russian are flective languages, so they combine the morphemes by fusion/flexion,
%not just putting it one after another. So for instance, if a substantive in Czech has categories
% number, gender and case, the morphemes presenting those categories will be represented only
%by one morpheme-ending. 
%As we tried to use the maximum of baseline data, we decided to derive stems from the words
%without using any additional morphological information like the list of word endlings that are to be eliminated.
%The technique is primitive - it presents taking the first n characters of a word and then selecting the optimal
%length of a stem that bring the better improvement of a Bleu score and OOV rate.
%The example of a stemmed text:\\
%En: \textit{the$|$the gaza$|$gaza cease$|$cease -$|$- fire$|$fire should$|$shoul be$|$be allowed$|$allow to$|$to facilitate$|$facil 
%reconciliation$|$recon between$|$betwe fatah$|$fatah and$|$and hamas$|$hamas}
%%We first tested it for Russian-to-Czech translation.
%The setup of this experiment is the same as the previous - factored, where stems are used instead of lemmas,
%the results are shown in Table \ref{tab:stem}.



%First we opted for taggers that are available  on-line. Those taggers
%(Morce for Czech, TreeTagger for Russian and English) assigned each word form with
%a lemma and a tag. 

%???????? opravdu??? je blbost jenom snizovat OOV.... 

%As our main task on the current stage was only to check
%how much words will be translated properly, we are more interested in increasing
%the OOV rate than a BLEU score. The latter is not supposed to be that 
%good for the evaluating translation into morphologically rich languages that often have
%free word order. Still, it will serves the purpose of comparison the translation 
%quality into the same language (we have chosen Czech as a target) under the same conditions(training data).

\section{Statistical Machine Translation setup}
%Statistical Machine Translation nowadays has become one of the easiest and cheapest paradigms of the MT systems. 
%To jsou proste vyplnove reci, mam pocit...
%Researchers can now usevarious toolkits to experiment with different language pairs.

\subsection{Moses}
From the various available tools, we chose to experiment with Moses, an open-source implementation of phrase-based statistical translation system.

%\subsection{Moses}
The Moses toolkit \cite{moses} is a complex system which includes many components
for data preprocessing and MT evaluation, for example
GIZA++ %\footnote{http://www.fjoch.com/GIZA++.html}
involved in finding word alignment, the SRI Language Modeling
Toolkit %\footnote{http://www.speech.sri.com/projects/srilm/} 
and the built-in implementation of model optimization (Minimum Error Rate Training, MERT) on a
given development set of sentences.
%We included direct translation as a basic setting.
%There are still weights
%combining phrase and language model into a log-linear model to be mentioned.
%They are tuned by Minimum error rate training (MERT) algorithm which is driven
%by BLEU feedback scored on held-out sentences.


Factors have been described in the section ???.

\subsection{Data}
Phrase-based SMT systems need huge amount of parallel data in order to
extract dictionaries of phrases and their translations, so called phrase tables.
In our work we exploited data from a parallel Czech-English-Russian
corpus called UMC (UFAL Multilingual Corpus) with automatic pairwise sentence
alignment. The texts were downloaded from from the Project
Syndicate\footnote{http://www.project-syndicate.org/} page. 
The data are divided into three sets: training set(train),
development set(dev) and test set.
The statistics of the data are summarized in the Table \ref{tab:corpus}.

As we can see, the number of sentences is slightly different; it is because the sentences are not always aligned one-to-one, but are often ????\footnote{TODO: najít přesně!!!}
\begin{table}

\begin{center}
\begin{tabular}{lcr}
  &  Languages & Sentences \\
\hline
Language Model & cs & 92,233 \\
Translation Model & ru $\rightarrow$ cs &  93395\\
Translation Model & en $\rightarrow$ cs &  92775\\ \hline %TODO ru-en
Dev     & cs, en, ru          &    765 \\
Test     & cs, en, ru          &  2000 \\
\hline
\end{tabular}
\end{center}
\caption{Parallel corpus size.}
\label{tab:corpus}
\end{table}



\subsection{Statistics of OOV words for simple models} 

Following is the table that demonstrates the correlation of bleu score and 
the oov rate for different type of languages. 
The oov rate was calculated 
rather in a primitive way - we inspected the translation output for alien characters. The words that contained
cyrillic alphabet letters were considered to be 'unknown' within the Czech or English text.
And otherwise, the latin characters in the Russian output text signalized in the majority of cases 
the out-of-vocabulary word. \footnote{These numbers are not very precise - words in latin within Russian text
can be just terms or proper names(like linux, Java, USA etc.) that can be tolerable in Russian text }

 
\begin{table*}
\begin{center}
\begin{tabular}{c c c}
\hline
translation pair& bleu & OOV \\
\hline
%cs $\rightarrow$ ru & 12\% & 7\% \\%s.translate.8df
ru $\rightarrow$ cs & 11\% & 6\% \\%\hline
ru $\rightarrow$  en & 15\% & 8\% \\ %TODO ru-en
%ru $\rightarrow$  en YAN & 22 & 1\%  \\
%en-ru YANDEX-data & 16 & 2.7\% \\
\hline
\end{tabular}
\end{center}
\caption{BLEU score for simple model - baseline.}
\label{tab:umc_yan}
\end{table*}

As we can see from the table, the morphological properties of languages seems to
affect the bleu score and the oov rate differently - in a rather predictable way though.
In the translation into English the oov rate was minimal. 
Bleu score is bound to the OOV rate, the more is the bleu  score, the less unknown words occur 
in the translated text.
We also tried to see if language type has some impact on the OOV rate, and it did not.
The only factor that mattered was the type of data - domain, size and quality. When 
trained on the corpus UMC with news thematics(100,000 sentences) the OOV rate was rather high.

\section{Using morphological analyzers to improve the translation of unknown words}

One of the ways to improve the out-of-vocabulary rate is using additional morphological
information, the method that was successfully implemented for example by \cite{turchi},
bringing a decrease of a OOV words without introducing more parallel data.
 First we opted for taggers that are available  on-line. Those taggers
(Morce for Czech, TreeTagger for Russian and English) assigned each word form with
a lemma and a tag. As our main task on the current stage was only to check
how much words will be translated properly, we are more interested in increasing
the OOV rate than a BLEU score. The latter is not supposed to be that 
good for the evaluating translation into morphologically rich languages that often have
free word order. Still, it will serves the purpose of comparison the translation 
quality into the same language (we have chosen Czech as a target) under the same conditions(training data).

In order to train a factored model we tagged and lemmatized the UMC corpus with
the help of TreeTagger %\cite{Treetagger} 
for English and Russian and Morce
morphological tagger for Czech. %\urlmorce
Each word form is assigned by a lemma and a morphological tag as described below:

\begin{figure}
Cz: \textit{Informace$|$informace$|$NNFP1-{}-{}-{}-{}-{}-{}-A-{}-{}- o$|$o-1$|$RR6-{}-{}-{}-{}-{}-{}- 
pástáké$|$pástákýIS6-{}-{}-{}-{}-1A-{}-{}- jaderné$|$jadernýIS61A-{}-{}-{}-{}-{}- 
programu$|$program-1$|$NNIS6-{}-{}-{}-{}-{}-{}-A-{}-{}-}\\
%{prostì$prostì$Dg-{}-{}-{}-{}-{}-{}-1A-{}-{}-{}- jsem$|$býB-S-{}-{}-1P-AA-{}-{}- brala$|$brá|$VpQW-{}-{}-XR-AA-{}-{}-}\\
%{\bf Ru:} \textit{??????????$|$??????????$|$Ncfsnn ?$|$?$|$Sp-l ???????$|$???????$|$Afpfslf ?????????$|$?????????$|$Ncfsln ?????????$|$????????$|$Ncms
%gn}\\
En: \textit{The$|$the$|$DT visionaries$|$visionary$|$NNS would$|$would$|$MD have$|$have$|$VH gotten$|$get$|$VVN nowhere$|$nowhere$|$RB}
\caption{Facored corpus, tagsets from TreeTagger(En) and Morce(Cs)}
\label{fig:fact}
\end{figure}

Our second experiment using the factored data is of a more complex structure.
The word alignment is made on lemmas so that various forms of the same word
were aligned, in the contrast to the simple model. We built two phrase tables:
first one contained the mapping lemma $\rightarrow$ form + tag, the second one 
form $\rightarrow$ form +tag. Then we constructed the language model for forms
and tags. The results of the experiment in terms of BLEU score and OOV rate are 
summarized 


\section{Stemming}
Stemming - exploiting a stem(root) of a word is a primitive thus efficient technique to support OOV words guessing.
Especially for the agglutinative languages stemming can bring some fruit because
each morphemic category is related one-to-one to its surface form%(is it really so??)
Czech and Russian are flective languages, so they combine the morphemes by fusion/flexion,
not just putting it one after another. So for instance, if a substantive in Czech has categories
 number, gender and case, the morphemes presenting those categories will be represented only
by one morpheme-ending. 
As we tried to use the maximum of baseline data, we decided to derive stems from the words
without using any additional morphological information like the list of word endlings that are to be eliminated.
The technique is primitive - it presents taking the first n characters of a word and then selecting the optimal
length of a stem that bring the better improvement of a Bleu score and OOV rate.
The example of a stemmed text:\\
En: \textit{the$|$the gaza$|$gaza cease$|$cease -$|$- fire$|$fire should$|$shoul be$|$be allowed$|$allow to$|$to facilitate$|$facil 
reconciliation$|$recon between$|$betwe fatah$|$fatah and$|$and hamas$|$hamas}
%We first tested it for Russian-to-Czech translation.
The setup of this experiment is the same as the previous - factored, where stems are used instead of lemmas,
the results are shown in Table \ref{tab:stem}.

\begin{table}
\begin{center}
\begin{tabular} {c c c}
\hline
stem length&BLEU&OOV\\
\hline
6 & 12.04&1.8\%\\
\bf{5} & \bf{12.22} &1.1\%\\
4 & 11.04& 0.6\%\\
3 & 11.99& 0.1\%\\
\hline
\end{tabular}
\caption{BLEU score for models on stems with different length.}
\label{tab:stem}
\end{center}
\end{table}

The alignment on stems that are 3 characters long brought the lowest OOV rate,
but we can not trust enough the unknown words that were guessed with this step.
The optimal number of characters selected as stems for a translation into a morphologically rich 
language was 5, so we applied it to other language pairs.
We examined the unknown words for the experimental setup stem-5, and it appeared, that it contained either 
rarely used named entities, less frequent spelling variants, typos, and a minimum of meaningful words. 


\subsection{Results}

In order to see which technique was more efficient for our 
task we compared all the experiments -simple, factored on lemmas and stems. 
described above. The results are shown in Table \ref{tab:overall}.
\begin{table}
\begin{center}
\begin{tabular} {ccccccc}
\hline
%lang pair & simple  &factored &stemmed-5\\\hline
lang pair &\multicolumn{2}{c}{Simple model} & \multicolumn{2}{c}{Factored-lemma} & \multicolumn{2}{c}{Factored-stem}\\ \hline
                       & BLEU   &OOV    & BLEU   & OOV & BLEU  & OOV \\ \hline
ru $\rightarrow$ cs    & 11.14  &6.41   &11.68   &2.81 & 12.22 &1.19 \\ %08b6, 9e688, a0f5
en $\rightarrow$ cs    & 14.58  &4.67   &15.49   &3.11 & 15.39  &3.47  \\ \hline %instead TODO ru-enru-ensimple: a8290, ru-en lenmma??? ru-en stem
\end{tabular}
\end{center}
\caption{Overall evaluation.}
\label{tab:overall}
\end{table}

It became evident, that our techniques to improve the translation quality help especially in the
case of MT between morphologically rich languages. The score for English-Czech translation, both simple
or factored, was higher than Russian-Czech, but have not gained much improvement when 
factored models were introduced. 

\section{Statistical Machine Translation setup}
Statistical Machine Translation nowadays has become one of the easiest and cheapest paradigms of the MT systems. 
%Researchers can now usevarious toolkits to experiment with different language pairs.
From the various available tools, we chose to experiment with Moses, an open-source implementation of phrase-based statistical translation system.

\subsection{Moses}
The Moses toolkit \cite{moses} is a complex system which includes many components
for data preprocessing and MT evaluation, for example
GIZA++ %\footnote{http://www.fjoch.com/GIZA++.html}
involved in finding word alignment, the SRI Language Modeling
Toolkit %\footnote{http://www.speech.sri.com/projects/srilm/} 
and the built-in implementation of model optimization (Minimum Error Rate Training, MERT) on a
given development set of sentences.
%We included direct translation as a basic setting.
%There are still weights
%combining phrase and language model into a log-linear model to be mentioned.
%They are tuned by Minimum error rate training (MERT) algorithm which is driven
%by BLEU feedback scored on held-out sentences.

To establish a baseline for further experiments, we trained translation models for direct translation
from Russian to Czech
(\emph{ru$\rightarrow$cs simple}) and Russian to English (\emph{ru$\rightarrow$en simple}), %TODO -1 udelame radsi en-ru, kde bude OOV jasne videt.
optimizing them on the development set.

In our second experiment, we used the so-called factored models. Factored translation is an extension of the basic translation model, where on both the source and the target side there doesn't have to be just form, but we can enrich the forms with some more information. We will touch on the specific factors later.


%In our second experiment we have trained a factored model.
%While the first one is based on pure data from a parallel corpus, the second one
%uses morphology to improve the out-of-vocabulary rate.
%%in order to avoid a problem caused by data sparseness. 
%Factored translation is an extention of the basic translation model,
%where each word form from the parallel corpus is enriched with a lemma and a tag.


%Although we could have
%used additional parallel data (English-Russian or English-Czech) to train the translation models, or
%add some huge monolingual corpus to train a language model.
%but we need English-Russian and Russian-Czech corpus to be comparable, so we used only this data.
\section{Conclusion}
In this paper we have shown two ways to improve the translation quality and lower out-of-vocabulary rate:
with the help of lemmatizing and stemming. These models have shown the slightest improvement in terms of BLEU
score and a considerable decrease of out-of-vocabulary words especially for the morphologically rich languages. 
The OOV rate for the translation between Czech and Russian reduced 2 times(lemma model) and 5 times(stem model) 
against the baseline. The improvement in terms of OOV for English-Czech translation was not
significant and the BLEU score has not changed a lot as well.
%Our future plans are to exploit more data
%and this improvement was even more significant than for the respective rate for English-to-Czech translation. 


\begin{thebibliography}{4} 

%\bibitem{baltic} Raivis Skadin¹, Karlis Goba, and Valters ©ics: Improving SMT for Baltic Languages with Factored Models. 
%In: Proceedings of the 2010 conference on Human Language Technologies, 125-132, 2010.

\bibitem{habash} Habash, N.: Four techniques for online handling of out-of-vocabulary words in 
Arabic-English statistical machine translation. In: Proceedings of the 46th Annual Meeting of 
the Association for Computational Linguistics on Human Language Technologies:
%Short Papers (HLT-Short '08). Association for Computational Linguistics, 
Stroudsburg, PA, USA, 57-60.

\bibitem{popovic} Popovic,M., Hermann, N.: Towards the Use of Word Stems and Suffixes for Statistical Machine Translation. 
In: Proceedings of 4th International Conference on Language Resources and Evaluation (LREC), Lisbon, Portugal, May 2004 

\bibitem{oflazer} Oflazer, K.:Statistical Machine Translation into a Morphologically Complex Language. 
In: Proceedings of CICLing 2008, Haifa, Israel, February 17-23, 2008. 

\bibitem{gispert} Gispert,A., Marino,J, Crego, J: Improving statistical machine translation by classifying and generalizing inflected verb forms.  
In: Proceedings of 9th European Conference on Speech Communication and Technology

\bibitem{turchi} Turchi, M., Ehrmann, M.: Knowledge Expansion of a Statistical Machine Translation System using Morphological Resources. 
In: Polibits, (43), 37-43, 2011.

\bibitem{bojartamchyna} Bojar,O., Tamchyna, A.: Forms Wanted: Training SMT on Monolingual Data.
In: Proceedings Research Workshop of the Israel Science Foundation University of Haifa, Israel. 2011.

\bibitem{moses}Koehn, P., H. Hoang, A. Birch et al: Moses: Open source toolkit for statistical machine translation.
In: Proceeding ACL '07 Proceedings of the 45th Annual Meeting of the ACL, pp. 177-180, ACL, 2007.

\end{thebibliography}


\end{document}


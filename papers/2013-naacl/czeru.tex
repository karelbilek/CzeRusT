%
% File naaclhlt2012.tex
%

\documentclass[11pt,letterpaper]{article}
\usepackage{naaclhlt2012}
\usepackage{times}
\usepackage{latexsym}
\setlength\titlebox{6.5cm}    % Expanding the titlebox
\usepackage[utf8]{inputenc}
\usepackage{float}

\title{Unknown words in Statistical Machine Translation between morphologically
rich and poor languages}

%\author{Author 1\\
%	    XYZ Company\\
%	    111 Anywhere Street\\
%	    Mytown, NY 10000, USA\\
%	    {\tt author1@xyz.org}
%	  \And
%	Author 2\\
% 	ABC University\\
% 	900 Main Street\\
% 	Ourcity, PQ, Canada A1A 1T2\\
%  {\tt author2@abc.ca}}

\date{}

\begin{document}
\maketitle
\begin{abstract}
In this paper we address the problem of unknown words in Statistical
Machine Translation(SMT) with the respect to the morphological complexity of
the languages. We trained the Statistical Machine Translation system Moses
for English-to-Russian - translating from the morphologically poor to the
morphologically rich language and Czech-to-Russian - the translation
between two morphologically rich related languages. After the analysis of
out-of-vocabulary word types we show the ways to improve the translation
of words not seen in the training data, exploiting morphological analyzers
and stemming techniques.
\end{abstract}

\section{Introduction}
The most frequent SMT errors that impede understanding a text 
are untranslated words (unknown words, out-of-vocabulary - OOV words). Some other mistakes - like
wrong morphological form of a word or syntax errors - makes a text inconvinient to read, but one
can still get a sense out of it, whereas the unknown words in other language gives us no information at all.
So it is crucial to give at least some translation, even in a wrong word form. % so that the chances to understand the text increase 
In order to decrease the OOV rate we should introspect into the main reasons of why this happens. 
The reason is that the word has not been seen in the training data, it may be completely out-of-domain
or the word was not found in the training data, though it can occur in some other morphemic form.
The latter presents a challenge especially when translating to/from morphologically rich language
\footnote{We use the term morphologically rich/poor in this context only comparatively, as 
even more morphologically 'poor' languages than English exist}.
where the number of unique word forms is very large. %in comparison to that of morphol
Many researchers have been improving the OOV rate both disregarding the morphological type
of the languages involved or either taking into account morphological properties of languages. 
%There are several techniques generally applied to our problem. 
%The first one is domain adaptation.
%The large percentage of the unknown words come from a domain different than training data
%so the ways to handle out-of-domain words were suggested in 
Some authors \cite{habash}, \cite{turchi}, \cite{bojartamchyna} address the problem of how to reduce the OOV 
rate suggesting
various techniques, as, for example, introducing morphological information or inroducing additional dictionary
resources. 
Exploiting the surface form of a word - like division into morphemes, stemming - brought
positive results in terms of increasing the percentage of translated words 
especially when building a translation  model from/to morphologically rich languages \cite{popovic},
\cite{oflazer}, \cite{gispert}.  
%Very similar goals were 
%As our training and test data are of the same domain (news) the 
%http://cl.naist.jp/~kevinduh/papers/duh10analysis.pdf
Our approach mainly follows the line of research described above - making use of 
morphological resources and exploiting simple stemming technique.

The paper also discusses the question if relatedness of the languages influence
the translation quality. Some decades ago when the statistical models were not so 
wide-spread and the rule-based systems were developed instead, it was stated that the 
Machine Translation between the related languages has been much more easy and the
translation quality has been higher. %This proved to be true for example for MT 
%between Czech and Slovak, but for more distant languages it was much worse.
As for Czech and Russian, the languages share the very similar morphological and syntactic
structure (like declension types, word order) and the surface form of morphemes.These properties 
might be useful for the rule-based machine translation. This similarity
however plays no role in the SMT, and even %the morphological complexity of 
the SMT between related Czech and Russian demonstrates lower quality output than 
those between typologically different English and either of the languages.

\section{Statistical Machine Translation setup}
Statistical Machine Translation nowadays has become one
of the easiest and cheapest paradigms of the MT systems. Researchers can now use
various toolkits to experiment with different language pairs. We experiment
with Moses , an open-source implementation of phrase-based statistical translation system.

\subsection{Moses}
The Moses toolkit \cite{moses} is a complex system which includes many components
for data preprocessing and MT evaluation, for example
GIZA++ %\footnote{http://www.fjoch.com/GIZA++.html}
involved in finding word alignment, the SRI Language Modeling
Toolkit %\footnote{http://www.speech.sri.com/projects/srilm/} 
and the built-in implementation of model optimization (Minimum Error Rate Training, MERT) on a
given development set of sentences.
%We included direct translation as a basic setting.
%There are still weights
%combining phrase and language model into a log-linear model to be mentioned.
%They are tuned by Minimum error rate training (MERT) algorithm which is driven
%by BLEU feedback scored on held-out sentences.
To establish a baseline, we trained translation models for direct translation
from Russian to Czech
(ru$\rightarrow$cs simple) and English to Czech (en$\rightarrow$cs simple), %TODO -1 udelame radsi en-ru, kde bude OOV jasne videt.
optimizing them on the development set.
in our second experiment we have trained a factored model.
While the first one is based on pure data from a parallel corpus, the second one
uses morphology to improve the out-of-vocabulary rate.
%in order to avoid a problem caused by data sparseness. 
Factored translation is an extention of the basic translation model,
where each word form from the parallel corpus is enriched with a lemma and a tag.

\subsection{Data}
Phrase-based SMT systems need huge amount of parallel data in order to
extract dictionaries of phrases and their translations, so called phrase tables.
In our work we exploited data from a parallel Czech-English-Russian
corpus called UMC (UFAL Multilingual Corpus) with automatic pairwise sentence
alignment. The texts were downloaded from from the Project
Syndicate\footnote{http://www.project-syndicate.org/} page. 
The data are divided into three sets: training set(train),
development set(dev) and test set.
The statistics of the data are summarized in the Table \ref{tab:corpus}.
\begin{table}

\begin{center}
\begin{tabular}{lcr}
  &  Languages & Sentences \\
\hline
Language Model & cs & 92,233 \\
Translation Model & ru $\rightarrow$ cs &  93395\\
Translation Model & en $\rightarrow$ cs &  92775\\ \hline %TODO ru-en
Dev     & cs, en, ru          &    765 \\
Test     & cs, en, ru          &  2000 \\
\hline
\end{tabular}
\end{center}
\caption{Parallel corpus size.}
\label{tab:corpus}
\end{table}

%Although we could have
%used additional parallel data (English-Russian or English-Czech) to train the translation models, or
%add some huge monolingual corpus to train a language model.
%but we need English-Russian and Russian-Czech corpus to be comparable, so we used only this data.

\section{Out-of-vocabulary words} 
High out-of-vocabulary rate and mistakes in morphological forms
are most typical of translating from and especially to 
the morphologically rich languages. Almost all works cited in the introduction 
presented a research on a MT where one language of the translation pair was morphologically rich.
Slavic languages are mostly inflecting languages characterizing by free word order and rich
inflectional paradigms. %(two facts depending on each other ). 
The table \ref{tab:slon} shows the exposion of word forms in Slavic languages on
the example of a noun phrase.
%Czech and Russian are very similar in 

\begin{table}
\begin{center}
\begin{tabular}{|l|l|l|}
\hline
English & Czech & Russian \\ \hline
\textit{jolly elephant} & veselý & veselyj slon \\
 & veselé slona & veselogo slona \\
 & veselé slonu & veselomu slonu \\
 & veselé slona & veselogo slona \\ 
 & veselé slonu & veselom slone\\
 & veselýonem& veselym slonom\\
 & veselýe &\\
\hline
\end{tabular}
\end{center}
\caption{Declension of a noun phrase.}
\label{tab:slon}
\end{table}

The above example of declension demonstrates the morphological complexity of Czech and Russian.
This creates a problem of data sparseness that increase the number of out-of-vocabulary words(forms).

\subsection{Statistics of OOV words for simple models} 

Following is the table that demonstrates the correlation of bleu score and 
the oov rate for different type of languages. 
The oov rate was calculated 
rather in a primitive way - we inspected the translation output for alien characters. The words that contained
cyrillic alphabet letters were considered to be 'unknown' within the Czech or English text.
And otherwise, the latin characters in the Russian output text signalized in the majority of cases 
the out-of-vocabulary word. \footnote{These numbers are not very precise - words in latin within Russian text
can be just terms or proper names(like linux, Java, USA etc.) that can be tolerable in Russian text }

 
\begin{table*}
\begin{center}
\begin{tabular}{c c c}
\hline
translation pair& bleu & OOV \\
\hline
%cs $\rightarrow$ ru & 12\% & 7\% \\%s.translate.8df
ru $\rightarrow$ cs & 11\% & 6\% \\%\hline
ru $\rightarrow$  en & 15\% & 8\% \\ %TODO ru-en
%ru $\rightarrow$  en YAN & 22 & 1\%  \\
%en-ru YANDEX-data & 16 & 2.7\% \\
\hline
\end{tabular}
\end{center}
\caption{BLEU score for simple model - baseline.}
\label{tab:umc_yan}
\end{table*}

As we can see from the table, the morphological properties of languages seems to
affect the bleu score and the oov rate differently - in a rather predictable way though.
In the translation into English the oov rate was minimal. 
Bleu score is bound to the OOV rate, the more is the bleu  score, the less unknown words occur 
in the translated text.
We also tried to see if language type has some impact on the OOV rate, and it did not.
The only factor that mattered was the type of data - domain, size and quality. When 
trained on the corpus UMC with news thematics(100,000 sentences) the OOV rate was rather high.

\section{Using morphological analyzers to improve the translation of unknown words}

One of the ways to improve the out-of-vocabulary rate is using additional morphological
information, the method that was successfully implemented for example by \cite{turchi},
bringing a decrease of a OOV words without introducing more parallel data.
 First we opted for taggers that are available  on-line. Those taggers
(Morce for Czech, TreeTagger for Russian and English) assigned each word form with
a lemma and a tag. As our main task on the current stage was only to check
how much words will be translated properly, we are more interested in increasing
the OOV rate than a BLEU score. The latter is not supposed to be that 
good for the evaluating translation into morphologically rich languages that often have
free word order. Still, it will serves the purpose of comparison the translation 
quality into the same language (we have chosen Czech as a target) under the same conditions(training data).

In order to train a factored model we tagged and lemmatized the UMC corpus with
the help of TreeTagger %\cite{Treetagger} 
for English and Russian and Morce
morphological tagger for Czech. %\urlmorce
Each word form is assigned by a lemma and a morphological tag as described below:

\begin{figure}
Cz: \textit{Informace$|$informace$|$NNFP1-{}-{}-{}-{}-{}-{}-A-{}-{}- o$|$o-1$|$RR6-{}-{}-{}-{}-{}-{}- 
pástáké|$pástákýIS6-{}-{}-{}-{}-1A-{}-{}- jaderné|$jadernýIS61A-{}-{}-{}-{}-{}- 
programu$|$program-1$|$NNIS6-{}-{}-{}-{}-{}-{}-A-{}-{}-}\\
%{prostì$prostì$Dg-{}-{}-{}-{}-{}-{}-1A-{}-{}-{}- jsem$|$býB-S-{}-{}-1P-AA-{}-{}- brala$|$brá|$VpQW-{}-{}-XR-AA-{}-{}-}\\
%{\bf Ru:} \textit{??????????$|$??????????$|$Ncfsnn ?$|$?$|$Sp-l ???????$|$???????$|$Afpfslf ?????????$|$?????????$|$Ncfsln ?????????$|$????????$|$Ncms
%gn}\\
En: \textit{The$|$the$|$DT visionaries$|$visionary$|$NNS would$|$would$|$MD have$|$have$|$VH gotten$|$get$|$VVN nowhere$|$nowhere$|$RB}
\caption{Facored corpus, tagsets from TreeTagger(En) and Morce(Cs)}
\label{fig:fact}
\end{figure}

Our second experiment using the factored data is of a more complex structure.
The word alignment is made on lemmas so that various forms of the same word
were aligned, in the contrast to the simple model. We built two phrase tables:
first one contained the mapping lemma $\rightarrow$ form + tag, the second one 
form $\rightarrow$ form +tag. Then we constructed the language model for forms
and tags. The results of the experiment in terms of BLEU score and OOV rate are 
summarized 


\section{Stemming}
Stemming - exploiting a stem(root) of a word is a primitive thus efficient technique to support OOV words guessing.
Especially for the agglutinative languages stemming can bring some fruit because
each morphemic category is related one-to-one to its surface form%(is it really so??)
Czech and Russian are flective languages, so they combine the morphemes by fusion/flexion,
not just putting it one after another. So for instance, if a substantive in Czech has categories
 number, gender and case, the morphemes presenting those categories will be represented only
by one morpheme-ending. 
As we tried to use the maximum of baseline data, we decided to derive stems from the words
without using any additional morphological information like the list of word endlings that are to be eliminated.
The technique is primitive - it presents taking the first n characters of a word and then selecting the optimal
length of a stem that bring the better improvement of a Bleu score and OOV rate.
The example of a stemmed text:\\
En: \textit{the$|$the gaza$|$gaza cease$|$cease -$|$- fire$|$fire should$|$shoul be$|$be allowed$|$allow to$|$to facilitate$|$facil 
reconciliation$|$recon between$|$betwe fatah$|$fatah and$|$and hamas$|$hamas}
%We first tested it for Russian-to-Czech translation.
The setup of this experiment is the same as the previous - factored, where stems are used instead of lemmas,
the results are shown in Table \ref{tab:stem}.

\begin{table}
\begin{center}
\begin{tabular} {c c c}
\hline
stem length&BLEU&OOV\\
\hline
6 & 12.04&1.8\%\\
\bf{5} & \bf{12.22} &1.1\%\\
4 & 11.04& 0.6\%\\
3 & 11.99& 0.1\%\\
\hline
\end{tabular}
\caption{BLEU score for models on stems with different length.}
\label{tab:stem}
\end{center}
\end{table}

The alignment on stems that are 3 characters long brought the lowest OOV rate,
but we can not trust enough the unknown words that were guessed with this step.
The optimal number of characters selected as stems for a translation into a morphologically rich 
language was 5, so we applied it to other language pairs.
We examined the unknown words for the experimental setup stem-5, and it appeared, that it contained either 
rarely used named entities, less frequent spelling variants, typos, and a minimum of meaningful words. 


\subsection{Results}

In order to see which technique was more efficient for our 
task we compared all the experiments -simple, factored on lemmas and stems. 
described above. The results are shown in Table \ref{tab:overall}.
\begin{table}
\begin{center}
\begin{tabular} {ccccccc}
\hline
%lang pair & simple  &factored &stemmed-5\\\hline
lang pair &\multicolumn{2}{c}{Simple model} & \multicolumn{2}{c}{Factored-lemma} & \multicolumn{2}{c}{Factored-stem}\\ \hline
                       & BLEU   &OOV    & BLEU   & OOV & BLEU  & OOV \\ \hline
ru $\rightarrow$ cs    & 11.14  &6.41   &11.68   &2.81 & 12.22 &1.19 \\ %08b6, 9e688, a0f5
en $\rightarrow$ cs    & 14.58  &4.67   &15.49   &3.11 & 15.39  &3.47  \\ \hline %instead TODO ru-enru-ensimple: a8290, ru-en lenmma??? ru-en stem
\end{tabular}
\end{center}
\caption{Overall evaluation.}
\label{tab:overall}
\end{table}

It became evident, that our techniques to improve the translation quality help especially in the
case of MT between morphologically rich languages. The score for English-Czech translation, both simple
or factored, was higher than Russian-Czech, but have not gained much improvement when 
factored models were introduced. 


\section{Conclusion}
In this paper we have shown two ways to improve the translation quality and lower out-of-vocabulary rate:
with the help of lemmatizing and stemming. These models have shown the slightest improvement in terms of BLEU
score and a considerable decrease of out-of-vocabulary words especially for the morphologically rich languages. 
The OOV rate for the translation between Czech and Russian reduced 2 times(lemma model) and 5 times(stem model) 
against the baseline. The improvement in terms of OOV for English-Czech translation was not
significant and the BLEU score has not changed a lot as well.
%Our future plans are to exploit more data
%and this improvement was even more significant than for the respective rate for English-to-Czech translation. 


\begin{thebibliography}{4} 

%\bibitem{baltic} Raivis Skadin¹, Karlis Goba, and Valters ©ics: Improving SMT for Baltic Languages with Factored Models. 
%In: Proceedings of the 2010 conference on Human Language Technologies, 125-132, 2010.

\bibitem{habash} Habash, N.: Four techniques for online handling of out-of-vocabulary words in 
Arabic-English statistical machine translation. In: Proceedings of the 46th Annual Meeting of 
the Association for Computational Linguistics on Human Language Technologies:
%Short Papers (HLT-Short '08). Association for Computational Linguistics, 
Stroudsburg, PA, USA, 57-60.

\bibitem{popovic} Popovic,M., Hermann, N.: Towards the Use of Word Stems and Suffixes for Statistical Machine Translation. 
In: Proceedings of 4th International Conference on Language Resources and Evaluation (LREC), Lisbon, Portugal, May 2004 

\bibitem{oflazer} Oflazer, K.:Statistical Machine Translation into a Morphologically Complex Language. 
In: Proceedings of CICLing 2008, Haifa, Israel, February 17-23, 2008. 

\bibitem{gispert} Gispert,A., Marino,J, Crego, J: Improving statistical machine translation by classifying and generalizing inflected verb forms.  
In: Proceedings of 9th European Conference on Speech Communication and Technology

\bibitem{turchi} Turchi, M., Ehrmann, M.: Knowledge Expansion of a Statistical Machine Translation System using Morphological Resources. 
In: Polibits, (43), 37-43, 2011.

\bibitem{bojartamchyna} Bojar,O., Tamchyna, A.: Forms Wanted: Training SMT on Monolingual Data.
In: Proceedings Research Workshop of the Israel Science Foundation University of Haifa, Israel. 2011.

\bibitem{moses}Koehn, P., H. Hoang, A. Birch et al: Moses: Open source toolkit for statistical machine translation.
In: Proceeding ACL '07 Proceedings of the 45th Annual Meeting of the ACL, pp. 177-180, ACL, 2007.

\end{thebibliography}


\end{document}


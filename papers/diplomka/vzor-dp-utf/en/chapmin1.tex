% Trochu volnější nastavení dělení slov, než je default.
\lefthyphenmin=2
\righthyphenmin=2

%%% Titulní strana práce

\pagestyle{empty}
\begin{center}

\large

Charles University in Prague

\medskip

Faculty of Mathematics and Physics

\vfill

{\bf\Large MASTER THESIS}

\vfill

\centerline{\mbox{\includegraphics[width=60mm]{../img/logo.pdf}}}

\vfill
\vspace{5mm}

{\LARGE Karel Bílek}

\vspace{15mm}

%% Název práce přesně podle zadání
{\LARGE\bfseries A Comparison of Methods of Czech-to-Russian Machine Translation}

\vfill

% Název katedry nebo ústavu, kde byla práce oficiálně zadána
% (dle Organizační struktury MFF UK)
%Ústav formální a aplikované lingvistiky
Institute of Formal and Applied Linguistics

\vfill

\begin{tabular}{rl}

Supervisor of the master thesis: & doc. RNDr. Vladislav Kuboň, Ph.D. \\
\noalign{\vspace{2mm}}
Study programme: & Informatics \\
\noalign{\vspace{2mm}}
Specialization: & Mathematical Linguistics \\
\end{tabular}

\vfill

% Zde doplňte rok
Prague 2014

\end{center}

\newpage

%%% Následuje vevázaný list -- kopie podepsaného "Zadání diplomové práce".
%%% Toto zadání NENÍ součástí elektronické verze práce, nescanovat.

%%% Na tomto místě mohou být napsána případná poděkování (vedoucímu práce,
%%% konzultantovi, tomu, kdo zapůjčil software, literaturu apod.)

\openright

\noindent

Those are all the people that I want to thank:

\begin{itemize}
\item Vladislav Kuboň -- my supervisor, for helping me with the thesis in general, and the related projects
\item Natalia Klyueva -- a college which I worked the most with
\item Ondřej Bojar and Aleš Tamchyna -- for helping me with the Moses system and eman evaluation manager
\item Martin Popel -- for helping me with TectoMT system and gave me lots of ideas in general
\item Rudolf Rosa -- for giving me ideas for better text structuring
\item countless friends that supported me when I just wanted to give up.
\end{itemize}

\newpage

%%% Strana s čestným prohlášením k diplomové práci

\vglue 0pt plus 1fill

\noindent
I declare that I carried out this master thesis independently, and only with the cited
sources, literature and other professional sources.

\medskip\noindent
I understand that my work relates to the rights and obligations under the Act No.
121/2000 Coll., the Copyright Act, as amended, in particular the fact that the Charles
University in Prague has the right to conclude a license agreement on the use of this
work as a school work pursuant to Section 60 paragraph 1 of the Copyright Act.

\vspace{10mm}

\hbox{\hbox to 0.5\hsize{%
In ........ date ............
\hss}\hbox to 0.5\hsize{%
signature of the author
\hss}}

\vspace{20mm}
\newpage

%%% Povinná informační strana diplomové práce

\vbox to 0.5\vsize{
\setlength\parindent{0mm}
\setlength\parskip{5mm}

Název práce:
Porovnáni metod česko-ruského automatického překladu

% přesně dle zadání

Autor:
Karel Bílek

Katedra:  % Případně Ústav:
Ústav formální a aplikované lingvistiky
% dle Organizační struktury MFF UK

Vedoucí diplomové práce:
doc. RNDr. Vladislav Kuboň, Ph.D.,
Ústav formální a aplikované lingvistiky
%Jméno a příjmení s tituly, pracoviště
% dle Organizační struktury MFF UK, případně plný název pracoviště mimo MFF UK

Abstrakt:
V této práci představuji několik metod česko-ruského automatického pře\-kla\-du, včetně jak více historických, tak více moderních systémů, a včetně jak frázovách, tak pravidlových systémů. Nejdříve stručně popisuji lingvistické základy češtiny a ruštiny a jejich společnou historii a rozdíly. Poté popisuji automatizaci, vytváření a zlepšování některých ze systémů automatického překladu, společně s jejich po\-ro\-vná\-ním, s použitím jak automatických metrik, tak omezené lidské anotace. Zároveň s tím také popisuji vytvoření několika korpusů česko-ruských paralelních dat a ruských monolingválních dat.
% abstrakt v rozsahu 80-200 slov; nejedná se však o opis zadání diplomové práce

Klíčová slova:
% 3 až 5 klíčových slov
Čeština, ruština, strojový překlad

\vss}\nobreak\vbox to 0.49\vsize{
\setlength\parindent{0mm}
\setlength\parskip{5mm}

Title:
% přesný překlad názvu práce v angličtině
{A Comparison of Methods of Czech-to-Russian Machine Translation}

Author:
Karel Bílek
%Jméno a příjmení autora

Department:
Institute of Formal and Applied Linguistics
%Ústav formální a aplikované lingvistiky
%Institute of Formal and Applied Linguistics
%Název katedry či ústavu, kde byla práce oficiálně zadána
% dle Organizační struktury MFF UK v angličtině

Supervisor:
doc. RNDr. Vladislav Kuboň, Ph.D.,
Institute of Formal and Applied Linguistics
%Jméno a příjmení s tituly, pracoviště
% dle Organizační struktury MFF UK, případně plný název pracoviště
% mimo MFF UK v angličtině

Abstract:
In this thesis, I am presenting several methods of Czech-to-Russian machine translation, including both historical approaches and more modern ones, and including both phrase-based and rule-based systems. I am first briefly describing the linguistic background of Czech and Russian, and their common history and differences. Then, I am describing automating, building and improving some of the machine translation systems, together with their comparison, using both an automated metric and a limited human annotation. Meanwhile, I am also describing the creation of a several corpora of Czech-Russian parallel data and Russian monolingual data.
%Abstrakt abstrakt.
% abstrakt v rozsahu 80-200 slov v angličtině; nejedná se však o překlad
% zadání diplomové práce

Keywords:
Czech, Russian, Machine translation
% 3 až 5 klíčových slov v angličtině

\vss}

\newpage

%%% Strana s automaticky generovaným obsahem diplomové práce. U matematických
%%% prací je přípustné, aby seznam tabulek a zkratek, existují-li, byl umístěn
%%% na začátku práce, místo na jejím konci.

\openright
\pagestyle{plain}
\setcounter{page}{1}
\tableofcontents



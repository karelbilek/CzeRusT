%%% Hlavní soubor. Zde se definují základní parametry a odkazuje se na ostatní části. %%%

%% Verze pro jednostranný tisk:
% Okraje: levý 40mm, pravý 25mm, horní a dolní 25mm
% (ale pozor, LaTeX si sám přidává 1in)
\documentclass[12pt,a4paper]{report}
\setlength\textwidth{145mm}
\setlength\textheight{247mm}
\setlength\oddsidemargin{15mm}
\setlength\evensidemargin{15mm}
\setlength\topmargin{0mm}
\setlength\headsep{0mm}
\setlength\headheight{0mm}
% \openright zařídí, aby následující text začínal na pravé straně knihy
\let\openright=\clearpage

%% Pokud tiskneme oboustranně:
% \documentclass[12pt,a4paper,twoside,openright]{report}
% \setlength\textwidth{145mm}
% \setlength\textheight{247mm}
% \setlength\oddsidemargin{15mm}
% \setlength\evensidemargin{0mm}
% \setlength\topmargin{0mm}
% \setlength\headsep{0mm}
% \setlength\headheight{0mm}
% \let\openright=\cleardoublepage

%% Použité kódování znaků: obvykle latin2, cp1250 nebo utf8:
%\usepackage[utf8]{inputenc}
%\usepackage[T2A]{fontenc}
%\usepackage[russian,czech,english]{babel}

%\usepackage{fontspec}

%\usepackage{xunicode,fontspec,xltxtra}
%\usepackage[english]{polyglossia}
%\setotherlanguages{russian} % set as "other" so English hyphenation active


\usepackage{libertine}
\setmonofont{Anonymous Pro}

%% Ostatní balíčky
\usepackage{verbatim}
\usepackage{graphicx}
%\usepackage{amsthm}

%% Balíček hyperref, kterým jdou vyrábět klikací odkazy v PDF,
%% ale hlavně ho používáme k uložení metadat do PDF (včetně obsahu).
%% POZOR, nezapomeňte vyplnit jméno práce a autora.
\usepackage[unicode]{hyperref}   % Musí být za všemi ostatními balíčky
\hypersetup{pdftitle=A Comparison of Methods of Czech-to-Russian Machine Translation}
\hypersetup{pdfauthor=Karel Bílek}

\usepackage{tabularx}

\usepackage{polyglossia}
\setmainlanguage{english}
\setotherlanguage{czech}

\usepackage{float}

\usepackage[
   backend=bibtex8      % if we want unicode 
  ,style=iso-authoryear % or iso-numeric for numeric citation method          
  ,babel=other        % to support multiple languages in bibliography
  ,sortlocale=cs_CZ   % locale of main language, it is for sorting
  %,bibencoding=UTF8   % this is necessary only if bibliography file is in different encoding than main document
]{biblatex}


\bibliography{library}
%%% Drobné úpravy stylu
% Tato makra přesvědčují mírně ošklivým trikem LaTeX, aby hlavičky kapitol
% sázel příčetněji a nevynechával nad nimi spoustu místa. Směle ignorujte.
\makeatletter
\def\@makechapterhead#1{
  {\parindent \z@ \raggedright \normalfont
   \Huge\bfseries \thechapter. #1
   \par\nobreak
   \vskip 20\p@
}}
\def\@makeschapterhead#1{
  {\parindent \z@ \raggedright \normalfont
   \Huge\bfseries #1
   \par\nobreak
   \vskip 20\p@
}}
\makeatother

% Toto makro definuje kapitolu, která není očíslovaná, ale je uvedena v obsahu.
\def\chapwithtoc#1{
\chapter*{#1}
\addcontentsline{toc}{chapter}{#1}
}


\newfloat{tabulka}{tbph}{lop}
\floatname{tabulka}{Table}


\newfloat{graff}{tbph}{lop}
\floatname{graff}{Figure}



\long\def\centergraf#1 {
{
    \begingroup
    \centering
    \par
    #1
    \par
    \endgroup
}

}

\def\uv#1{``#1''}

\long\def\jednatabulkan#1#2#3#4{
 \begin{tabulka}[H]
     
         \centergraf{
         \footnotesize
    \begin{tabular}{ #2 }
        #3
        
    \end{tabular}
             
             }
     
     \caption{#4} 
     \label{tabulka:#1}
 \end{tabulka}
}


\long\def\jednatabulkam#1#2#3#4{
 \begin{tabulka}[h]
     
         \centergraf{
         \scriptsize
    \begin{tabularx}{\textwidth}{ #2 }
        #3
        
    \end{tabularx}
             
             }
     
     \caption{#4} 
     \label{tabulka:#1}
 \end{tabulka}
}

\long\def\jednatabulka#1#2#3#4{
 \begin{tabulka}[h]
     
         \centergraf{
         \footnotesize
    \begin{tabularx}{\textwidth}{ #2 }
        #3
        
    \end{tabularx}
             
             }
     
     \caption{#4} 
     \label{tabulka:#1}
 \end{tabulka}
}

\long\def\grafff#1#2#3{
 \begin{graff}[h]
         \centergraf{
      \includegraphics[width=#3mm]{./#1.pdf}
       \caption{#2}
      \label{graf:#1}
 }
     \end{graff}
 }



\newenvironment{pitemize}{
\begin{itemize}
  \setlength{\itemsep}{5pt}
  \setlength{\parskip}{0pt}
  \setlength{\parsep}{0pt}
}{\end{itemize}}


\long\def\priklady#1 {

\begin{pitemize}
#1

    \end{pitemize}

}

\usepackage{stringstrings}



\makeatletter
\newcommand*{\stripstartspaces}{%
  \expandafter\@ifnextchar\expandafter X\expandafter{\expandafter
  }\expandafter{\expandafter}%
}
\makeatother


\newcommand{\removelinebreaks}[1]{%
  \begingroup\def\\{}#1\endgroup}

\long\def\priklad#1#2#3 {
 \item\textbf{#1}\par\nobreak#2\nobreak\noindent\par\emph{\stripstartspaces#3}

}

\begin{document}

% Trochu volnější nastavení dělení slov, než je default.
\lefthyphenmin=2
\righthyphenmin=2

%%% Titulní strana práce

\pagestyle{empty}
\begin{center}

\large

Charles University in Prague

\medskip

Faculty of Mathematics and Physics

\vfill

{\bf\Large MASTER THESIS}

\vfill

\centerline{\mbox{\includegraphics[width=60mm]{../img/logo.pdf}}}

\vfill
\vspace{5mm}

{\LARGE Karel Bílek}

\vspace{15mm}

% Název práce přesně podle zadání
{\LARGE\bfseries A Comparison of Methods of Czech-to-Russian Machine Translation}

\vfill

% Název katedry nebo ústavu, kde byla práce oficiálně zadána
% (dle Organizační struktury MFF UK)
%Ústav formální a aplikované lingvistiky
Institute of Formal and Applied Linguistics

\vfill

\begin{tabular}{rl}

Supervisor of the master thesis: & doc. RNDr. Vladislav Kuboň, Ph.D. \\
\noalign{\vspace{2mm}}
Study programme: & Informatika  ????  \\
\noalign{\vspace{2mm}}
Specialization: & Asi něco s lingvistikou \\
\end{tabular}

\vfill

% Zde doplňte rok
Prague 2014

\end{center}

\newpage

%%% Následuje vevázaný list -- kopie podepsaného "Zadání diplomové práce".
%%% Toto zadání NENÍ součástí elektronické verze práce, nescanovat.

%%% Na tomto místě mohou být napsána případná poděkování (vedoucímu práce,
%%% konzultantovi, tomu, kdo zapůjčil software, literaturu apod.)

\openright

\noindent
Dedication. I love everybody. (TODO)

\newpage

%%% Strana s čestným prohlášením k diplomové práci

\vglue 0pt plus 1fill

\noindent
I declare that I carried out this master thesis independently, and only with the cited
sources, literature and other professional sources.

\medskip\noindent
I understand that my work relates to the rights and obligations under the Act No.
121/2000 Coll., the Copyright Act, as amended, in particular the fact that the Charles
University in Prague has the right to conclude a license agreement on the use of this
work as a school work pursuant to Section 60 paragraph 1 of the Copyright Act.

\vspace{10mm}

\hbox{\hbox to 0.5\hsize{%
In ........ date ............
\hss}\hbox to 0.5\hsize{%
signature of the author
\hss}}

\vspace{20mm}
\newpage

%%% Povinná informační strana diplomové práce

\vbox to 0.5\vsize{
\setlength\parindent{0mm}
\setlength\parskip{5mm}

Název práce:
Porovnáni metod česko-ruského automatického překladu

% přesně dle zadání

Autor:
Karel Bílek

Katedra:  % Případně Ústav:
Ústav formální a aplikované lingvistiky
% dle Organizační struktury MFF UK

Vedoucí diplomové práce:
doc. RNDr. Vladislav Kuboň, Ph.D.,
Ústav formální a aplikované lingvistiky
%Jméno a příjmení s tituly, pracoviště
% dle Organizační struktury MFF UK, případně plný název pracoviště mimo MFF UK

Abstrakt:
Abstrakt abstrakt.
% abstrakt v rozsahu 80-200 slov; nejedná se však o opis zadání diplomové práce

Klíčová slova:
% 3 až 5 klíčových slov
Čeština, ruština, strojový překlad

\vss}\nobreak\vbox to 0.49\vsize{
\setlength\parindent{0mm}
\setlength\parskip{5mm}

Title:
% přesný překlad názvu práce v angličtině
{A Comparison of Methods of Czech-to-Russian Machine Translation}

Author:
Karel Bílek
%Jméno a příjmení autora

Department:
Institute of Formal and Applied Linguistics
%Ústav formální a aplikované lingvistiky
%Institute of Formal and Applied Linguistics
%Název katedry či ústavu, kde byla práce oficiálně zadána
% dle Organizační struktury MFF UK v angličtině

Supervisor:
doc. RNDr. Vladislav Kuboň, Ph.D.,
Institute of Formal and Applied Linguistics
%Jméno a příjmení s tituly, pracoviště
% dle Organizační struktury MFF UK, případně plný název pracoviště
% mimo MFF UK v angličtině

Abstract:
Abstrakt abstrakt.
% abstrakt v rozsahu 80-200 slov v angličtině; nejedná se však o překlad
% zadání diplomové práce

Keywords:
Czech, Russian, Machine translation
% 3 až 5 klíčových slov v angličtině

\vss}

\newpage

%%% Strana s automaticky generovaným obsahem diplomové práce. U matematických
%%% prací je přípustné, aby seznam tabulek a zkratek, existují-li, byl umístěn
%%% na začátku práce, místo na jejím konci.

\openright
\pagestyle{plain}
\setcounter{page}{1}
\tableofcontents

%%% Jednotlivé kapitoly práce jsou pro přehlednost uloženy v samostatných souborech
\chapter*{Introduction}
\addcontentsline{toc}{chapter}{Introduction}
Slavic  family of languages consists of around 12 languages, de\-pen\-ding on the exact definition.\footnote{See for example \cite{sussex2011slavic} or \cite{siewierska1998overview} --  the final language count depends on whether Upper and Lower Sorbian are taken as different languages or not; whether Kashubian is a separate language, or just a variety of Polish; whether Serbian and Croatian are treated as separate languages or not}
They are usually divided into three sub-groups - South Slavic, West Slavic and East Slavic. While languages inside each of those groups are mutually intelligible, intelligibility accross those groups is lower; still, the Slavic languages in general have very similar morphology, syntax and vocabulary.

Czech language belongs to the West Slavic group, while Russian belongs to the East Slavic group.  Therefore, they are not fully mutually intelligible -- but, as noted, as Slavic languages, they are in some respects still very similar. 
This creates a great environment for building machine translations tools -- the languages are not too close to make the task meaningless, while we can still take an advantage of the closeness.

For political reasons, significant efforts were directed towards building an automatic translation system between Russian and Czech, mainly during mid-eighties with the RUSLAN system. The efforts were terminated in 1990 in the final phases of development.\footnote{See for example \cite{recycled}, \cite{hajic1987}, \cite{olivaruslan}} 

After the velvet revolution, political dependence on Russian-speaking countries were weakened, while the economical and cultural dependence on English-speaking countries were strengthened. Therefore, the efforts of automatic translation were more focused on English. 

In the process, general direction of machine translation efforts shifted from rule-based systems to more statistical-based systems. This allows us to compare older, rule-based approaches to the more modern statistical approaches.

In this thesis, I am trying to describe the rich history of automatic Czech-to-Russian translation, then experiment with the available tools -- both historical and more recent ones -- on a given set of data, and finally compare the results.

\chapter{Brief comparison of Czech and Russian}

\section{Slavic in general}
This section is an overview of history of Slavic languages and also of some characteristics of Slavic languages in general.
%, mainly with the respect of machine translation.

%I am focusing mainly on syntactical changes -- not because there are no morphological, phonological or lexical changes, but because there are too many to cover here.

If a reader is interested in a further study of Slavic languages as a whole from linguistic perspective, I can only recommend \cite{sussex2011slavic}.
Unless otherwise noted, information in this chapter are taken from this book.\subsection{History}

\label{ch:common_history}

The Indo-Europeans occupied European lands from approximately 4000 BC. Their language, Proto-Indo-European (PIE), is not known and can only be linguistically reconstructed. 

In approximately 2000 BC, a Proto-Slavic (PS) language started to emerge. Until approximately 400 AD, the language was fairly uniform and the land occupied by Slavs was broadly coherent, stretched approximately from Oder river (Odra in Polish) to Dnieper (Днепр in Russian).

\subsection{Slavic languages overview}
\subsubsection{Morhpology}
On the morphological typology, Slavic languages belong to synthetic inflectional languages. It's morphologically rich, with a sophisticated system of prefixes, roots and suffixes, with some analytical approaches (for example, in verb morphology).

Slavic words consist of one or more roots -- the main component of the word -- and then several prefixes and suffixes. Prefixes usually modify the word's meaning somehow (for example, \uv{ne-} for negative), while suffixes are either derivational, inflectional or post-inflectional (appearing in this order).

The derivational suffixes can determine word's class with processes like nominalization, verbalization, adjectivalization. Next are inflectional suffixes (also called endings), that mark one of several categories.

\jednatabulkan{reflex} { |l |l | l | }
{
         \hline

to wash (someone) & mýt & мыть \\ \hline
to wash (self) & mýt se & мыться \\ \hline


} {Reflectives in Russian vs. Czech} 

Post-inflectional suffixes are found in some languages (like Russian) for example for reflectivity, whereas in other languages a separate word is used. See for example Table~\ref{tabulka:voices} (also compare to the voices in the section~\ref{ch:mediopassive}).

\jednatabulkan{categories} { |l |l | l | }
{
         \hline

\textbf{Verbs} & \textbf{Nominals} & \textbf{Both} \\ \hline
Tense & Case & Person \\ \hline
Aspect & Definiteness & Gender \\ \hline
Mood & Deixis & Number \\ \hline
Voice &  &  \\ \hline


} {Inflectional categories} 

Inflectional categories that are presented in the Slavic languages are described in the Table~\ref{tabulka:categories} (definiteness and deixis as inflectional categories are, however, not relevant to either Czech or Russian, but are used as such in Macedonian or Bulgarian).

Slavic languages are surprisingly uniform in the word formation system, and the operations and grammatical processes are simmilar -- however, the affixes themselves are usually either wholly different, or have vastly different semantic and stylistic properties -- either absolutely, or in combination with other words.

The result of the described morphological richness is that the number of word forms arise significantly, especially when compared with more analytical languages (like English). As you can see in the section XXX, in the same corpus, Czech and Russian have \emph{significantly} more word forms, than English.
%That can hurt us 
\subsubsection{Word order}
%Word-classes in Slavic languages are very similar to other European languages. They have open word-classes: nous, verbs, adjectives, adverbs; and closed word-classes: auxiliaries, determiners, pronouns, prepositions, conjunctions and interjections. Across the Slavic languages, those part of speech are fairly simmilar.
In English, syntactic relations are marked mainly by word order. In Slavic languages, however, those relations are marked instead by three types of inflexion: concord, agreement and government.

For example, subject \emph{agrees} with predicate in number, person and gender. 

Because the relations are marked by inflexion, it \uv{allows} the language to be of freer word order; and indeed, in Slavic languages, the word order variations are more common than in other languages. The \emph{standard} order is SVO, however, this is possible to change when different emphasis is needed.

In particular, this reffers to the so-called Functional Sentence Perspective. Very simply said, it states that sentence could be split in two parts -- Topic and Comment, appearing in this order and being separated by a verb. Topic is the part of sentence, \emph{about which something is said} -- while Comment is the \emph{new information presented}. Most importantly, Topic doesn't have to be a grammatical subject of the sentence -- for example, the Czech sentence \uv{Ve městě bydlí strašidla} (\emph{Ghosts live in the city}, literally \emph{In city live ghosts}).

Slavic languages usually (with some exceptions like Bulgarian and Macedonian) don't have particles or any other means of marking definitiveness -- except for definite information being in the topic of the sentence. In other view, the word order and functional sentence perspective can be seen as \uv{replacing} the definitive particles, and is usually translated as such in translation to English.

In respect to the translation between English and Slavic languages, word order can be seen as a nuisance and hard to translate correctly. However, with translation between Slavic languages, it can actually help us, since we can preserve the word order and move the words less.
\subsection{Some changes from PIE to PS}

\subsubsection{Cases}
\jednatabulka{korintskym} { |l|X |X | X | }
{
         \hline
genitive &
invocant nomen \textbf{Domini} nostri Iesu Christi
&
   who call on the name \textbf{of} our \textbf{Lord} Jesus Christ         
&
všem, kdo na jakémkoli místě vzývají jméno našeho společného \textbf{Pána} Ježíše Krista
\\

    \hline
ablative
&
gratia vobis et pax a Deo Patre nostro et \textbf{Domino} Iesu Christo
&
Grace and peace to you \textbf{from} God our Father and \textbf{the Lord} Jesus Christ
&
Milost vám a pokoj od Boha, našeho Otce, a od \textbf{Pána} Ježíše Krista \\
    \hline
} {Demonstrating ablative and genitive conflation on 1 Corinthians} 


PIE had seven cases - nominative, accusative, genitive, dative, instrumental, locative, ablative and vocative. 
In PS, ablative and genitive were conflated into just genitive. 

Two phrases from New Testament can offer us a demonstration.
%To illustrate, I have added a simple comparison of old latin and Czech. 

Old Latin retained both ablative and genitive. Latin genitive of \emph{dominus} (lord, master) is \emph{dominī}, latin ablative of the same word is  \emph{dominō}. Both these forms were used in the very beginning of 1~Corinthians in Latina Vulgata (latin version of The Bible).

In the Table~\ref{tabulka:korintskym}, I have shown the translation to both Czech and English. We can see that both cases are inflated in Czech as genitive \uv{(našeho) pána}.\footnote{Bible sources: \cite{latinavulgata}, \cite{bibleniv}, \cite{bible21}. Note that English and Czech translations are not actually translations from Latin, but from more primary sources, but it will suffice for this simple comparison.}

\subsubsection{Numbers}
\jednatabulkan{kravy} { |l|l |l | l | }
{
         \hline
singular &
one cow
&
ena krava
&
jedna kráva
\\
   \hline
dual (in Slovenian) &
two cows
&
dve kravi
&
dvě krávy
\\
   \hline
plural &
three cows
&
tri krave
&
tři krávy
\\


    \hline
} {Demonstrating duals on Slovenian} 

PIE had three numbers, singular, plural and dual. In PS, dual slowly disappeared, while still retaining some of its usage.

However, dual disappeared in most of Slavic languages later (including Czech and Russian), leaving only traces in the grammar. One of the languages where dual remained is Slovenian.

To illustrate, I have added a comparison of \uv{one cow}, \uv{two cows} and \uv{three cows} in Slovenian and Czech in Table~\ref{tabulka:kravy}.

In Czech, dual was retained in declensions of several words, like \uv{hands}; in Russian, the dual \uv{рукама} survives in some dialects, but is generally incorrect.\footnote{See \cite{offord1996using}, page 18.}

\subsubsection{Genders}
PIE had three genders, masculine, feminine and neutral. PS retained these genders, adding categories Personal and Animate.

\subsubsection{Tenses}
PIE had six tenses: present, future, aorist, imperfect, perfect and pluperfect. 

The tenses were somehow retained in PS; however, future, perfect and pluperfect were in many cases re-formed analytically --- creating more complex forms with auxiliary verb and either an infinitive or past participle -- for example, \uv{budu zpívat} (I will sing) in Czech, or \uv{буду петь} in Russian.

\subsubsection{Moods}
PIE had four moods: indicative, subjunctive, optative and imperative. In PS, imperative forms were replaced by the optatives, and subjunctive mood slowly became conditional.

\subsubsection{Voices}
\label{ch:mediopassive}
\jednatabulkan{voices} { |l|l |l | l | }
{
         \hline

active & νιπτω & I wash (someone) & myji (někoho) \\ \hline
medium & νίπτομαι & I wash (myself) & myji se \\ \hline
passive & νίπτομαι & I am washed (by somebody) & jsem myt \\ \hline


} {Voices in Classic Greek vs. Czech} 

PIE had an active and a mediopassive voice. 

In PS, this was refined as a reflexive and a non-reflexive voice, with the addition of a passive voice.

To illustrate, I have added an example of Classic Greek that still retained the mediopassive voice, in Table~\ref{tabulka:voices}.\footnote{See for example \cite{greek1}, \cite{greek2}}



\subsubsection{Aspects}
\jednatabulkan{nosim} { |l|l |l |  }
{
         \hline
imperfective determinate &
нести́
&
nést
\\
   \hline
imperfective indeterminate &
носи́ть
&
nosit
\\
   \hline
perfective& 
понести 
&
ponést

\\


    \hline
} {Aspects in Czech and Russian} 

PIE had distinction between two aspects -- eventive and stative; eventive aspect being further divided into perfective and imperfective aspect.\footnote{See \cite{ringe2008proto}, page 24}

In PS, the stative aspect is degramatized\footnote{See \cite{andersen2013origin}}; however, the perfective/imperfective distinction became more important in PS than in other Indo-European languages. Also, the imperfective motion verbs were further split into determinate and non-determinate.

The determinate/non-determinate and perfective/imperfective distinction is retained in both Czech and Russian. See the Table~\ref{tabulka:nosim} and note, how hard would be to correctly translate the distinction into English.


\section{Drifting of Czech and Russian}
In about 5th century AD, the relative unity of Slavic started to break up, caused mainly by migration, which was partly caused by expansion to the north and east by the Eastern Slavs, partly by dissolution of Roman and Hun empires and the resulting vacuum in Central Europe.

One group of Slavs moved westwards, reaching approximately what is now Poland, Czech Republic and parts of Germany, creating the so-called West Slavic language group, which Czech language is a member of. The other group moved south to Balkan, creating a South Slavic language group, which I won't discuss further in this thesis. Eastern Slavs moved north and east, eventually creating an Eastern Slavic language group. Slavic was divided into those three groups at about 10th century AD.

The differences between the languages themselves are actually less syntactical and are more phonological, morphonological and morphological, with some broad-scale lexical changes. For that reason, I am not describing the linguistic differences of the languages any further, since it would be, again, not in a scope for this thesis. Again, I can recommend the book \cite{sussex2011slavic} for anybody interested in the deeper differences.



\chapter{Translation systems}
\section{Statistical vs. rule-based -- an overview}
Machine translation (MT) systems has historically used many different approaches. One way of classifying the approaches is on the axis of rule-based vs. statistical.

In general, we can re-use the descriptions, used by \cite{bojar}, which is as follows:
\begin{itemize}
\item rule-based MT systems:
\begin{itemize}
\item use analysis, transfer and synthesis steps
\item use formal grammars
\item use hand-made dictionaries
\item have linguistic information hard-coded and therefore aren't lan\-guage-ag\-nos\-tic
\end{itemize}
\item statistical MT systems
\begin{itemize}
    \item use more variants of outputs, rank them with some score, and choose the best one
    \item train internal dictionaries from big parallel data
    \item have more compact translation core, their inner working are less obvious
    \item use statistics instead of linguistic rules and therefore are more language-agnostic
\end{itemize}
\end{itemize}

However, with some of the modern systems, the distinction is not as clear-cut as we would like, for the purpose of our comparison. 
For example, statistical MT systems like Moses can get significantly better with some added linguistic information; 
on the other hand, systems like TectoMT, which can sometimes be classified as more rule-based, actually have big modules more or less based on statistics.


\section{Unrunnable systems}
In this section I will present some historical systems, that I haven't been able to get successfully running.

\subsection{RUSLAN}

RUSLAN is a machine-translation system, developed between 1984 and 1988 at several departments of Charles University, Prague. RUSLAN firmly belongs to the \emph{rule-based} category, since at that timeframe, statistical machine translation wasn't even invented yet.

\subsubsection{Q-Systems}
Q-Systems (sometimes also Systems Q) -- Q stands for \uv{Quebec} -- are a tool for machine translation, developed at Montreal University by Alain Colmerauer, also the creator of Prolog.\footnote{See \cite{qsystems}.} 

Q-Systems are similarly declarative as Prolog, focusing more on the result than the procedural order of analysis. If there is any ambiguity, all the possibilities are explored \emph{in parallel}. In theory, this could make writing the lexical rules easier and the rules themselves more readable; in reality, the rules are still quite unreadable, as will be seen later.

Q-Systems are not very widely used or widely worked with. One of the reasons might be the fact that all documentation is in French. However, \cite{nguyen2009systemes} describes\footnote{Unsurprisingly, also in French.} modern-day experiments with Systems Q, also mentioning, that all Fortran implementations has been lost\footnote{Which would make UFAL's version, if it was working, quite unique.} and that he reimplemented it in C. I haven't been able to try this version.


\subsubsection{RUSLAN components}

Description of the system can be found in \cite{olivaruslan} or \cite{hajic1987} -- however, the reader has to bear in mind that not only the systems are severly dated, but so is their manual and description.\footnote{At least for me personally, especially the book \cite{olivaruslan} was hard to read and hard to find the needed information in.} Contemporary (but not as detailed) description of the system can be found in \cite{recycled}.

The whole RUSLAN system has several components:
\begin{pitemize}
\item preprocessing, written in Pascal
\item morphological analysis, using dictionary, written in Q-Systems and interpreted in Fortran IV
\item syntactico-semantical analysis, using morphology, also written in Q-Systems; this component uses FGD as its theoretical starting point
\item generation, also using Q-Systems
\item morphological synthesis, using Pascal
\end{pitemize}

RUSLAN uses a Czech-to-Russian dictionary, written by hand in afforementioned Q-Systems. Dictionary item looks like this:

\begin{verbatim}
DLOUH==M(RS(+(*INT)),MI2289,DLINNYJ).
DLOUH==M(RS(-(*INT)),MI2276,DOLGIJ).
\end{verbatim}

This describes two possibilites of the translation of the word \uv{dlouhý} to Russian: the first is \uv{длинный} and the second is \uv{долгий}. They differ by the semantic feature INT they require or forbid from the word they depend on. 

More complex dictionary item looks like this:
\begin{verbatim}
C3ES3TIN
  ==Z(@(*A), MIO109, $(JAZYK), 
     2(POS, #($), &, $(MI28), $(C2ES2KIJ),
       1(=,@($), #($),$($))),
     1(=,@($),#($),$($))).
\end{verbatim}

This item translates the word \uv{čeština} to Russian words \uv{чешский язык} and also describes their relationship.

Maybe because memory was more expensive than today, all similar rules are in the dictionary without any comments, leaving only very difficult-to-decypher rules.

The rules for analysis are even less readable. Random examplo of two such rules are as follows:

\begin{verbatim}
1(B*, X*1, /, X*2, F*1(C*), X*3, /, X*4, @(V*), X*5, %(X*),
 I(*), 1(X*6, $($)), X*7)

1Z(A*9), (Z*2)
  == 1(D*, /, X*2, F*1/X*),/,@(V*),X*5,%(X*),1*,1(X*6,$($)),X*7,
      A*B,
     5(U*1, @(U*2), U*3, $(U*), 3(E*(Y*1), B*(C*), W*1, W*3, %(X*), 
        $(W*)),
     +1Z(A*9, Z*2)
       / -NON- (, + -DANS- X*9 -ET- +(V*) -HORS- X*9, +(VZT)
        -ET- -(V*) -HORS- X*9, *
        -ET- C* = S
        -ET- X*3,* -HORS- /,N(S), S(S), D(S), A(S), L(S), I(S)
        -ET- / -HORS- X*2, 2
        -ET- (, Y%1 = -NUL-
             -OU- E*(Y*1) -HORS- U*2,*
             -OU- E*(+(V*, *)) -HORS- U*2, *
             -OU- -NON- E*(-(V*)) -HORS- U*2, * .)
        -ET- (. E*(Y*1) *N
             -OU- H(B*(C*)) -DANS- U*1 .) .
\end{verbatim}

This is all left with very little comments. For example, the only comment for the two rules above and about 20 more is \texttt{RELATIVE CLAUSES ADJOINED TO THEIR HEADS}.

\subsubsection{Dictionary coverage of WebColl}

The dictionary contains about 8,000 lexical items. However, the domain of the translation and, therefore, the dictionary, was manuals for old computers from 1980's. 

In different experiments (\cite{florida}), we tried to measure how many nouns from the RUSLAN dictionary appear at all in a modern text.

For that, we used a monolingual Czech corpus WebColl, consisting of roughly 7~million sentences (114~million tokens)\footnote{See \cite{webcoll}}.

RUSLAN dictionary has 2,783 nouns. In the WebColl corpus, from those nouns, 611 appear less than 10 times -- and from those, 412 don't appear \emph{at all}.

The reverse is similarly infavourable: from 39,434,505 nouns in the corpus, only 11,862,221 is in the dictionary.

\subsubsection{Experiments}

Despite the general un-maintainability of the RUSLAN code and despite the small dictionary, we tried to run the system on our test data.

However, all our experiments ended in some sort of error.

Because I am not able to code in neither Systems-Q nor FORTRAN (in which the Systems-Q interpreter is coded), I gave up on this experiment. (?????)


\subsection{Česílko 1.0}

\uv{Česílko} is a name for two entirely different machine translation systems with slightly different goals and, more importantly, slightly different structure. Both were originally intended for Czech-to-Slovak translation.

Česílko 1.0 was a system, developed in 2000, and was aimed for direct translation between Czech and Slovak and intended to assist a translation memory\footnote{See \cite{cesilko1}.}. The translation works lemma-by-lemma in a following fashion:
\begin{pitemize}
\item morphological analysis of source (Czech) language
\item disambiguation
\item direct translation, lemma-by-lemma
\item morphological synthesis
\end{pitemize}



The system is written in a mixture of C, C++ and Flex (fast lexical analyser generator). The code itself is not really well documented and modular, but that can be attributed to the age of some of the components -- despite the whole system being developed in 2000, some files seem to be as old as 1991.

This system itself is not very extendable from Slovak to Russian. Partly because of the design itself, partly because -- as we can see on the examples in the section \ref{sec:experiments} -- the sentences are not really translatable word-by-word.

For that reason I decided not to further experiment with Česílko 1.0 for Czech-to-Russian machine translation.

\subsection{Česílko 2.0}
Česílko 2.0 is a different project with similar goals, but using different frameworks and adding more transfer rules\footnote{See \cite{cesilko2}}. However, it has its own shares of problems, that prevented us to use it.

The system works in a following fashion:
\begin{pitemize}
\item \textbf{non-deterministic} morphological analysis of source Czech language
\item translation of lemmas
\item applying transfer rules by changing syntactic tree
\item morphological synthesis
\item ranking of all the generated sentences
\end{pitemize}

Unlike Česílko 1.0, Česílko 2.0 uses a non-deterministic parser and explores all the possi\-bi\-li\-ties in parallel. 

The more advanced transfer would, in an ideal world, make the system more modular and extensionable for our purposes. However, the implementation details prevented us from doing any significant work on Česílko 2.0.

To illuminate why, let me focus a little on the technical details.

\subsubsection{Objective-C}
Objective-C is a very simple and elegant extension of C language, developed by Brad Cox in 1980s by adding Smalltalk features to C\footnote{See \cite{cocoa4}}. 

Objective-C is, in my opinion, very easy to learn and understand, at least compared to C++, its more popular counterpart.

Objective-C is not a proprietary language and is possible to compile with either gcc or Clang/LLVM compilers. However, what is proprietary is its most used standard library, Cocoa.

\subsubsection{Cocoa}
When Steve Jobs left Apple, he made a smaller company called NeXT. Among other things, they produced a proprietary operating system called NeXTSTEP, based on Unix.\footnote{For a more detailed history, see \url{https://developer.apple.com/legacy/library/documentation/Cocoa/Conceptual/CocoaFundamentals/WhatIsCocoa/WhatIsCocoa.html\#//apple\_ref/doc/uid/TP40002974-CH3-SW12}.}

This operating system used Objective-C as its standard language, and proprietary libraries, called OpenStep.\footnote{Despite the name, OpenStep is not open source -- the Open allude to the fact that its API specification was open.}

Several years later, Apple (now merged with NeXT) made its new version of Mac OS, called Mac OS X; this operating system was partially based on NeXTSTEP and also used some of its proprietary libraries, now renamed Cocoa.\footnote{The kernel of Mac OS X is open source, as is its \uv{underlying} operating system called Darwin -- however, this system does not contain Cocoa libraries.}

Cocoa is not the only library for Objective-C, but because Apple is the main investor in Objective-C-based systems, it's a de-facto standard library.

\subsubsection{GNUstep}
GNUstep is a free re-implementation of OpenStep/Cocoa.\footnote{See \url{http://www.gnustep.org/}}.

Its development started in the NeXTSTEP days; however, it still hasn't met feature parity with Cocoa's OS X.

Aaron Hillegass in 2nd edition of his popular book \emph{Cocoa Programming on Mac OS X} discouraged people from using GNUStep. He redacted this note in later versions of the book, perhaps because of protests from GNUstep developers\footnote{\url{http://www.gnustep.org/resources/BookClarifications.html}}, but in my opinion, his notes are still valid.

GNUstep implementations are very often buggy and not feature-complete with Cocoa and, most unfortunately, unpredictable. This is what hurt us with Česílko 2.0.

\subsubsection{Cocoa and Česílko}
When Petr Homola was writing Česílko 2.0, he decided to use Cocoa and Objective-C for development.

On Mac OS X, this configuration is just fine; however, on Linux, where we wanted to run the MT systems (and where only GNUstep is available), this creates unpredictable results.

In my experiments with Czech-to-Slovak translations, I noticed that on Mac OS X, there are about 5-times more sentences generated, than on Linux -- while the program was compiled from the same sources.

After thorough inspection, I found out the error was in GNUstep implementation of NSDictionary -- Cocoa's implementation of associative array\footnote{\url{https://developer.apple.com/library/mac/documentation/Cocoa/Reference/Foundation/Classes/NSDictionary\_Class/Reference/Reference.html}} -- in some unpredictable cases, NSDictionary returns two different values for two equal NSString keys\footnote{it might have to do something with Unicode; however, NSStrings are supposed to be UTF-8 by default}. As a result, one of the modules returned wrong inflection patterns for a number of words and the morphological analyzer then returned only a fraction of the results.

After a \uv{hacky}, but working workaround for this issue, the system returned same correct results on both OS X and Linux. However, I am not at all confident there aren't more similar issues in GNUstep to further develop the system for Russian;
fixing the issues of the standard frameworks, copying API of a closed-source library, that's normally very rarely used, is way beyond the scope of this thesis.

%when the very basic frameworks themselves are unstable and unreliable, the development ceases to make sense.

Reading the paper \cite{evalquality_cesilko}, that presents Česílko 2.0 with a very low BLEU, I think the same issue plagued the authors of that paper -- it's unprobable the BLEU of the correctly working system would be that low, when in \cite{cesilko2}, the results of Česílko 2.0 were slightly better than of Česílko 1.0.

\section{Black-box systems}
\label{blackbox}
In this chapter, I am describing all the \uv{black-box} systems -- that is, without any access to the source code -- that we successfully tried.

\subsection{PC Translator}
\label{langsoft}
\subsubsection{Description}

PC Translator is a commercial translation system from a Czech company LangSoft (\url{http://www.langsoft.cz/translator.htm}). PC Translator can translate several language pairs, all with Czech on either source or target side.

Authors of PC Translator don't publish any papers or other literature about the system -- what can we tell about its functionality is gathered only from its promotional website and from the experiments with the software itself.

PC translator seems to be purely rule-based. The system seems to work in following steps:

\begin{pitemize}
\item some (probably rule-based) morphological analysis of the source language
\item translation of the lemmas from source language to target language by searching in a large dictionary
\item some synthesis of morphological information and the translated lemma
\end{pitemize}

The system doesn't seem to do any kind of reordering. It also doesn't seem to do any analysis on a deeper level, like sentence constituents. Some of the phrases in the dictionary are longer than one word, but not too many of them.

One of the advantages of PC Translator is its large dictionary -- however, the dictionary is sometimes choosing very odd and inprobable choices when disambiguating between more possible translations. For example, the English sentence \uv{I like dogs} is translated as \uv{Mám rád kleště}, because the term \uv{dog} can be also translated as \uv{kleště}\footnote{from Collins' Dictionary: \uv{dog -- 5. a mechanical device for gripping or holding, esp one of the axial slots by which gear wheels or shafts are engaged to transmit torque}}. This can be seen as a proof that PC Translator is a purely rule-based system.

According to its marketing materials, PC Translator v14 uses a Czech-Russian dictionary with above 650.000 words.

\subsubsection{Experiments}
We found out it's not easy to automate translating with PC Translator. Its GUI is suited for translating by hand, sentence-by-sentence, but not for automated translation of thousands of sentences. Also, by definition, Windows GUI is harder to automate on Linux machine from a script.

However, we were able to work around that, with the help of VMWare Player virtualization software (\url{http://www.vmware.com/cz/products/player}) and Au\-to\-Hot\-key GUI scripting software, that allows us to emulate screen clicking (\url{http://www.autohotkey.com/}). Our workflow therefore is:

\begin{pitemize}
\item on Linux machine, encode the source from UTF-8 to windows-friendly encoding
\item encode the source as HTML code
\item start a virtual machine with PC Translator pre-installed
\item on the start of the virtual machine, run AutoHotkey script from an outer-machine folder (thanks to VMWare shared folders and Windows Startup scripts)
\item via this AutoHotkey script, run PC Translator and click on \uv{translate file} feature 
\item translate the HTML file (also shared in the VMWare shared folder)
\item turn off the virtual machine
\item turn the file back from HTML and Windows encodings back to UTF-8
\end{pitemize}

The HTML part is needed because PC Translator had some problems with translating ordinary text files, plus we can pair the translated sentences better thanks to \texttt{id} parameters in \texttt{div} tags.

We used the newest version of PC Translator available at the time, which is PC Translator v14.




\subsection{Google Translate}
\label{google}
Google Translate is a popular free online translation service by Google, an American web search giant (\url{http://translate.google.com}). 
Although Google is producing many academic papers on machine translation, the whole system is still proprietary and we cannot fully inspect it, as in the case of PC Translator.

Google Translate uses mostly statistical approach to machine translation, see for example \cite{och}\footnote{F. J. Och is a head of Google Translate group in Google}. Its results are often seen as \uv{state-of-the-art} in machine translation.

However, thanks to its purely statistical approach, it either needs huge amounts of data for every language pair, or it needs to use so-called \uv{pivot languages}\footnote{See for example \cite{koehn2010statistical}} -- in the case of Google Translate, it's usually English; specific English word order and English idioms are then re-translated into the target language and sometimes introduce downright wrong translations.

\subsubsection{API}
To automate Google Translate, we cannot use the website itself, simply because pasting tens of thousands of lines into a browser window usually crashes the browser and is probably against Google Translate's Terms of Use.

There are some workarounds around this, such as \uv{faking} browser environment using some automation tools and/or libraries, but we used more stable option.

Google Translate, apart from being a website, has a paid translation API\footnote{\url{https://developers.google.com/translate/?hl=cs}}. We figured out it's not too expensive for our testing purposes, so we ended up paying for the API.\footnote{The cost is measured per character on the source side. We used about 3 million characters and paid about 60 dollars. This is rather high for any repeated experiments, but not that high for one-time translation.}

%We used an unofficial Java library for Google Translate API, called prosaically \uv{google-api-translate-java} (\url{https://code.google.com/p/google-api-translate-java/}).
We used a Java library for Google Translate API, called prosaically \uv{google-api-translate-java} (\url{http://code.google.com/p/google-api-translate-java}).

The tests were done on 3rd May, 2014.\footnote{I think it's important to note the date of the tests, because the quality of online services might change overtime.}

\subsection{Microsoft Bing Translator}
\label{bing}
Another online service that we  tried is Microsoft Translator/Bing Translator. (In Microsoft's own materials, the system is usually called Bing Translator when referring to the website and Microsoft Translator when referring to the API. I will call it Microsoft Translator further.)

Microsoft Translator is very similar to Google Translate -- it is an online website with an easy GUI, and an additional paid API. Again, the team occasionally publishes some scientific papers, but is again otherwise proprietary.

In different experiment, we found out (non-scientifically), that for some language pairs, Microsoft Translator does more post-editation, that seemed a little rule-based (for example, better verb separation in English-to-German translation). However the system as a whole seems similarly statistical as Google Translate.


\subsubsection{API}
Again, we used Microsoft Translator API (confusingly marketed as a \uv{dataset} inside Windows Azure platform).

The API is slightly more complex than Google's API because of the auto-expiring token, but we used the example PHP script from the API documentation\footnote{\url{http://msdn.microsoft.com/en-us/library/hh454950.aspx}}.

The pricing is slightly different in Microsoft Translator than in Google Translate, but in general is slightly cheaper. First 2 million letters are for free, next 2 million are for about 40 US dollars.

\subsection{Yandex Translate}
\label{yandex}
Yandex (\url{http://www.yandex.ru}) is a Russian search portal that, according to its website\footnote{\url{http://company.yandex.com/}}, generates 61 percent of web search traffic in Russia.

Apart from being a search engine, Yandex offers a variety of other services. One of them is Yandex Translate (\url{http://translate.yandex.com})\footnote{Or \url{http://translate.yandex.ru} for Russian version} -- again, a simple website for automatized translations, similar to aforementioned Google Translate or Microsoft Translator.

I wanted to include Yandex Translate, because -- in theory -- as a Russian service, it could have better Russian language models and better Russian support in general.

\subsubsection{API}
%From the three online services, Yandex API is probably the simpliest to use. It uses a simple JSON interface, which requires an API key.
Yandex Translate also has a translation API. 
The API itself is absolutely free and is probably the easiest of the three online services to implement; however, it has strange and vaguely defined usage limits with no way of checking the actual usage.

In our experiments, the API simply stopped returning sentences after approximately 1 million characters per 24 hours. After 24 hour period, the API became usable again.

\section{Moses}
\label{moses}
Moses is an open-source machine translation toolkit with GPL licence, developed as a successor to the closed-source Pharaoh system.\footnote{See for example \cite{mosespaper} or \cite{moseslink} -- but Moses is used so often and so extensively that many other papers describing it can be found}

The system is very modular and very customizable, which makes it a bit harder to describe. In this section, I will try to describe our Moses set-up; first, I describe the overview of the entire system, and then I further describe some of the elements and our contributions.
%; bear in mind, however, that a completely different set-up is also possible.


\subsection{Pipeline overview}
At the start, we have a bilingual corpora of a given language pair, and bigger monolingual corpora of the target language.

The bilingual corpora has to be prepared by aligning the sentences, so every sentence has exactly one translation. We describe our corpora in the part ???

The sentences are then word-aligned, which means pairing words to their translations. We are using MGIZA++\footnote{See \cite{mgiza}}. From this word alignment, Moses learns a so-called \emph{phrase-based translation model}. From the monolingual corpora, we then learn a statistical \emph{language model} -- we use SRILM language model\footnote{See \cite{srilm}}.

Moses is then used for so-called \emph{decoding} of the information from both the language model and the translation model, which choses the best possible translation, using algorithms like beam-search.

However, for the best translation, we need to tune Moses parameters for optimal results. This is done using so-called \emph{minimum rate error rating} -- or MERT for short, which is tuning the parameters on a small separate development set.

After MERT tuning, we finally have working language model, translation model and Moses parameters, which is our complete translation system.

To reiterate, this is our Moses pipeline
\begin{pitemize}
\item getting sentence-aligned parallel Czech-Russian corpus, plus Russian monolingual corporus
\item world-alignment on parallel corpus
\item creating phrase-based translation model
\item creating Russian language model
\item tuning the parameters for Moses decoder
\end{pitemize}

%For managing input and output from the various steps, we use \emph{eman} system, which we transformed a little.

\subsection{Managing experiments}
For managing the steps described further, such as training models, we need some overaching system -- steps variously fail, don't compile, don't fit in memory, etc. We would also like to reuse partial results in more experiments.

Moses itself has built-in perl-based experiment managment system, called prosaically Experiment Management System (EMS). However, this system is not very widely used on UFAL.

Instead of EMS, we use instead another perl-based tool called eman (experiment manager). Eman is described well in \cite{eman} or at its website, \url{http://ufal.mff.cuni.cz/eman}.  

Eman breaks down experiment into so-called \uv{steps}. Step encapsulates an atomic part of an experiment and can be in one of a few various states. More importantly, step can be dependent on various other states; if a step fails, all steps dependent on it automatically fail. The whole experiment is then just another step, dependent on all the necessary substeps.

Step is represented by a directory in a playground directory. Step is created by copying a script, called \uv{seed}, from a library of seeds, to a new directory.

In my opinion, while eman itself is well written, I found the seeds themselves hard to read, too repetitive, and with large amount of code copied and pasted over. 

For that reason, I tried to rewrite the seeds as perl modules instead of bash scripts for more clarity and reusability. I am, however, not personally sure if my effort in this regard was successful. I decided to use the module \texttt{MooseX::Declare}\footnote{\url{http://search.cpan.org/~ether/MooseX-Declare-0.38/lib/MooseX/Declare.pm}}, which seemed to us like a modern way to write modules in perl. 

Unfortunately, that module is using very difficult-to-understand perl concepts and source code transformations through \texttt{Devel::Declare}, and as a result, it takes long to run and, perhaps worse, returns very confusing and undecypherable errors. 
So as a result of my rewrite, I have seeds with code that's probably easier to read and refractor, but on the other hand, it's slow and produces very opaque errors.

Author of \texttt{MooseX::Declare} is now recommending \texttt{Moops} module instead for declarative syntax; this module is, however, requiring perl version 14 and above, while on UFAL's network, only perl 10 is installed.

\subsection{Word alignment}
For word alignment, we are using MGIZA++\footnote{See \cite{mgiza}}, which is a GPL toolkit based on GIZA++\footnote{See \cite{giza}}, which is itself based on models, sometimes called IBM Model 1 to IBM Model 5\footnote{See \cite{ibm}}, which is itself based on expectation–maximization algorithm (EM).

IBM Models and the underlying EM algorithms are explained perfectly in Chapter~4 of \cite{koehn2010statistical} or in those slides by the same author -- \url{http://www.inf.ed.ac.uk/teaching/courses/mt/lectures/ibm-model1.pdf}.

GIZA++ is an implementation of those models. MGIZA++ is just its mu\-lti-thre\-aded variant, which makes the word alignment slightly faster.

\subsection{Phrase-extraction}
In this step, Moses takes the word alignment from the previous step and learns a so-called \uv{phrase table}.
Unlike word alignment, phrase extraction spans multiple words on every side in so-called \uv{phrases}.

Phrase table consists of list of phrases, their probabilities in both ways of translation, and their lexical weighting -- lexical weighting is the probability of the translated phrase counted by individual word pairs. The exact meaning of the numbers is well explained in \cite{koehn2003}.

The phrase-table defines a so-called \uv{translation model}. 

\subsection{Language model}
Language model is a part of the system, that tries to model the probability of a target language sentence alone. It's trained on a monolingual corpus.

We use SRILM, which is an open source language modeling toolkit. (Although it's open-source, it uses its own license, that allows free use only for non-commercial and educational purposes.) Current status of SRILM is described in \cite{srilm}, original design is described in \cite{srilm_old}. 

SRILM uses several models, one of them is n-gram word model, described well in \cite{koehn2010statistical}\footnote{chapter 7}. We use n-grams model to the order 3 with words and order 5 with tags (see section \ref{factors}). We smooth the models with Kreser-Ney smoothing with Chen and Goodman's modification\footnote{See \cite{chen} and \url{http://www.speech.sri.com/projects/srilm/manpages/ngram-discount.7.html}}. 


\subsection{Language model interpolation}
\label{interpol}
As described in the section XX, we had more than one monolingual Russian corpora, but we weren't sure of how high quality each of them were and how helpful it would be. For this reason, we used so-call interpolation (also called mixing).

Linear interpolation in general is described for example in \cite{gutkin}. On a separate heldout data, set of \emph{lambdas} are trained -- the resulting probabilities are then just the individual probabilities, multiplied by the lambdas and summed.

Linear interpolation is supported by Moses by undocumented script in the codebase, called \texttt{interpolate-lm.perl}, which in turn uses SRILM's undocumented AWK script \texttt{compute-best-mix.gawk} and SRILM's \texttt{ngram} with \texttt{-mix-lm} option\footnote{See \url{http://www.speech.sri.com/projects/srilm/manpages/ngram.1.html}}. 
Eman manager then uses these scripts in the \texttt{mixlm} seed.

We used a linear interpolation instead of log-linear interpolation simply because we didn't notice the option until later in the project.

\subsection{Factored translation}
\subsubsection{Overview}


The pipeline, described in the previous sections, translates phrases from the source language to the target language \uv{as is}. Only the exact phrases, found on the source side, can be translated to the exact phrases on the target side; and as they are decoded by Moses, only the phrases themselves are taken into account.

However, with morphologically rich languages such as Russian or Czech, this can result in worse translations because of the number of word forms and resulting data sparsity.
With so-called factored translation, we can add some morphological information while still keeping the main ideas of phrase-based translation. Factored translation was introduced in \cite{factored}.

With factored translation, 
phrased-based approach is extended with 
morphological (or other) information. We can add additional information (for example, lemma or morphological tag) to either side of the translation, on a word level -- this is called a \emph{factor}. Then, instead of training language models and/or translation models on the words alone, we train them on some combination of these factors and then, with the help of Moses that supports factored translation models, combine them together.

\subsubsection{Our factored translation experiments}
\label{factors}
In a separate set of experiments only on UMC data (this dataset is described in the section ??), we realized our Moses results have a high OOV rate\footnote{Out Of Vocabulary; how many words were untranslated due to not being found in the phrase table}; this is easily recognizable by Latin script appearing in Czech-to-Russian translation (or Cyrillics in the opposite direction). We then tried to compare several set-ups for factored tranlation to get lower OOV rate and higher BLEU scores.

\grafff{backoff}{Backoff model}{60}

We used a modified version of a set-up described for example in \cite{backoff} as \texttt{lemma backoff}. The set-up is illustrated on Figure~\ref{graf:backoff}, on the left.

The primary translation model is from full word on source side to the full word and morphological tag on target side. The backoff translation model is from lemma on source side to the full word and morphological tag on target side. Then we are using two language models, one for tags and one for words (both separately interpolated, as described in \ref{interpol}).

We were not using interpolated backoff, simply because regular backoff is easier to use with Moses. We were not using models that generate the words from lemma+tag, because we didn't have a working module for Russian morphological generation -- as a result, we can only get the word-forms found on the target side of the parallel training data.

%Primary, we are translating from full word to fill word and morphological tag; only as a backoff, we are translating from 


%However, since we did not have Russian morhpology fully working, we used only the system described as \texttt{lemma backoff} -- with the exception of not translating to lemma. 
%We were not using interpolated backoff, simply because regular backoff is easier to use with Moses.

%The main model translates from a word form on the source side to word form and tag on the target side. The backoff model translates from a lemma (or a stem -- see below) to form and tag on the target side.

For tagging Russian, we used TreeTagger software\footnote{\url{http://www.cis.uni-muenchen.de/~schmid/tools/TreeTagger/}, also see \cite{treetagger1} and \cite{treetagger2}} with a Russian parameter file\footnote{trained on a corpus created by Serge Sharoff, see \url{http://corpus.leeds.ac.uk/mocky/}}. TreeTagger is a closed-source software with a restrictive license, but for free for research purposes.

For Czech, we used tokenizer from UFAL project Treex (described further in section XXX) and for lemmatizing, we used morphological analyzer Morče (\url{http://ufal.mff.cuni.cz/morce/references.php}); however, as described further, in the final system we didn't actually use its output.

With further experimenting, we discovered that using not lemma on the source side, but a \emph{very crude} stem -- just using the first $n$ letters of a word -- gets better results.\footnote{Using stems instead of lemmas is suggested for example in \cite{stemy}. However, their stems are more linguistically motivated, while we just crudely take first few letters. It's actually debatable if our \uv{stems} can be called stems at all.} 
The model is illustrated on Figure~\ref{graf:backoff}, on the right.

\grafff{stem-plot-csru}{Comparison of various set-ups}{100}

The results of our experiment are seen on Figure~\ref{graf:stem-plot-csru} -- \emph{baseline} is original moses with no factors, \emph{1-lemma} and \emph{1-stem} are the \uv{backoff} models without the main model, and \emph{2-stem} and \emph{2-lemma} are the whole models with backoff.
 
We can see that stem with length 6 gets the best results. So, we used stemma with the length 6 in further experiments, such as the WMT submission \cite{mujpaper}.
\subsection{Recasing}
Our language and translation models are all trained on lowercased corpora. Because we evaluate BLEU as case-sensitive, we need to train a recaser that will convert the translated text from lower case back to upper case.

We could make a rule-based recaser, such as the ones that are included in Moses, however, we decided to train a statistical recaser. The recaser is basically a complete Moses model, trained as a translation from lower-cased corpus to a cased corpus, where any (case-sensitive) monolingual corpus can serve asa source for the language model -- where source language is the lowercased corpus and the target language is the original corpus. 

\section{Treex}
\label{tecto}
TODO
%limited to 1 million

%\chapter{Experiment setting}
\section{Statistical vs. rule-based translation systems}

Machine translation (MT) systems has historically used many different approaches. One way of classifying the approaches is on the axis of rule-based vs. statistical.

In general, we can re-use the descriptions, used by \cite{bojar}, which is as follows:
\begin{itemize}
\item rule based MT systems:
\begin{itemize}
\item use analysis, transfer and synthesis steps
\item use formal grammars
\item use hand-made dictionaries
\item have linguistic information hard-coded and therefore aren't language-agnostic
\end{itemize}
\item statistical MT systems
\begin{itemize}
    \item use more variants of outputs, rank them with some score, and choose the best one
    \item train internal dictionaries from big parallel data
    \item have more compact translation core, their inner working are less obvious
    \item use statistics instead of linguistic rules and therefore are more language-agnostic
\end{itemize}
\end{itemize}

However, with some of the modern systems, the distinction is not as clear-cut as we would like, for the purpose of our comparison. 
For example, statistical MT systems like Moses can get significantly better with some added linguistic information; 
on the other hand, systems like TectoMT, which can sometimes be classified as more rule-based, actually have big modules more or less based on statistics.

%In this thesis, I will present several existing solutions to Czech-Russian machine translation, describe their position on the aforementioned axis, present their history and discuss their results on a testing data.
\section{Development and testing data}

To compare several machine translation systems, we need some consistent data to show the deficiencies and/or improvements of different MT systems.

In general, we want to produce a system that's general enough and performs well on unseen sentences; for that reason, we should have -- in the case of statistical MT systems -- separated test set and training/development set.\footnote{As noted in for example \cite{koehn2010statistical}, p. 274, or \cite{bishop}, pg. 6}

We have tried to do that in our experiments with our data models. However, some systems evaluated in this thesis are only black-box systems (Google Translate, PC Translator). Especially with Google Translate, we cannot rule out the possibility that they already use some parts of the test/development data.


In general, we use two sets with two different domains - WMT data and Intercorp data.

\subsection{WMT}
One of the sets is the WMT test data. 
WMT (short for Workshop on Statistical Machine Translation) is an annual workshop, where various teams compete on a shared translation task with a shared test data.\footnote{See for example \cite{wmt_findings_2013}, or the rich history on \url{http://www.statmt.org}.}

As noted in \cite{wmt_findings_2013}, in 2013, Russian was added as one of the languages; it was still included in 2014. In both these years, the available data were divided into a test set and a development/training set.

The sentences in the training set are manually translated; for the year 2013, the sources are described in \cite{wmt_findings_2013}; for the year 2014, they are not yet formally described, but from an overview we can tell they used news sources again. The set contains both Czech and Russian.


Most Czech and Russian sentences in these sets are not a direct translation of each other though\footnote{Again, described in \cite{wmt_findings_2013}}; they are actually different translations of the same source sentences from various languages. 

Some of the source languages are Czech or Russian, so for those cases the data are indeed direct translations, but if we used only those sentences, the data would be significantly smaller.

However, if there is a different source language for both Czech and Russian side, the advantage of similarity of these two languages is lost -- we can actually see completely different language constructs and word order on Czech and Russian side, even when similar constructs and word order would be used when translation directly from Czech to Russian.

We extracted pairs of Czech and Russian from the test dataset; this dataset is 3000 sentences long for the year 2013 (WMT2013) and ??? sentences for the year 2014 (WMT2014). We use WMT2014 as a final test set, and WMT2013 as a development set, for fine-tuning the results.


\subsection{Intercorp}

Intercorp\footnote{Sometimes written as InterCorp} is a parallel corpus for many language pairs, including Czech and Russian. The history and other information is thoroughly described in \cite{intercorp}. 

The data itself is organized by source, and each data source is given an information of the original language, from which a given source is translated. We were able to extract just the data, that are either direct translations from Russian to Czech or vice versa, thanks to this information.

We also removed all the data, used in as a training data of our Moses model (see section ??)\footnote{this might sound illogical and backwards, but it's mainly because we built the Moses system first as a part of a different research with data from an older version of Intercorp corpus. See \cite{mujpaper}}. The resulting data is purely fictional novels, except for Jiří Levý's Art of Translation, which is a translation theory book. Interestingly, this is also the only book that has been translated from Czech to Russian and not the other way.

\jednatabulkan{icorpdata} { |l|l |l | l | }
{
         \hline
\textbf{Author}
&
\textbf{English name}
&
\textbf{Year}
&
\textbf{Sent.}

\\ \hline
Nikolai Ostrovsky &
How the Steel Was Tempered &
1936 &
9844
\\ \hline
Ilya Ilf, Yevgeni Petrov &
The Twelve Chairs &
1928 &
8525

\\ \hline

Mikhail Bulgakov &
The Master and Margarita &
1967 &
7124 
\\ \hline

Nikolai Nikolaevich Nosov &
 The Adventures of Neznaika and His Friends 
&1953-1954 &
3523




\\ \hline

Jiří Levý &
The Art of Translation &
1957 &
3149

\\ \hline

Aleksandr Solzhenitsyn
&
One Day in the Life of Ivan Denisovich
&
1962
&
3090

\\ \hline

Alexander Pushkin &
The Captain's Daughter 
&1863&
2984 
\\ \hline

Aleksandr Solzhenitsyn &
An Incident at Krechetovka Station &
1963&
1467 

\\ \hline
Aleksandr Solzhenitsyn &
Matryona's Place  
&1963
&
880

\\ \hline

} {Intercorp data} 


All the used novels are in the Table~\ref{tabulka:icorpdata}, \footnote{English transcriptions, English title translations and years of publication are taken from English wikipedia.}
sorted by the sentenced count.

As the reader can probably see, this dataset is markedly different from the first dataset. The data are bigger and are translated directly, on the other hand, the youngest book is from 1967 and the language itself is -- as a prosaic text -- harder to translate in general. Because the two corpora are too different, we did not try to join the WMT and the Intercorp sets.

From this set, we randomly selected 3000 sentences for final testing (IC-test); the rest is used as a development set (IC-dev).



\section{Experiments overview}
\label{sec:experiments}
\jednatabulkam{randsent1} { |X|X|X | }
{
\hline



Je mi teprve čtyřiadvacet let a nemohu prožit celý svůj život s legitimací invalidy práce a potloukat se po nemocnicích , když vím , že je to marné .   &   Мне всего двадцать четыре года , и я не могу доживать свой век с книжечкой инвалида труда , скитаться по лечебницам , зная , что это ни к чему .   &   Mne vsego dvadczat` chety're goda , i ya ne mogu dozhivat` svoj vek s knizhechkoj invalida truda , skitat`sya po lechebniczam , znaya , chto e`to ni k chemu . \\ \hline
Generál chodil po pokoji sem a tam , kouře svou pěnovku .   &   Генерал ходил взад и вперед по комнате , куря свою пенковую трубку .   &   General xodil vzad i vpered po komnate , kurya svoyu penkovuyu trubku . \\ \hline
" Možná že žádné brilianty neexistují ? "   &   - Может быть , никаких брильянтов нет ?   &   - Mozhet by't` , nikakix bril`yantov net ? \\ \hline
Nesměl při tom udělat chybu , vyžadovalo to stejnou přesnost , jako když se zaměřuje dělo .   &   Эта работа не допускала описки - так же как прицел орудия .   &   E`ta rabota ne dopuskala opiski - tak zhe kak pricel orudiya . \\ \hline
S hrdostí vzpomněl , jak snadno dobyl kdysi srdce krásné Heleny Baurové .   &   Он с гордостью вспомнил , как легко покорил когда-то сердце прекрасной Елены Боур .   &   On s gordost`yu vspomnil , kak legko pokoril kogda-to serdce prekrasnoj Eleny' Bour . \\ \hline





}{Random sentences and translations -- Intercorp}

\jednatabulkam{randsent2} { |X|X|X | }
{
\hline

Thiago Silva, který patří k nejlepším obráncům na světě, taky umožňuje ostatním vedle sebe růst.   &   Тьяго Силва, который является одним из лучших защитников в мире, также мотивирует всех двигаться вперед.   &   T`yago Silva, kotory'j yavlyaetsya odnim iz luchshix zashhitnikov v mire, takzhe motiviruet vsex dvigat`sya vpered. \\ \hline
"Dávali mi pět let života a už je to sedm," říká bez emocí na svém lůžku v domě pro paliativní péči Victor-Gadbois v Beloeil, kam přijel předešlý den.   &   "Мне давали пять лет, я прожил семь", - говорит он, между жизнью и смертью, лежа в кровати в приюте паллиативного ухода Виктор-Гадбуа в Белёй, куда прибыл накануне.   &   "Mne davali pyat` let, ya prozhil sem`", - govorit on, mezhdu zhizn`yu i smert`yu, lezha v krovati v priyute palliativnogo uxoda Viktor-Gadbua v Belyoj, kuda priby'l nakanune. \\ \hline
Opatrnost je ovšem na místě například na některých přemostěních, kde může být povrch namrzlý a kluzký.   &   Однако внимательность нужна, например, на мостах, где поверхность может быть намерзшая и скользкая.   &   Odnako vnimatel`nost` nuzhna, naprimer, na mostax, gde poverxnost` mozhet by't` namerzshaya i skol`zkaya. \\ \hline
Prostě je ignoruji.   &   Я просто не обращаю внимания.   &   Ya prosto ne obrashhayu vnimaniya. \\ \hline
Podle doktorky Christiane Martelové není quebecký zdravotnický systém dostatečně výkonný, aby zajistil přístup všech osob ke kvalitní paliativní péči, než bude možno souhlasit s provedením eutanazie.   &   По словам доктора Кристиан Мартель, система здравоохранения Квебека недостаточно эффективна, чтобы обеспечить право на паллиативный уход высокого качества до того, как будет разрешен переход к эвтаназии.   &   Po slovam doktora Kristian Martel`, sistema zdravooxraneniya Kvebeka nedostatochno e`ffektivna, chtoby' obespechit` pravo na palliativny'j uxod vy'sokogo kachestva do togo, kak budet razreshen perexod k e`vtanazii. \\ \hline


}{Random sentences and translations -- WMT2013}

For every translation system described, we tried to translate the same two development sets, described in the section above - WMT2013 and IC-dev. We describe the results on the sets, with a BLEU metric.

For illustration purposes, we have randomly selected 5 sentences from IC-dev and 5 from WMT-2013; the results are shown in Tables~\ref{tabulka:randsent1} and \ref{tabulka:randsent2}. 
On those, I want to demonstrate the conrete results of various machine translation systems. 
%I am also describing some common mistakes with selected sentences, demonstrating those mistakes.

I am presenting all Russian sentences with both original Cyrillic and GOST 7.79 RUS transliteration\footnote{See \cite{gost}}, for the convenience of the reader.


\subsection{Baseline}
\jednatabulkan{bleubase} { |r|r|r | }
{
\hline
&
intercorp&
WMT\\ \hline
BLEU & 0.77\% $\pm$ 0.06\%
&
0.83\% $\pm$ 0.16\%

\\ \hline
}{Baseline BLEU}

As a baseline for the translation, I use an automatic transliteration between Czech and Russian, mentioned above; to make transliteration possible, we also remove Czech diacritics.

Resulting BLEU is very low, as expected.

\chapter{Data}
In this chapter, I am describing both data, used for testing the systems, and training data used for training Moses and Treex(???) models.

\section{Testing data}
To compare several machine translation systems, we need some consistent data to show the deficiencies and/or improvements of different MT systems.

In general, we want to produce a system that's general enough and performs well on unseen sentences; for that reason, we should have -- in the case of statistical MT systems -- separated test set and training set.\footnote{As noted in for example \cite{koehn2010statistical}, p. 274, or \cite{bishop}, pg. 6}

We have tried to do that in experiments with systems that we control. However, the systems described in \ref{blackbox} are systems that we don't control. Especially with Google Translate, we cannot rule out the possibility that they already use some parts of the test data. The only way to absolutely prevent this would be to make our own test data, but we don't have enough resources to do that on a bigger scale.

In general, we use two test sets with two different domains - WMT data and Intercorp data.



\include{chap3_results}
%\chapter{Unrunnable systems}
\section{RUSLAN}

RUSLAN is a machine-translation system, developed between 1984 and 1988 at several departments of Charles University, Prague.

\subsection{Q-Systems}
Q-Systems (sometimes also Systems Q) -- Q stands for \uv{Quebec} -- are a tool for machine translation, developed at Montreal University by Alain Colmerauer, also the creator of Prolog.\footnote{See \cite{qsystems}.} 

Q-Systems are similarly declarative as Prolog, focusing more on the result than the procedural order of analysis. If there is any ambiguity, all the possibilities are explored \emph{in parallel}. In theory, this could make writing the lexical rules better; in reality, the rules are still quite unreadable, as will be seen later.

Q-Systems are not very widely used or widely worked with. One of the reasons might be the fact that all documentation is in French. However, \cite{nguyen2009systemes} describes modern-day experiments with Systems Q, also mentioning, that all Fortran implementations has been lost\footnote{Which would make UFAL's version, if it was working, quite unique.} and reimplementing it in C. I haven't been able to try this version.


\subsection{RUSLAN components}

Description of the system can be found in \cite{olivaruslan} or \cite{hajic1987} -- however, the reader has to bear in mind that not only the systems are severly dated, but so is their manual and description. \footnote{At least for me personally, the book \cite{olivaruslan} especially was hard to read and hard to find needed information in.} Contemporary (but not as detailed) description of the system can be found in \cite{recycled}.

The whole RUSLAN system has several components:
\begin{itemize}
\item preprocessing, written in Pascal
\item morphological analysis, using dictionary, written in Q-Systems and interpreted in Fortran IV
\item syntactico-semantical analysis, using morphology, also written in Q-Systems; this component uses FGD as its theoretical starting point
\item generation, also using Q-Systems
\item morphological synthesis, using Pascal
\end{itemize}

RUSLAN uses a Czech-to-Russian dictionary, written by hand in afforementioned Q-Systems. Dictionary item looks like this:

\begin{verbatim}
DLOUH==M(RS(+(*INT)),MI2289,DLINNYJ).
DLOUH==M(RS(-(*INT)),MI2276,DOLGIJ).
\end{verbatim}

This describes two possibilites of the translation of the word \uv{dlouhý} to Russian: the first is \uv{длинный} and the second is \uv{долгий}. They differ by the semantic feature INT (TODO: find out) they require or forbid from the word they depend on. 

More complex dictionary item looks like this:
\begin{verbatim}
C3ES3TIN
  ==Z(@(*A), MIO109, $(JAZYK), 
     2(POS, #($), &, $(MI28), $(C2ES2KIJ),
       1(=,@($), #($),$($))),
     1(=,@($),#($),$($))).
\end{verbatim}

This item translates the word \uv{čeština} to Russian words \uv{чешский язык} and also describes their relationship.

Maybe because memory was more expensive than today, all similar rules are in the dictionary without any comments, leaving only very difficult-to-decypher rules.

The rules for analysis are even less readable. Random examplo of two such rules are as follows:

\begin{verbatim}
1(B*, X*1, /, X*2, F*1(C*), X*3, /, X*4, @(V*), X*5, %(X*),
 I(*), 1(X*6, $($)), X*7)

1Z(A*9), (Z*2)
  == 1(D*, /, X*2, F*1/X*),/,@(V*),X*5,%(X*),1*,1(X*6,$($)),X*7,A*B,
     5(U*1, @(U*2), U*3, $(U*), 3(E*(Y*1), B*(C*), W*1, W*3, %(X*), 
        $(W*)),
     +1Z(A*9, Z*2)
       / -NON- (, + -DANS- X*9 -ET- +(V*) -HORS- X*9, +(VZT)
        -ET- -(V*) -HORS- X*9, *
        -ET- C* = S
        -ET- X*3,* -HORS- /,N(S), S(S), D(S), A(S), L(S), I(S)
        -ET- / -HORS- X*2, 2
        -ET- (, Y%1 = -NUL-
             -OU- E*(Y*1) -HORS- U*2,*
             -OU- E*(+(V*, *)) -HORS- U*2, *
             -OU- -NON- E*(-(V*)) -HORS- U*2, * .)
        -ET- (. E*(Y*1) *N
             -OU- H(B*(C*)) -DANS- U*1 .) .
\end{verbatim}

Of course this is all left without any comments, and the only comment for these two rules and about 20 more is \uv{RELATIVE CLAUSES ADJOINED TO THEIR HEADS}.

\subsection{Dictionary coverage of WebColl}

The dictionary contains about 8,000 lexical items. However, the domain of the translation and, therefore, the dictionary, was manuals for old computers from 1980's. 

In different experiments (\cite{florida}), we tried to measure how many nouns from the RUSLAN dictionary appear at all in a modern text.

For that, we used a monolingual Czech corpus WebColl, consisting of roughly 7 million sentences (114 million tokens)\footnote{See \cite{webcoll}; the corpus is also more thoroughly described in the chapter ????}.

RUSLAN dictionary has 2,783 nouns. In the WebColl corpus, from those nouns, 611 appear less than 10 times -- and from those, 412 don't appear \emph{at all}.

The reverse is similarly infavourable: from 39,434,505 nouns in the corpus, only 11,862,221 is in the dictionary.

\subsection{Experiments}

Despite the general un-maintainability of the RUSLAN code and despite the small dictionary, we tried to run the system on our test data.

However, all our experiments ended in some sort of error.

Because I am not able to code in neither Systems-Q nor FORTRAN (in which the Systems-Q interpreter is coded), I gave up on this experiment.


\section{Česílko}

Česílko is actually a name for two different machine translation systems with slightly different goals and, more importantly, slightly different structure.

\subsection{Česílko 1.0}
Česílko 1.0 was a system, developed in 2000, and was aimed for direct translation between Czech and Slovak and was intended to assist a translation memory\footnote{see \cite{cesilko1}}. The translation works lemma-by-lemma in a following fashion:
\begin{itemize}
\item morphological analysis of source Czech language
\item disambiguation
\item direct translation, lemma-by-lemma
\item morphological synthesis
\end{itemize}



The system is written in a mixture of C, C++ and Flex (fast lexical analyser generator). The code itself is not really well documented and modular, but that can be attributed to the age of some of the components -- despite the whole system being developed in 2000, some files seem to be as old as 1991.

This system itself is not very extendable from Slovak to Russian, because -- as we can see on the examples in the section \ref{sec:experiments} -- the sentences are not really translatable word-by-word.

For that reason I decided not to further experiment with Česílko 1.0 for Czech-to-Russian machine translation.

\subsection{Česílko 2.0}
Česílko 2.0 is a different project with similar goals, but using different frameworks and adding more transfer rules\footnote{See \cite{cesilko2}}. However, it has its own shares of problems, that prevented us to use it.

The system works in following fashion:
\begin{itemize}
\item \textbf{non-deterministic} morphological analysis of source Czech language
\item translation of lemmas
\item applying transfer rules by changing syntactic tree
\item morphological synthesis
\item ranking of all the generated sentences
\end{itemize}

Unlike Česílko 1.0, Česílko 2.0 uses a non-deterministic parser and explores all the possibilities in parallel. 

The more advanced transfer would, by itself, make the system more modular and extensionable for our purposes. However, the implementation details prevented us from doing so.

To illuminate why, let me focus a little on the technical details.

\subsubsection{Objective-C}
Objective-C is a very simple and elegant extension of C language, developed by Brad Cox in 1980s by adding Smalltalk features to C\footnote{See \cite{cocoa4}}. 

Objective-C is, in my opinion, very easy to learn and understand, at least compared to C++, its more popular counterpart.

Objective-C is not a proprietary language and is possible to compile with either gcc or Clang/LLVM compilers. However, what is proprietary is its most used standard library, Cocoa.

\subsubsection{Cocoa}
When Steve Jobs left Apple, he made a smaller company called NeXT. Among other things, they produced a proprietary operating system called NeXTSTEP, based on Unix.\footnote{For a more detailed history, see \url{https://developer.apple.com/legacy/library/documentation/Cocoa/Conceptual/CocoaFundamentals/WhatIsCocoa/WhatIsCocoa.html\#//apple\_ref/doc/uid/TP40002974-CH3-SW12}.}

This operating system used Objective-C as its standard language, and proprietary libraries, called OpenStep.\footnote{Despite the name, OpenStep is not open source -- the Open allude to the fact that its API specification was open.}

Several years later, Apple (now merged with NeXT) made its new version of Mac OS, called Mac OS X; this operating system was partially based on NeXTSTEP and also used some of its proprietary libraries, now renamed Cocoa.\footnote{The kernel of Mac OS X is open source, as is its \uv{underlying} operating system called Darwin -- however, this system does not contain Cocoa libraries.}

Cocoa is not the only library for Objective-C, but because Apple is the main investor in Objective-C-based systems, it's a de-facto standard library.

\subsubsection{GNUstep}
GNUstep is a free re-implementation of OpenStep/Cocoa.\footnote{See \url{http://www.gnustep.org/}}.

Its development started in the NeXTSTEP days; however, it still hasn't met feature parity with Cocoa's OS X.

Aaron Hillegass in 2nd edition of his popular book \emph{Cocoa Programming on Mac OS X} discouraged people from using GNUStep. He redacted this note in later versions of the book, perhaps because of protests from GNUstep developers\footnote{\url{http://www.gnustep.org/resources/BookClarifications.html}}, but in my opinion, his notes are still valid.

GNUstep implementations are very often buggy and not feature-complete with Cocoa and, most unfortunately, unpredictable. This is what hurt us with Česílko 2.0.

\subsubsection{Cocoa and Česílko}
When Petr Homola was writing Česílko 2.0, he decided to use Cocoa and Objective-C for development.

On Mac OS X, this configuration is just fine; however, on Linux, where we wanted to run the MT systems, this creates unpredictable results.

In my experiments with Czech-to-Slovak translations, I noticed that on Mac OS X, there are about 5-times more sentences generated, than on Linux -- while the program was compiled from the same sources.

After thorough inspection, I found out the error was in GNUstep implementation of NSDictionary -- Cocoa's implementation of associative array\footnote{\url{https://developer.apple.com/library/mac/documentation/Cocoa/Reference/Foundation/Classes/NSDictionary\_Class/Reference/Reference.html}} -- that in some unpredictable cases for two equal NSStrings returns two different values. As a result, one of the modules returned wrong inflection patterns for a number of words and the morphological analyzer then returned only a fraction of the results.

After workaround for this issue, the system returned same results on both OS X and Linux. However, I am not at all confident there aren't more similar issues in GNUstep to further develop the system for Russian; when the very basic frameworks themselves are unstable and unreliable, the development ceases to make sense.

It's possible the same issue plagued the authors of the paper \cite{evalquality_cesilko} -- it's unprobable the BLEU of the correctly working system would be that low, when in \cite{cesilko2}, the results of Česílko 2.0 were slightly better, than in Česílko 1.0. 

%\chapter{Black-box systems}

\section{PC Translator}

\subsection{Description}

PC Translator is a commercial translation system from a Czech company LangSoft (\url{http://www.langsoft.cz/translator.htm}). PC Translator can translate several language pairs with Czech on either source or target side.

Authors of PC Translator don't publish any papers or other literature about the system -- what can we tell about its functionality is gathered only from its promotional website and from the experiments with the software itself.

PC translator seems to be purely rule-based. The system seems to work in following steps:

\begin{itemize}
\item some (probably rule-based) morphological analysis of the source language
\item translation of the lemmas from source language to target language by searching in a large dictionary
\item some synthesis of morphological information and the translated lemma
\end{itemize}

The system doesn't seem to do any kind of reordering. It also doesn't seem to do any analysis on a deeper level, like sentence constituents. Some of the phrases in the dictionary are longer than one word, but not too many of them.

One of the advantages of PC Translator is its large dictionary -- however, the dictionary is sometimes choosing very odd and inprobable choices when disambiguating between more possible translations. For example, the English sentence \uv{I like dogs} is translated as \uv{Mám rád kleště}, because the term \uv{dog} can be also translated as \uv{kleště}\footnote{from Collins' Dictionary: \uv{dog -- 5. a mechanical device for gripping or holding, esp one of the axial slots by which gear wheels or shafts are engaged to transmit torque}}.

According to its marketing materials, PC Translator v14 uses a Czech-Russian dictionary with above 650.000 words.

\subsection{Experiments}
We found out it's not easy to automatize translating with PC Translator. Its GUI is suited for translating by hand, sentence-by-sentence, but not for automated translation of thousands of sentences. Also, by definition, Windows GUI is harder to automate on Linux machine from a script.

However, we were able to do some work-around, with the help of VMWare Player virtualization software (\url{http://www.vmware.com/cz/products/player}) and AutoHotkey GUI scripting software (\url{http://www.autohotkey.com/}). Our workflow therefore is:

\begin{itemize}
\item on Linux machine, encode the source from UTF-8 to windows-friendly encoding
\item encode the source as HTML code
\item start a virtual machine with PC Translator pre-installed
\item on the start of the virtual machine, run AutoHotkey script from an outer-machine folder (thanks to VMWare shared folders and Windows Startup scripts)
\item via this AutoHotkey script, run PC Translator and click on \uv{translate file} feature 
\item translate the HTML file (also shared in the VMWare shared folder)
\item turn off the virtual machine
\item turn the file back from HTML and Windows encodings back to UTF-8
\end{itemize}

The HTML part is needed because PC Translator had some problems with translating ordinary text files, plus we can pair the translated sentences better thanks to \texttt{id} parameters in \texttt{div} tags.

We used the newest version of PC Translator available at the time, which is PC Translator v14.



\subsection{Results}

\jednatabulkan{bleutrans} { |r|r|r | }
{
\hline
&
intercorp&
WMT\\ \hline
BLEU & 5.02\% $\pm$ 0.01\%
&
6.74\% $\pm$ 0.32\%

\\ \hline
}{PC translator BLEU}

\jednatabulka{transsent1} { |X|X|X | }
{
\hline
Je mi teprve čtyřiadvacet let a nemohu prožit celý svůj život s legitimací invalidy práce a potloukat se po nemocnicích , když vím , že je to marné .   &   Есть мне только двадцать четыре рейс и не могу пережив весь свой жизнь с удостоверение личности инвалидами труд и болтаться по больницах , когда знаю , что это бесполезные .   &   Est` mne tol`ko dvadczat` chety're rejs i ne mogu perezhiv ves` svoj zhizn` s udostoverenie lichnosti invalidami trud i boltat`sya po bol`niczax , kogda znayu , chto e`to bespolezny'e .\\ \hline
Generál chodil po pokoji sem a tam , kouře svou pěnovku .   &   Генерал ходил по комнате туда - сюда , дыма свою пенковая трубка .   &   General xodil po komnate tuda - syuda , dy'ma svoyu penkovaya trubka .\\ \hline
" Možná že žádné brilianty neexistují ? "   &   " возможно никакие бриллианты отсутствовать ?    &   " vozmozhno nikakie brillianty' otsutstvovat` ? \\ \hline
Nesměl při tom udělat chybu , vyžadovalo to stejnou přesnost , jako když se zaměřuje dělo .   &   Обмен причём ошибиться , требовало то такой же аккуратность , как когда специализируется пушка .   &   Obmen prichyom oshibit`sya , trebovalo to takoj zhe akkuratnost` , kak kogda specializiruetsya pushka .\\ \hline
S hrdostí vzpomněl , jak snadno dobyl kdysi srdce krásné Heleny Baurové .   &   С гордостью вспомнил , как ходко захватил когда - то сердце красивое Елена Баурове .   &   S gordost`yu vspomnil , kak xodko zaxvatil kogda - to serdce krasivoe Elena Baurove .\\ \hline

}{PC Translator demonstration - InterCorp}
\jednatabulka{transsent2} { |X|X|X | }
{
\hline
Thiago Silva, který patří k nejlepším obráncům na světě, taky umožňuje ostatním vedle sebe růst.   &   Тhиаго Силва, какой принадлежать к лучшим защитник в мире, тоже даёт возможность остальным плечом к плечу рост.   &   Thiago Silva, kakoj prinadlezhat` k luchshim zashhitnik v mire, tozhe dayot vozmozhnost` ostal`ny'm plechom k plechu rost.\\ \hline
"Dávali mi pět let života a už je to sedm," říká bez emocí na svém lůžku v domě pro paliativní péči Victor-Gadbois v Beloeil, kam přijel předešlý den.   &   "давали мне пять лет жизни и уже это семь," говорит минус эмоций в своем гнезде в доме для башка опеку Victor-Гадбоис зажечься Бєлоєил, куда доехал предшествующий день.   &   "davali mne pyat` let zhizni i uzhe e`to sem`," govorit minus e`mocij v svoem gnezde v dome dlya bashka opeku Victor-Gadbois zazhech`sya Bєloєil, kuda doexal predshestvuyushhij den`.\\ \hline
Opatrnost je ovšem na místě například na některých přemostěních, kde může být povrch namrzlý a kluzký.   &   Осторожность есть конечно на месте например на некоторых перекрытие, где может быть покрытие замёрзший и сальный.   &   Ostorozhnost` est` konechno na meste naprimer na nekotory'x perekry'tie, gde mozhet by't` pokry'tie zamyorzshij i sal`ny'j.\\ \hline
Prostě je ignoruji.   &   Запросто есть игнорирую.   &   Zaprosto est` ignoriruyu.\\ \hline
Podle doktorky Christiane Martelové není quebecký zdravotnický systém dostatečně výkonný, aby zajistil přístup všech osob ke kvalitní paliativní péči, než bude možno souhlasit s provedením eutanazie.   &   По врачи Christiane Мартєлове нет qуєбєцкэ медицинский комплекс достаточно производительный, чтобы обеспечил подход всех» личностей к качественный башка опеку, нежели можно будет согласиться с проведением єутаназиє.   &   Po vrachi Christiane Martєlove net quєbєczke` medicinskij kompleks dostatochno proizvoditel`ny'j, chtoby' obespechil podxod vsex» lichnostej k kachestvenny'j bashka opeku, nezheli mozhno budet soglasit`sya s provedeniem єutanaziє.\\ \hline


}{PC Translator demonstration - WMT2013}

We can see the results are not ideal. The sentences are evidently translated word-by-word, without any regard for the whole sentence, or the correct word order.

Because the languages are similar, the morphological information is sometimes transferred correctly -- for example, \uv{s hrdostí vzpomněl} -> \uv{с гордостью вспомнил} -- but sometimes, very incorrectly -- for example, genitive \uv{překrásné} is not with the correct ending as \uv{-ой}, but incorrectly as \uv{-ое}.

It's obvious the dictionary is quite big and complete, so a lot of the words were translated somehow correctly; however, some of the words had strange ?????????


\section{Google Translate}
\subsection{Description}
Google Translate is a popular free online translation service by Google, an American web search giant (\url{http://translate.google.com}). 
Although Google is producing many academic papers on machine translation, the whole system is still proprietary and we cannot fully inspect it, as in the case of PC Translator.

Google Translate uses mostly statistical approach to machine translation, see for example \cite{och}\footnote{F. J. Och is a head of Google Translate group in Google}. Its results are often seen as \uv{state-of-the-art} in machine translation.

However, thanks to its purely statistical approach, it either needs huge amounts of data for every language pair, or it needs to use so-called \uv{pivot languages}\footnote{See for example \cite{koehn2010statistical}} -- in the case of Google Translate, it's usually English; specific English word order and English idioms are then re-translated into the target language and sometimes introduce downright wrong translations.

\subsection{Experiments}
To automate Google Translate, we cannot use the website itself, simply because pasting tens of thousands of lines into a browser window usually crashes the browser itself.

There are some workarounds around this, such as \uv{faking} browser environment using some automation tools and/or libraries, but we used more stable option.

Google Translate, apart from being a website, has a paid translation API\footnote{\url{https://developers.google.com/translate/?hl=cs}}. We figured out it's not too expensive for our testing purposes, so we ended up paying for the API.\footnote{The cost is measured per character on the source side. We used about 3 million characters and paid about 60 dollars. This is rather high for any repeated experiments, but not that high for one-time translation.}

We used an unofficial Java library for Google Translate API, called prosaically \uv{google-api-translate-java}\footnote{https://code.google.com/p/google-api-translate-java/}.

The tests were done on 3rd May, 2014.\footnote{I think it's important to note the date of the tests, because the quality of online services might change overtime.}

\subsection{Results}


\jednatabulkan{bleugoog} { |r|r|r | }
{
\hline
&
intercorp&
WMT\\ \hline
BLEU & 8.79\% $\pm$ 0.15\%
&
17.96\% $\pm$ 0.32\%

\\ \hline
}{Google translate BLEU}

The first thing to note is the radically different results for Intercorp and WMT corpus. This disprepancy is not repeated in our other experiments, where both sets have BLEU that's closer to each other.

This might be caused by two reasons. One reason might be that the complete WMT data are simply already included in Google's training data, while Intercorp is not. It's not that far-fetched theory: WMT is one year old already, and Google is one of WMT sponsors. On the other hand, in this case, the results should be even better than they actually are -- Google's results are, after all, worse, than our Moses experiments in chapter ??.

Other reason might be that Google is domain specificity of Google's training data. It's possible that Google is generally using far more newspaper data for training than just fiction, which might disadvantage Google's system in Intercorp case. Since we don't know Google's training data structure, we can't know for sure.



\jednatabulkam{googsent1} { |X|X|X | }
{
\hline

Je mi teprve čtyřiadvacet let a nemohu prožit celý svůj život s legitimací invalidy práce a potloukat se po nemocnicích , když vím , že je to marné .   &   Я всего двадцать четыре года, и я живу свою жизнь с купоном недействительным работы и болтаться в больнице, зная, что это бесполезно.   &   Ya vsego dvadczat` chety're goda, i ya zhivu svoyu zhizn` s kuponom nedejstvitel`ny'm raboty' i boltat`sya v bol`nice, znaya, chto e`to bespolezno.\\ \hline
Generál chodil po pokoji sem a tam , kouře svou pěnovku .   &   Генеральный ходил по комнате взад и вперед, его дым пены.   &   General`ny'j xodil po komnate vzad i vpered, ego dy'm peny'.\\ \hline
" Možná že žádné brilianty neexistují ? "   &   "Возможно, не алмазы там?"   &   "Vozmozhno, ne almazy' tam?"\\ \hline
Nesměl při tom udělat chybu , vyžadovalo to stejnou přesnost , jako když se zaměřuje dělo .   &   Он не мог ошибиться в этом, это требуется такой же точности, как при фокусировке пушки.   &   On ne mog oshibit`sya v e`tom, e`to trebuetsya takoj zhe tochnosti, kak pri fokusirovke pushki.\\ \hline
S hrdostí vzpomněl , jak snadno dobyl kdysi srdce krásné Heleny Baurové .   &   Мы с гордостью вспомнил, как легко покорил когда-то сердце прекрасной Елены Баур.   &   My' s gordost`yu vspomnil, kak legko pokoril kogda-to serdce prekrasnoj Eleny' Baur.\\ \hline


}{Google Translate demonstration -- Intercorp}


\jednatabulkam{googsent2} { |X|X|X | }
{
\hline

Thiago Silva, který patří k nejlepším obráncům na světě, taky umožňuje ostatním vedle sebe růst.   &   Тьяго Силва, один из лучших защитников в мире, также позволяет другой параллельный рост.   &   T`yago Silva, odin iz luchshix zashhitnikov v mire, takzhe pozvolyaet drugoj parallel`ny'j rost.\\ \hline
"Dávali mi pět let života a už je to sedm," říká bez emocí na svém lůžku v domě pro paliativní péči Victor-Gadbois v Beloeil, kam přijel předešlý den.   &   "Они дали мне пять лет жизни, и это семь", говорит он без эмоций на его постели у себя дома для паллиативной помощи Виктор-Гадбуа в Beloeil, куда они прибыли в предыдущий день.   &   "Oni dali mne pyat` let zhizni, i e`to sem`", govorit on bez e`mocij na ego posteli u sebya doma dlya palliativnoj pomoshhi Viktor-Gadbua v Beloeil, kuda oni priby'li v predy'dushhij den`.\\ \hline
Opatrnost je ovšem na místě například na některých přemostěních, kde může být povrch namrzlý a kluzký.   &   Внимание, однако, находится в месте, например, некоторые перемычки, где поверхность может быть ледяной и скользкий.   &   Vnimanie, odnako, naxoditsya v meste, naprimer, nekotory'e peremy'chki, gde poverxnost` mozhet by't` ledyanoj i skol`zkij.\\ \hline
Prostě je ignoruji.   &   Просто игнорируйте их.   &   Prosto ignorirujte ix.\\ \hline
Podle doktorky Christiane Martelové není quebecký zdravotnický systém dostatečně výkonný, aby zajistil přístup všech osob ke kvalitní paliativní péči, než bude možno souhlasit s provedením eutanazie.   &   По словам доктора Кристиана Martel Квебеке система здравоохранения не является достаточно мощным, чтобы обеспечить доступ для всех людей на высококачественной паллиативной помощи, прежде чем он может согласиться проводить эвтаназию.   &   Po slovam doktora Kristiana Martel Kvebeke sistema zdravooxraneniya ne yavlyaetsya dostatochno moshhny'm, chtoby' obespechit` dostup dlya vsex lyudej na vy'sokokachestvennoj palliativnoj pomoshhi, prezhde chem on mozhet soglasit`sya provodit` e`vtanaziyu.\\ \hline


}{Google Translate demonstration -- WMT2013}

NĚJAKÉ HODNOCENÍ VÝSLEDKŮ - já osobně to nedokážu vůbec zhodnotit.

\section{Microsoft Bing Translator}
\subsection{Description}
Another online service that we decided to try is Microsoft Translator/Bing Translator. (In Microsoft's own materials, the system is usually called Bing Translator when referring to the website and Microsoft Translator when referring to the API.)

Microsoft Translator is very similar to Google Translate -- it is an online website with an easy GUI, and an additional paid API. Again, the team occasionally publishes some scientific papers, but is again otherwise proprietary.

In different experiment, we found out (non-scientifically), that for some language pairs, Microsoft Translator does more post-editation, that seemed a little rule-based (for example, better verb separation in English-to-German translation). Therefore, we decided to try it alongside Google Translate in our experiments.


\subsection{Experiments}
Again, we used Microsoft's Translator API (marketed as part of Windows Azure).

The API is slightly more complicated than Google's API because of the auto-expiring token, but we used the example PHP script from the API documentation\footnote{\url{http://msdn.microsoft.com/en-us/library/hh454950.aspx}}.

The pricing is slightly different in Microsoft Translator than in Google Translate, but in general is slightly cheaper. First 2 million letters are for free, next 2 million are for about 40 US dollars.

%\chapter{Moses}

\section{General description}
Moses is an open-source machine translation toolkit with GPL licence, developed as a successor to the closed-source Pharaoh system.\footnote{See for example \cite{mosespaper} or \cite{moseslink} -- but Moses is used so often and so extensively that many other papers could be found}

The system is very modular and very customizable, which makes it a bit harder to describe. In this section, I will try to describe our Moses set-up; bear in mind, however, that a completely different set-up is also possible.


\subsection{Pipeline overview}
At the start, we have a bilingual corpora of a given language pair, and bigger monolingual corpora of the target language.

The bilingual corpora has to be prepared by aligning the sentences, so every sentence has exactly one translation. We describe our corpora in the part ???

The sentences are then word-aligned, which means pairing words to their translations. We are using MGIZA++\footnote{See \cite{mgiza}}.

From this word alignment, Moses learns a so-called \emph{phrase-based translation model}.

From the monolingual corpora, we then learn a statistical language model. We use SRILM language model\footnote{See \cite{srilm}}.

Moses is then used for so-called \emph{decoding} of the information from both the language model and the translation model, which choses the best possible translation, using things like beam-search.

However, for the best translation, we need to tune Moses parameters for optimal results. This is done using so-called \emph{minimum rate error rating} -- or MERT for short, which is tuning the parameters on a small separate development set.

After MERT tuning, we finally have working language model, translation model and Moses parameters, which is our complete translation system.

To reiterate, this is our Moses pipeline
\begin{pitemize}
\item getting sentence-aligned parallel Czech-Russian corpus, plus Russian monolingual corporus
\item world-alignment on parallel corpus
\item creating phrase-based translation model
\item creating Russian language model
\item tuning the parameters for Moses decoder
\end{pitemize}

For managing input and output from the various steps, we use \emph{eman} system, which we transformed a little.

\subsection{Factored translation}

The pipeline, described in the previous section, translates phrases from the source language directly to the target language.

However, with morphologically rich languages such as Russian or Czech, this can result in too low results. With so-called factored translation, we can add some morphological information while still keeping the main ideas of phrase-based translation.

With factored translation, we add some morphological (or other) information to either source, target or both -- for example, lemma or morphological tag -- this is called a \emph{factor}. Then, instead of making language models and/or translation models on the words alone, we train them on some combination of these factors and then somehow put them together.


\section{Our Moses setup}
In this section, I am describing our exact setup and our data.

\subsection{Parallel data}
\subsubsection{Intercorp}
In general, we described Intercorp corpus in \ref{intercorp_p1}.


We created the Moses system before we started evaluating the system on the evaluation corpus, as described in Section \ref{intercorp_p1}. 
The data we had access to were more limited at that time: they were from the previous version of Intercorp, didn't have all the metadata, such as the publishing dates and information about the original translation source, and were shuffled. However, we can reconstruct the metadata by comparing this shuffled data with the newest version of intercorp.


TODO: popis intercorp dat

\subsubsection{Subtitles}
Another set of data that we used were subtitles from OpenSubtitles database.

In a separate FilmTit project\footnote{???citovat ročníkový projekt/dokumentaci k němu?}, we tried to make a machine translation project for subtitle translation from English to Czech.

For that project, we were given access to the set of subtitles from the server OpenSubtitles (\url{http://opensubtitles.org}). This dataset was not a pair of aligned sentences, not even a pair of aligned \emph{files}; we were given just a set of SRT files, and a table which paired each of those files with a movie (identified by IMDB number). Each movie usually has more subtitle files.

Subtitle files have the sentences paired with timestamps.

From a set of SRT files paired with a movie, we selected just one Czech and just one English SRT file which we find most similar, based on the timestamps.
From the pair of the files, we then extract the sentences that have the most similar timestamps.

These two pairings -- pairings of subtitle files and pairing of the actual lines -- are non-trivial task, and require a \uv{tolerance} -- how different can the times of a sentence be to be still paired together.

Higher tolerance produces bigger corpus with more errors, while lower tolerance produces smaller, but more correct corpus.

When we were experimenting on the afforementioned project, we found out, that the best results (tested both on another movie subtitles and a different corpus) are -- without exception -- with \emph{bigger corpus} and \emph{higher tolerance}. Even when that introduced a lot of incorrect sentence pairs, the overall results were still better with the biggest possible corpus.

For Czech-to-Russian experiments, we were given another, similar set of subtitels from OpenSubtitles. The set was naturally much smaller than with English and Czech, but we were still able to use the same algorithms to build a parallel corpora.

We again used the highest possible tolerance, and therefore surely introduced a lot of errors. Unfortunately, for a lack of time, we weren't able to replicate the experiments for the ideal tolerance here, however, we hoped that the results would be similar than in English-to-Czech translation, that is, the biggest corpus resulting in the best translation.

\subsubsection{UMC corpus}
UMC (UFAL Multilingual Corpus) is a Czech-English-Russian corpus\footnote{See \cite{umc}}. The data are downloaded from Project Syndicate, a Prague-based non-profit news organization, translating news and opinions from around the world.

TODO: počty vět a tokenů, jediná věc, co o UMC můžu napsat

\subsubsection{Wiki titles}
We also extracted all of the titles from the Czech and Russian wikipedia, that correspond to each other.

Those are usually only noun phrases and the main word is usually in nominative singular, so the morphology isn't that rich -- however, we hope that the model will learn some phrases needed for the translation of the named entities.

\subsection{Monolignual data}
\subsubsection{News Crawl}
The largest part of our monolingual data is corpus from WMT workshops that they call \uv{News Crawl}.

According to \cite{wmt_findings_2009}, WMT workshop has been continously crawling web articles since 2007 for making test sets. This allowed them to make a big, randomized corpus from all these sources.


\subsection{Managing experiments}
For managing the steps described further, such as training models, we need some overaching system -- steps variously fail, don't compile, don't fit in memory, etc., plus we would like to reuse partial results in more experiments.

Moses itself has built-in perl-based experiment managment system, called prosaically Experiment Management System (EMS). However, this system is not very widely used in on UFAL.

Instead of EMS, we use instead another perl-based tool called eman (experiment manager). Eman is described well in \cite{eman} or at its website, \url{http://ufal.mff.cuni.cz/eman}.  

Eman breaks down experiment into so-called \uv{steps}. Step encapsulates an atomic part of an experiment and can be in one of a few various states. More importantly, step can be dependent on various other states; if a step fails, all steps dependent on it automatically fail. The whole experiment is then just another step, dependent on all the necessary substeps.

Step is represented by a directory in a playground directory. Step is created by copying a script, called \uv{seed}, from a library of seeds, to a new directory.

In our opinion, eman itself is well written, while we found the seeds themselves hard to read, too repetitive, and with large amount of code copied and pasted over. We tried to rewrite the seeds as perl modules, so the reusability is higher.

I am not personally sure if this effort was successful. We decided to use MooseX::Declare, which seemed to us like a modern way to write modules in perl. 

Unfortunately, MooseX::Declare is using very difficult-to-understand perl concepts and source code transformations, and as a result, it takes long to run and, perhaps worse, returns very confusing and undecypherable errors. So as a result of our refractoring, we have seeds that have code that's probably easier to read and refractor, but on the other hand, it's slow and produces very opaque errors.

\subsection{Word alignment}
For word alignment, we are using MGIZA++\footnote{See \cite{mgiza}}, which is a GPL toolkit based on GIZA++\footnote{See \cite{giza}}, which is itself based on models, sometimes called IBM Model 1 to IBM Model 5\footnote{See \cite{ibm}}, which is itself based on expectation–maximization algorithm (EM).

IBM Models and the underlying EM algorithms are explained perfectly in Chapter~4 of \cite{koehn2010statistical} or in those slides by the same author -- \url{http://www.inf.ed.ac.uk/teaching/courses/mt/lectures/ibm-model1.pdf}.

GIZA++ is an implementation of those models. MGIZA++ is just its multi-threaded variant, which makes the word alignment slightly faster.

\subsection{Phrase-extraction}
In this step, Moses takes the word alignment from the previous step and learns a so-called \uv{phrase table}.
Unlike word alignment, phrase extraction spans multiple words on every side in so-called \uv{phrases}.

Phrase table consists of list of phrases, their probabilities in both ways of translation, and their lexical weighting -- lexical weighting is the probability of the translated phrase counted by individual word pairs. The exact meaning of the numbers is well explained in \cite{koehn2003}.

The phrase-table - defines a so-called \uv{translation model}. 

\subsection{Language model}
Language model is a part of the system, that tries to model the probability of a target language sentence alone. It's trained


% Ukázka použití některých konstrukcí LateXu (odkomentujte, chcete-li)
% \include{example}


\chapter*{Conclusion}
\addcontentsline{toc}{chapter}{Conclusion}

I have automated, built, improved, demonstrated and compared (both by human annotators and by automated metrics) several translation systems, both phrase-based and rule-based, between Czech and Russian.

From the systems I have tried, phrase-based translation systems are simply easier to build and give better results.

TectoMT as a more hybrid system shows promise, but with this language pair, the work is only starting; however, it is telling, that it's probably easier to build a new system based on Moses that reaches about the same translation quality as \uv{state-of-the-art} systems, than it would be with TectoMT -- and impossible with purely rule-based systems.

%However, this result can't be taken as final, because the work on the 

\subsubsection{Future work}
The first future work, as already mentioned in \ref{future:morpho}, should probably be a better Russian parser and a better Russian morphology. This would allow us to experiment more with post-editing; we could also use it in factored translation models in Moses; and of course it would allow us to build better models with TectoMT.


%%% Seznam použité literatury
\include{bibliography}

%%% Tabulky v diplomové práci, existují-li.
\chapwithtoc{List of Tables}

%%% Použité zkratky v diplomové práci, existují-li, včetně jejich vysvětlení.
\chapwithtoc{List of Abbreviations}

%%% Přílohy k diplomové práci, existují-li (různé dodatky jako výpisy programů,
%%% diagramy apod.). Každá příloha musí být alespoň jednou odkazována z vlastního
%%% textu práce. Přílohy se číslují.
\chapwithtoc{Attachments}

\openright
\end{document}

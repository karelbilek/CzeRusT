%%% Hlavní soubor. Zde se definují základní parametry a odkazuje se na ostatní části. %%%

%% Verze pro jednostranný tisk:
% Okraje: levý 40mm, pravý 25mm, horní a dolní 25mm
% (ale pozor, LaTeX si sám přidává 1in)
\documentclass[12pt,a4paper]{report}
\setlength\textwidth{145mm}
\setlength\textheight{247mm}
\setlength\oddsidemargin{15mm}
\setlength\evensidemargin{15mm}
\setlength\topmargin{0mm}
\setlength\headsep{0mm}
\setlength\headheight{0mm}
% \openright zařídí, aby následující text začínal na pravé straně knihy
\let\openright=\clearpage

%% Pokud tiskneme oboustranně:
% \documentclass[12pt,a4paper,twoside,openright]{report}
% \setlength\textwidth{145mm}
% \setlength\textheight{247mm}
% \setlength\oddsidemargin{15mm}
% \setlength\evensidemargin{0mm}
% \setlength\topmargin{0mm}
% \setlength\headsep{0mm}
% \setlength\headheight{0mm}
% \let\openright=\cleardoublepage

%% Použité kódování znaků: obvykle latin2, cp1250 nebo utf8:
%\usepackage[utf8]{inputenc}
%\usepackage[T2A]{fontenc}
%\usepackage[russian,czech,english]{babel}

%\usepackage{fontspec}

%\usepackage{xunicode,fontspec,xltxtra}
%\usepackage[english]{polyglossia}
%\setotherlanguages{russian} % set as "other" so English hyphenation active


\usepackage{libertine}
\setmonofont{Anonymous Pro}

%% Ostatní balíčky
\usepackage{verbatim}
\usepackage{graphicx}
%\usepackage{amsthm}

%% Balíček hyperref, kterým jdou vyrábět klikací odkazy v PDF,
%% ale hlavně ho používáme k uložení metadat do PDF (včetně obsahu).
%% POZOR, nezapomeňte vyplnit jméno práce a autora.
\usepackage[unicode]{hyperref}   % Musí být za všemi ostatními balíčky
\hypersetup{pdftitle=A Comparison of Methods of Czech-to-Russian Machine Translation}
\hypersetup{pdfauthor=Karel Bílek}



\usepackage{tabularx}

\usepackage{polyglossia}
\setmainlanguage{english}
\setotherlanguage{czech}

\usepackage{float}

\usepackage[
   backend=bibtex8      % if we want unicode 
  ,style=iso-authoryear % or iso-numeric for numeric citation method          
  ,babel=other        % to support multiple languages in bibliography
  ,sortlocale=cs_CZ   % locale of main language, it is for sorting
  %,bibencoding=UTF8   % this is necessary only if bibliography file is in different encoding than main document
]{biblatex}


\bibliography{library}
%%% Drobné úpravy stylu
% Tato makra přesvědčují mírně ošklivým trikem LaTeX, aby hlavičky kapitol
% sázel příčetněji a nevynechával nad nimi spoustu místa. Směle ignorujte.
\makeatletter
\def\@makechapterhead#1{
  {\parindent \z@ \raggedright \normalfont
   \Huge\bfseries \thechapter. #1
   \par\nobreak
   \vskip 20\p@
}}
\def\@makeschapterhead#1{
  {\parindent \z@ \raggedright \normalfont
   \Huge\bfseries #1
   \par\nobreak
   \vskip 20\p@
}}
\makeatother

\usepackage{setspace}
\setstretch{1.3}

% Toto makro definuje kapitolu, která není očíslovaná, ale je uvedena v obsahu.
\def\chapwithtoc#1{
\chapter*{#1}
\addcontentsline{toc}{chapter}{#1}
}


\newfloat{tabulka}{tbph}{lop}
\floatname{tabulka}{Table}


\newfloat{graff}{tbph}{lop}
\floatname{graff}{Figure}



\long\def\centergraf#1 {
{
    \begingroup
    \centering
    \par
    #1
    \par
    \endgroup
}

}

\def\uv#1{``#1''}

\long\def\jednatabulkan#1#2#3#4{
 \begin{tabulka}[tpb]
 %\begin{tabulka}[H]
    
         \centergraf{
         \footnotesize
    \begin{tabular}{ #2 }
        #3
        
    \end{tabular}
             
             }
     
     \caption{#4} 
     \label{tabulka:#1}
 \end{tabulka}
}


\long\def\jednatabulkam#1#2#3#4{
 \begin{tabulka}[h]
     
         \centergraf{
         \scriptsize
    \begin{tabularx}{\textwidth}{ #2 }
        #3
        
    \end{tabularx}
             
             }
     
     \caption{#4} 
     \label{tabulka:#1}
 \end{tabulka}
}

\long\def\jednatabulka#1#2#3#4{
 \begin{tabulka}[tpb]
     
         \centergraf{
         \footnotesize
    \begin{tabularx}{\textwidth}{ #2 }
        #3
        
    \end{tabularx}
             
             }
     
     \caption{#4} 
     \label{tabulka:#1}
 \end{tabulka}
}

\long\def\grafff#1#2#3{
 \begin{graff}[tpb]
         \centergraf{
      \includegraphics[width=#3mm]{./#1.pdf}
       \caption{#2}
      \label{graf:#1}
 }
     \end{graff}
 }



\newenvironment{pitemize}{
\begin{itemize}
  \setlength{\itemsep}{5pt}
  \setlength{\parskip}{0pt}
  \setlength{\parsep}{0pt}
}{\end{itemize}}

\newenvironment{quotee}{
\begin{quote}
``}{''\end{quote}}

\long\def\priklady#1 {

\begin{pitemize}
#1

    \end{pitemize}

}

\usepackage{stringstrings}



\makeatletter
\newcommand*{\stripstartspaces}{%
  \expandafter\@ifnextchar\expandafter X\expandafter{\expandafter
  }\expandafter{\expandafter}%
}
\makeatother


\newcommand{\removelinebreaks}[1]{%
  \begingroup\def\\{}#1\endgroup}

\long\def\priklad#1#2#3 {
 \item\textbf{#1}\par\nobreak#2\nobreak\noindent\par\emph{\stripstartspaces#3}

}

\begin{document}

%% Jednotlivé kapitoly práce jsou pro přehlednost uloženy v samostatných souborech
% Trochu volnější nastavení dělení slov, než je default.
\lefthyphenmin=2
\righthyphenmin=2

%%% Titulní strana práce

\pagestyle{empty}
\begin{center}

\large

Charles University in Prague

\medskip

Faculty of Mathematics and Physics

\vfill

{\bf\Large MASTER THESIS}

\vfill

\centerline{\mbox{\includegraphics[width=60mm]{../img/logo.pdf}}}

\vfill
\vspace{5mm}

{\LARGE Karel Bílek}

\vspace{15mm}

%% Název práce přesně podle zadání
{\LARGE\bfseries A Comparison of Methods of Czech-to-Russian Machine Translation}

\vfill

% Název katedry nebo ústavu, kde byla práce oficiálně zadána
% (dle Organizační struktury MFF UK)
%Ústav formální a aplikované lingvistiky
Institute of Formal and Applied Linguistics

\vfill

\begin{tabular}{rl}

Supervisor of the master thesis: & doc. RNDr. Vladislav Kuboň, Ph.D. \\
\noalign{\vspace{2mm}}
Study programme: & Informatics \\
\noalign{\vspace{2mm}}
Specialization: & Mathematical Linguistics \\
\end{tabular}

\vfill

% Zde doplňte rok
Prague 2014

\end{center}

\newpage

%%% Následuje vevázaný list -- kopie podepsaného "Zadání diplomové práce".
%%% Toto zadání NENÍ součástí elektronické verze práce, nescanovat.

%%% Na tomto místě mohou být napsána případná poděkování (vedoucímu práce,
%%% konzultantovi, tomu, kdo zapůjčil software, literaturu apod.)

\openright

\noindent

Those are all the people that I want to thank:

\begin{itemize}
\item Vladislav Kuboň -- my supervisor, for helping me with the thesis in general, and the related projects
\item Natalia Klyueva -- a college which I worked the most with
\item Ondřej Bojar and Aleš Tamchyna -- for helping me with the Moses system and eman evaluation manager
\item Martin Popel -- for helping me with TectoMT system and gave me lots of ideas in general
\item Rudolf Rosa -- for giving me ideas for better text structuring
\item countless friends that supported me when I just wanted to give up.
\end{itemize}

\newpage

%%% Strana s čestným prohlášením k diplomové práci

\vglue 0pt plus 1fill

\noindent
I declare that I carried out this master thesis independently, and only with the cited
sources, literature and other professional sources.

\medskip\noindent
I understand that my work relates to the rights and obligations under the Act No.
121/2000 Coll., the Copyright Act, as amended, in particular the fact that the Charles
University in Prague has the right to conclude a license agreement on the use of this
work as a school work pursuant to Section 60 paragraph 1 of the Copyright Act.

\vspace{10mm}

\hbox{\hbox to 0.5\hsize{%
In ........ date ............
\hss}\hbox to 0.5\hsize{%
signature of the author
\hss}}

\vspace{20mm}
\newpage

%%% Povinná informační strana diplomové práce

\vbox to 0.5\vsize{
\setlength\parindent{0mm}
\setlength\parskip{5mm}

Název práce:
Porovnáni metod česko-ruského automatického překladu

% přesně dle zadání

Autor:
Karel Bílek

Katedra:  % Případně Ústav:
Ústav formální a aplikované lingvistiky
% dle Organizační struktury MFF UK

Vedoucí diplomové práce:
doc. RNDr. Vladislav Kuboň, Ph.D.,
Ústav formální a aplikované lingvistiky
%Jméno a příjmení s tituly, pracoviště
% dle Organizační struktury MFF UK, případně plný název pracoviště mimo MFF UK

Abstrakt:
V této práci představuji několik metod česko-ruského automatického pře\-kla\-du, včetně jak více historických, tak více moderních systémů, a včetně jak frázovách, tak pravidlových systémů. Nejdříve stručně popisuji lingvistické základy češtiny a ruštiny a jejich společnou historii a rozdíly. Poté popisuji automatizaci, vytváření a zlepšování některých ze systémů automatického překladu, společně s jejich po\-ro\-vná\-ním, s použitím jak automatických metrik, tak omezené lidské anotace. Zároveň s tím také popisuji vytvoření několika korpusů česko-ruských paralelních dat a ruských monolingválních dat.
% abstrakt v rozsahu 80-200 slov; nejedná se však o opis zadání diplomové práce

Klíčová slova:
% 3 až 5 klíčových slov
Čeština, ruština, strojový překlad

\vss}\nobreak\vbox to 0.49\vsize{
\setlength\parindent{0mm}
\setlength\parskip{5mm}

Title:
% přesný překlad názvu práce v angličtině
{A Comparison of Methods of Czech-to-Russian Machine Translation}

Author:
Karel Bílek
%Jméno a příjmení autora

Department:
Institute of Formal and Applied Linguistics
%Ústav formální a aplikované lingvistiky
%Institute of Formal and Applied Linguistics
%Název katedry či ústavu, kde byla práce oficiálně zadána
% dle Organizační struktury MFF UK v angličtině

Supervisor:
doc. RNDr. Vladislav Kuboň, Ph.D.,
Institute of Formal and Applied Linguistics
%Jméno a příjmení s tituly, pracoviště
% dle Organizační struktury MFF UK, případně plný název pracoviště
% mimo MFF UK v angličtině

Abstract:
In this thesis, I am presenting several methods of Czech-to-Russian machine translation, including both historical approaches and more modern ones, and including both phrase-based and rule-based systems. I am first briefly describing the linguistic background of Czech and Russian, and their common history and differences. Then, I am describing automating, building and improving some of the machine translation systems, together with their comparison, using both an automated metric and a limited human annotation. Meanwhile, I am also describing the creation of a several corpora of Czech-Russian parallel data and Russian monolingual data.
%Abstrakt abstrakt.
% abstrakt v rozsahu 80-200 slov v angličtině; nejedná se však o překlad
% zadání diplomové práce

Keywords:
Czech, Russian, Machine translation
% 3 až 5 klíčových slov v angličtině

\vss}

\newpage

%%% Strana s automaticky generovaným obsahem diplomové práce. U matematických
%%% prací je přípustné, aby seznam tabulek a zkratek, existují-li, byl umístěn
%%% na začátku práce, místo na jejím konci.

\openright
\pagestyle{plain}
\setcounter{page}{1}
\tableofcontents



\include{chap0_intro}
\chapter{Brief comparison of Slavic languages}

This section is an overview of both the history of Slavic languages and of some characteristics of those languages in general.

%If a reader is interested in a further study of Slavic languages as a whole, I can only recommend \cite{sussex2011slavic}, which covers both history and
%linguistic characteristics of Slavic languages on approximately 660 pages.
%Most of the factual information in this chapter is from the afforementioned book.

\section{History}

%Unless otherwise noted, information in this chapter are taken from this book.

\subsection{Common history}

\label{ch:common_history}

Using comparative linguistics, Slavic languages in general can be traced back to Indo-European language family, with Proto-Indo-European (PIE) as its reconstructed ancestral language. 

As greatly described in \cite{oxfordintro}, taking Proto-Indo-Europeans as unified people with unified history can be somehow controversial -- we are, as noted there, trying to \uv{put absolute dates on a hypothetical construct}. Still, based on the reconstructed Proto-Indo-European language having, for example, word for wheeled wehicles, we can date Indo-Europeans on Great Eurasian Plain to approximately 4000 BC, as noted by \cite{sussex2011slavic}.

\cite{sussex2011slavic} further notes that our knowledge of this part of European pre-history is \uv{sketchy and partly conjectural}. He further notes:

\begin{quotee}Although we can only guess how far their territory extended, it is possible that at least the European center of the Indo-European homeland -- if
not the original homeland itself (on one widely held view) -- was in what is now Western Ukraine, and that they spoke a fairly homogeneous language.\end{quotee}

\cite{oxfordintro} further describes the reconstruction of this PIE. In here, however, this excerpt from \cite{ringe2008proto} will suffice:

\begin{quotee}Though there continue to be gaps in our knowledge of PIE, an astonishing
proportion of its grammar and vocabulary are se\-cu\-re\-ly re\-con\-struct\-a\-ble by the
comparative method. As might be expected from the way the method works,
the phonology of the lan\-guage is re\-la\-ti\-ve\-ly cer\-tain. Though syntactic reconstruction is in its infancy, PIE syntax is also relatively uncontroversial because
the earliest-attested daughter languages agree so well.\end{quotee}

With similar logic, we can reconstruct an ancestral language to all Slavic languages, call it Proto-Slavic and put it on a place in space and time.

\cite{sussex2011slavic} puts the emergence of Proto-Slavic at around 2000-1500 BC. 


%on the other hand, names the language at the time of Slavic unity \uv{Early Slavic} and \uv{Middle Slavic} and places Proto-Slavic only at 600 AD.

\cite{schenker1993proto} adds, rather poetically:
\begin{quotee}The Slavs were the last Indo-Europeans to appear in the annals of
history. Slavonic texts were not recorded till the middle of the ninth
century and the first definite reference to the Slavs' arrival on the frontiers
of the civilized world dates from the sixth century AD, when the Slavs
struck out upon their conquest of central and south-eastern Europe. Before
that time the Slavs dwelled in the obscurity of their ancestral home, out of
the eye-reach of ancient historians. Their early fates are veiled by the
silence of their neighbours, by their own unrevealing oral tradition and by
the ambiguity of such non-verbal sources of information as archaeology,
anthropology or palaeobotany.\end{quotee}

\cite{kortlandt1982early} breaks this period into several sub-periods, like Balto-Slavic, Middle Slavic and Proto-Slavic; while Balto-Slavic, in his approximation, appears at 2000 BC, Late Proto-Slavic disappears and disintegrate in 900-1200 AD. This division is also slightly mentioned in \cite{schenker1993proto}.

\cite{sussex2011slavic} also sums up the spacial location of Slavic areas before breaking to individual languages:

\begin{quotee}By the fourth century AD the Slav area
stretched from the Oder (Pol \emph{Odra}) River in the west to the Dnieper (Rus \emph{Dnepr},
Ukr \emph{Dnipró}) in the east. In the north they had reached the Masurian Lakes in
central Poland, the Baltic Sea and the Pripet (Pol \emph{Prypeć}; also Eng \emph{Pripyat}, from
Ukr \emph{Prýp’jať}) Marshes. During this period the Slavs would have spoken a fairly
uniform language. Although dialect differences soon began to appear, resulting
\emph{inter alia} in the division into Baltic, Slavic or an intermediate Balto-Slavic, the pace
of linguistic change was relatively slow.\end{quotee}

According to \cite{curta2004slavic}, some form of Slavic was still used as a \emph{lingua franca} in Avar qaganate by about 700 AD, but no further than in 800 AD.

\subsection{Division of the languages}


Continuing its narrative, \cite{sussex2011slavic} notes about breaking of Proto-Slavic:
\begin{quotee}According to general consensus in what is still a controversial area, the real break-up
of Proto-Slavic unity began about the fifth century AD. There seems to have been a
steady expansion to the north and east by the Eastern Slavs. For the others there is
evidence that their migrations were related partly to the disintegration of the Roman
and Hun empires and the ensuing vacuum in Central Europe. 

One group of Slavs
moved westwards, reaching what is now western Poland and the Czech Republic, and
the eastern and north-eastern part of modern Germany. 

A second wave broke away
to the south towards the Balkan Peninsula, where they became the dominant ethnic
group in the seventh century, some (in the east) in turn being conquered by the
Bulgars, a non-Slavic people of Turkic Avar origin.\end{quotee}

\cite{oxfordintro} also puts the point of Slavic break-up at arount 500, while for example \cite{kortlandt1982early} puts it at 900-1200 AD, and \cite{schenker1993proto} puts it at about 900 AD.


While sources disagree on what exactly are all the Slavic languages in which group (partly because question of a language distinction is a political one), all sources\footnote{\cite{sussex2011slavic}, \cite{oxfordintro}, \cite{kortlandt1982early} and others already mentioned} agree on the division to South Slavic languages, East Slavic languages and West Slavic languages; West Slavic being the westwards moving group from the above narrative, South Slavic being the group moving to Balkan and Eastern Slavic being the group moving eastwards.

According to \cite{sussex2011slavic}, the West Slavic language group consists of Czech, Slovak, Polish, Kashubian and Sorbian. As mentioned, some sources divide the languages slightly differently; for example \cite{siewierska1998overview} breaks Sorbian into Upper and Lower Sorbian; other sources take Kashubian as only a variant of Polish.

South Slavic language group consists of Serbo-Croatian, Bulgarian, Slovenian and Macedonian. As before, this list is taken from \cite{sussex2011slavic} -- the question of Serbo-Croatian language unity, for example, is a highly politiced one, thanks to recent military conflicts in the region.

The Eastern Slavic language group consists of Russian, Ukrainian and Belarussian, again according to \cite{sussex2011slavic}.

%The differences between the languages themselves are less syntactical and are more phonological, morphonological and morphological, of course with some lexical changes. 

%In the section \ref{slavic_overview}, I am describing
% For that reason, I am not describing the linguistic differences of the languages any further, since it would be, again, not in a scope for this thesis. Again, I can recommend the book \cite{sussex2011slavic} for anybody interested in the deeper differences.


\section{Slavic languages overview}
\label{slavic_overview}
In this section, I am almost exclusively citing \cite{sussex2011slavic}. While there are books like \cite{comrie2003slavonic}, they describe the languages one-by-one, whereas \cite{sussex2011slavic} compares language properties across all Slavic languages in a concise and clear manner that I haven't been able to find anywhere else.

Therefore, even when the book is not quoted and cited \emph{directly}, the general information in this chapter is heavily informed by it.

\subsection{Morphology}
\label{linguo:morphology}
On the morphological typology, Slavic languages belong to synthetic inflectional languages. They are morphologically rich, with a sophisticated affix system and a little analytical approach to verb morphology (for forms like future tense).

Slavic word is composed of roots (which can be one or several) and affixes (prefixes and suffixes). Prefixes usually modify the word's meaning somehow (for example, \uv{ne-} for negative), while suffixes modify the word's class or one of its grammatical categories.
%are either derivational, inflectional or post-inflectional (appearing in this order).

Suffixes are of several types. The ones appearing first are the derivational suffixes, which can determine the word's class \uv{in familiar processes like abstract and agent nominalization, verbalization, adjectivalization}, as noted in \cite{sussex2011slavic}. Next type of suffix are endings\footnote{Called inflectional suffixes in \cite{sussex2011slavic}} that mark one of several grammatical categories -- like \uv{infinitive, person,
number, tense, case and gender}, as noted  in \cite{sussex2011slavic}.

\jednatabulkan{reflex} { |l |l | l | }
{
         \hline

to wash (someone) & mýt & мыть \\ \hline
to wash (self) & mýt se & мыться \\ \hline


} {Reflectives in Russian vs. Czech} 


In some Slavic languages -- like Russian -- reflexivity is expressed by special suffix, called \emph{post-inflectional suffix} by \cite{sussex2011slavic};
whereas in others -- like Czech -- a separate word is used. See for example Table~\ref{tabulka:reflex} (also compare to the voices in the section~\ref{ch:mediopassive}).

\jednatabulkan{categories} { |l |l | l | }
{
         \hline

\textbf{Verbs} & \textbf{Nominals} & \textbf{Both} \\ \hline
Tense & Case & Person \\ \hline
Aspect & Definiteness & Gender \\ \hline
Mood & Deixis & Number \\ \hline
Voice &  &  \\ \hline


} {Inflectional categories} 

Inflectional categories that exist in  Slavic languages are described in the Table~\ref{tabulka:categories} (definiteness and deixis as inflectional categories are, however, not relevant to either Czech or Russian, but are used as such in Macedonian or Bulgarian).\footnote{The table is, again, from \cite{sussex2011slavic}.}

Morphological process are quite similar across Slavic languages -- however, using either different affixes or using the same affixes, but with slightly shifted properties (diferent categories, etc.).

%Slavic languages are surprisingly uniform in the word formation system, and the operations and grammatical processes are simmilar -- however, the affixes themselves are usually either wholly different, or have vastly different semantic and stylistic properties -- either absolutely, or in combination with other words.

In general, it can be said that languages from the Slavic family are morphologically richer than other languages. The result of this richness is that the number of word types in a given corpus is significantly higher, when compared with more analytical languages (like English). %As you can see in the section TODO, in the same corpus, Czech and Russian use \emph{significantly} more word forms, compared to English.
%That can hurt us 
\subsection{Word order}

Languages like English use word order for marking constituents
in a sentence. Slavic languages, on the other hand, use inflective processes, like agreement\footnote{\cite{sussex2011slavic} lists concord, agreement and government}, for the same type of information -- subject \emph{agrees} with predicate, and so on.

Because those relations are marked by inflexion, it \uv{allows} the Slavic languages to be of freer word order. The \emph{standard} order is SVO -- however, this is possible to change when different emphasis is needed.

In particular, this reffers to the so-called Functional Sentence Perspective. Very simply said, it describes the sentence as consisting of two parts -- Topic and Comment, appearing in this order and being separated by a verb. Topic is the part of sentence, \emph{about which we say some information} -- while Comment is the \emph{new information}. Most importantly, Topic doesn't have to be a grammatical subject of the sentence -- for example, the Czech sentence \uv{Ve městě bydlí strašidla} (\emph{Ghosts live in the city}, literally \emph{In city live ghosts}).

Slavic languages usually (with some exceptions like Bulgarian and Macedonian) don't have particles or any other means of marking definitiveness -- the only mean is the definite information being in the topic of the sentence. In other view, the functional sentence perspective can be viewed as \uv{replacing} the definitive particles, and is usually translated as such in translation to English.

In respect to the translation between English and Slavic languages, word order can be seen as something that's hard to translate correctly. 
With respect to translation between Slavic languages, it could possibly help us, since we don't have to, in theory, change the word order too much.
\subsection{Grammatical categories and their changes}
In this section, I will show several grammatical categories and their change from PIE to Slavic languages.

\subsubsection{Cases}

PIE had at least eight cases -- nominative, accusative, genitive, dative, instrumental, locative, ablative and vocative. \cite{sussex2011slavic} lists these eight cases; \cite{ringe2008proto} also adds allative (which, however, survived only in old Hittite).

In Slavic languages, ablative and genitive were conflated into just genitive.
We can demonstrate this conflation on of two phrases from New Testament.
%To illustrate, I have added a simple comparison of old latin and Czech. 

\jednatabulka{korintskym} { |l|X |X | X | }
{
         \hline
genitive &
in\-vo\-cant no\-men \textbf{Do\-mi\-ni} no\-stri Ie\-su Chris\-ti
&
   who call on the name \textbf{of} our \textbf{Lord} Jesus Christ         
&
všem, kdo na ja\-kém\-ko\-li mí\-stě vzý\-va\-jí jmé\-no na\-še\-ho spo\-leč\-né\-ho \textbf{Pá\-na} Je\-ží\-še Kris\-ta
\\

    \hline
ablative
&
gra\-tia vo\-bis et pax a Deo Pa\-tre no\-stro et \textbf{Do\-mi\-no} Ie\-su Chris\-to
&
Grace and pe\-ace to you \textbf{from} God our Fa\-ther and \textbf{the Lord} Je\-sus Christ
&
Mi\-lost vám a po\-koj od Bo\-ha, na\-še\-ho Ot\-ce, a od \textbf{Pá\-na} Je\-ží\-še Kris\-ta \\
    \hline
} {Demonstrating ablative and genitive conflation on 1 Corinthians} 


Old Latin retained both ablative and genitive. Latin genitive of \emph{dominus} (lord, master) is \emph{dominī}, latin ablative of the same word is  \emph{dominō}. Both these forms were used in the very beginning of 1~Corinthians in Latina Vulgata (latin version of The Bible).

In the Table~\ref{tabulka:korintskym}, you can see the translation to both Czech and English. Both cases are inflated in Czech as genitive \uv{(našeho) pána}.\footnote{Bible sources: \cite{latinavulgata}, \cite{bibleniv}, \cite{bible21}. Note that English and Czech translations are not actually translations from Latin, but from more primary sources, but it will suffice for this simple comparison.}

\subsubsection{Numbers}
\jednatabulkan{kravy} { |l|l |l | l | }
{
         \hline
singular &
one cow
&
ena krava
&
jedna kráva
\\
   \hline
dual (in Slovenian) &
two cows
&
dve kravi
&
dvě krávy
\\
   \hline
plural &
three cows
&
tri krave
&
tři krávy
\\


    \hline
} {Demonstrating duals on Slovenian} 

PIE had three numbers -- singular, plural and dual.\footnote{According to both \cite{sussex2011slavic} and \cite{ringe2008proto}.} 

In most of the Slavic languages (including Czech and Russian), dual disappeared, leaving only traces in the grammar. One of the languages where dual remained in full is Slovenian. To illustrate dual in Slovenian, I have added a comparison of \uv{one cow}, \uv{two cows} and \uv{three cows} in Slovenian and Czech in Table~\ref{tabulka:kravy}.

In Czech, dual was retained in declensions of several words, like \uv{hands}; in Russian, the dual \uv{рукама} survives in some dialects, but is generally incorrect.\footnote{See \cite{offord1996using}, page 18.}

\subsubsection{Genders}
PIE had three genders, masculine, feminine and neutral.\footnote{As noted in both \cite{sussex2011slavic} and \cite{ringe2008proto}.} Slavic languages retained these genders, refining them with added features Personal and Animate.

\subsubsection{Tenses}
According to \cite{sussex2011slavic}, late PIE had six tenses: present, future, aorist, imperfect, perfect and pluperfect. 

The tenses were somehow retained in Slavic languages; however, they are used more analytically and with the help of auxiliary verbs -- for example, \uv{budu zpívat} (I will sing) in Czech, or \uv{буду петь} in Russian.

\subsubsection{Moods}
PIE had four moods: indicative, subjunctive, optative and imperative.\footnote{According to both \cite{sussex2011slavic} and \cite{ringe2008proto}} 

In Slavic languages, imperative was replaced by the optatives, and subjunctive mood became conditional.

\subsubsection{Voices}
\label{ch:mediopassive}
\jednatabulkan{voices} { |l|l |l | l | }
{
         \hline

active & νιπτω & I wash (someone) & myji (někoho) \\ \hline
medium & νίπτομαι & I wash (myself) & myji se \\ \hline
passive & νίπτομαι & I am washed (by somebody) & jsem myt \\ \hline


} {Voices in Classic Greek vs. Czech} 

PIE had an active and a mediopassive voice.\footnote{According to both \cite{sussex2011slavic} (where the mediopassive voice is called \uv{middle voice}) and \cite{ringe2008proto}} 

In Slavic languages, this was refined as active and a passive voice, while reflexives, in a way, took the function of a mediopassive voice.

Since mediopassive voice will probably be unknown in general to the reader, I have added an example of Classic Greek that still retained it, in Table~\ref{tabulka:voices},\footnote{See for example \cite{greek1}, \cite{greek2}} with both English and Czech translation.


\subsubsection{Aspects}
\jednatabulkan{nosim} { |l|l |l |  }
{
         \hline
imperfective determinate &
нести́
&
nést
\\
   \hline
imperfective indeterminate &
носи́ть
&
nosit
\\
   \hline
perfective& 
понести 
&
ponést

\\


    \hline
} {Aspects in Czech and Russian} 

PIE had distinction between two aspects -- eventive and stative; eventive aspect being further divided into perfective and imperfective aspect.\footnote{See \cite{ringe2008proto}, page 24}

In Slavic, the sta\-tive as\-pect is de\-gra\-mma\-tized\footnote{\cite{andersen2013origin}}; ho\-we\-ver, the per\-fec\-tive / im\-per\-fe\-ctive di\-stin\-ction be\-came more important than in other Indo-European languages. \cite{sussex2011slavic} calls the growing distinction \uv{the most important development in Proto-Slavic}.

Imperfective motion verbs were also added determinate/non-determinate distinction.

This determinate / non-determinate and perfective / imperfective distinction is present in both Czech and Russian. For the demonstration on the two languages, see Table~\ref{tabulka:nosim} and note, how hard would be to correctly translate the distinction into English.


\chapter{Translation systems}
Machine translation (MT) is a task that's as old as the computer. When the very first computers were created for the task of encryption and decryption, one of the other areas of interest was translation of natural languages.\footnote{As noted in the introduction in \cite{koehn2010statistical} -- \uv{The history of machine translation goes back over 60 years, almost
immediately after the first computers had been used to break encryption codes in the war, which seemed an apt metaphor for translation: what is
a foreign language but encrypted English?}}

Translation between Czech and Russian has, too, some history. TODO:líp

In this section, I am describing both historical and more recent approaches for machine translation between those two languages. In the next section, I will describe our experiments with those systems.

\section{Statistical vs. rule-based -- an overview}
\label{axis}
MT systems has historically used many different approaches. One way of classifying them is on the axis of rule-based vs. statistical.

In general, we can re-use the definition, used in \cite{bojar}, which is as follows:
\begin{itemize}
\item rule-based MT systems:
\begin{itemize}
\item use analysis, transfer and synthesis steps
\item use formal grammars
\item use hand-made dictionaries
\item have linguistic information hard-coded and therefore aren't lan\-guage-ag\-nos\-tic
\end{itemize}
\item statistical MT systems
\begin{itemize}
    \item use more variants of outputs, rank them with some score, and choose the best one
    \item train internal dictionaries from big parallel data
    \item have more compact translation core, their inner working are less obvious
    \item use statistics instead of linguistic rules and therefore are more language-agnostic
\end{itemize}
\end{itemize}

However, with actual, real-life systems, the distinction is usually not as clear-cut. 
For example, statistical MT systems like Moses (see TODO) can get significantly better results with added linguistic information; 
on the other hand, systems like TectoMT (see TODO), which can for some intents and purposes be classified as more rule-based, have individual parts in some way based on statistics.

\section{Rule-based systems}

%The whole reason for focusing on Czech-to-Russian translation in this thesis, as opposed to the opposite direction, was the history of Cze.

%Unfortunately, I haven't been able to get any of the historical syst

%\section{Unrunnable systems}
%In this section I will present some historical systems, that I haven't been able to get successfully running.

\subsection{RUSLAN}

RUSLAN is a machine-translation system, developed between 1984 and 1988 at several departments of Charles University, Prague. RUSLAN firmly belongs to the \emph{rule-based} category, since at that timeframe, statistical machine translation wasn't even invented yet.




%However, \cite{nguyen2009systemes} describes (unsurprisingly, also in French.) modern-day experiments with Systems Q, also mentioning, that all Fortran implementations has been lost\footnote{Which would make UFAL's version, if it was working, quite unique.} and that he reimplemented it in C. I haven't been able to try this version.



Description of the system can be found in \cite{olivaruslan} or \cite{hajic1987} -- however, the reader has to bear in mind that both the systems \emph{and} their manuals and descriptions are severly dated. (At least for me personally, especially the book \cite{olivaruslan} was hard to read and navigate in.) Contemporary (but not as detailed) description of the system can be found in \cite{recycled}.

The whole RUSLAN system has several components:
\begin{pitemize}
\item preprocessing, written in Pascal
\item morphological analysis, using dictionary, written in Q-Systems (described further) and interpreted in Fortran IV
\item syntactico-semantical analysis, using morphology, also written in Q-Systems; this component uses FGD as its theoretical starting point
\item generation, also using Q-Systems
\item morphological synthesis, using Pascal
\end{pitemize}

\subsubsection{Q-Systems}
Q-Systems (sometimes also Systems Q) -- Q stands for \uv{Quebec} -- are a tool for machine translation, developed at Montreal University by Alain Colmerauer, also the creator of Prolog.\footnote{See \cite{qsystems}.} 

Q-Systems are similarly declarative as Prolog. This means the author can focus more on the \emph{result} than the \emph{order} of analysis. If there is any ambiguity, all the possibilities are explored \emph{in parallel} (this is how Q-Systems differ from, for example, Prolog). 

In theory, this could make writing lexical rules easier and resulting in simplier rules; in reality, the resulting rules are quite unreadable, as will be seen later.

Q-Systems are not very widely used or widely worked with. One of the reasons might be the fact that all documentation is in French.  


\subsubsection{Dictionary}
RUSLAN uses a Czech-to-Russian dictionary, written by hand in afforementioned Q-Systems. Dictionary item looks like this:

\begin{verbatim}
DLOUH==M(RS(+(*INT)),MI2289,DLINNYJ).
DLOUH==M(RS(-(*INT)),MI2276,DOLGIJ).
\end{verbatim}

This describes two possibilites of the translation of the word \uv{dlouhý} to Russian: the first is \uv{длинный} and the second is \uv{долгий}. They differ by the semantic feature INT they require or forbid from the word they depend on. 

More complex dictionary item looks like this:
\begin{verbatim}
C3ES3TIN
  ==Z(@(*A), MIO109, $(JAZYK), 
     2(POS, #($), &, $(MI28), $(C2ES2KIJ),
       1(=,@($), #($),$($))),
     1(=,@($),#($),$($))).
\end{verbatim}

This item translates the word \uv{čeština} to Russian words \uv{чешский язык} and also describes their relationship.

Maybe because memory was more expensive than today, all dictionary items are used without any comments, leaving only very difficult-to-decypher rules.
\subsubsection{Analysis}

The rules for analysis are even less readable. Random example of two such rules are as follows:

\begin{verbatim}
1(B*, X*1, /, X*2, F*1(C*), X*3, /, X*4, @(V*), X*5, %(X*),
 I(*), 1(X*6, $($)), X*7)

1Z(A*9), (Z*2)
  == 1(D*, /, X*2, F*1/X*),/,@(V*),X*5,%(X*),1*,1(X*6,$($)),X*7,
      A*B,
     5(U*1, @(U*2), U*3, $(U*), 3(E*(Y*1), B*(C*), W*1, W*3, %(X*), 
        $(W*)),
     +1Z(A*9, Z*2)
       / -NON- (, + -DANS- X*9 -ET- +(V*) -HORS- X*9, +(VZT)
        -ET- -(V*) -HORS- X*9, *
        -ET- C* = S
        -ET- X*3,* -HORS- /,N(S), S(S), D(S), A(S), L(S), I(S)
        -ET- / -HORS- X*2, 2
        -ET- (, Y%1 = -NUL-
             -OU- E*(Y*1) -HORS- U*2,*
             -OU- E*(+(V*, *)) -HORS- U*2, *
             -OU- -NON- E*(-(V*)) -HORS- U*2, * .)
        -ET- (. E*(Y*1) *N
             -OU- H(B*(C*)) -DANS- U*1 .) .
\end{verbatim}

Those are all left with next to no comments. For example, the only comment for the group of more than 20 rules, including the two rules above, is \texttt{RELATIVE CLAUSES ADJOINED TO THEIR HEADS}.

\begin{comment}
\subsubsection{Dictionary coverage of WebColl}

The dictionary contains about 8,000 lexical items. However, the domain of the translation and, therefore, the dictionary, was manuals for old computers from 1980's. 

In different experiments (\cite{florida}), we tried to measure how many nouns from the RUSLAN dictionary appear at all in a modern text.

For that, we used a monolingual Czech corpus WebColl, consisting of roughly 7~million sentences (114~million tokens)\footnote{See \cite{webcoll}}.

RUSLAN dictionary has 2,783 nouns. In the WebColl corpus, from those nouns, 611 appear less than 10 times -- and from those, 412 don't appear \emph{at all}.

The reverse is similarly infavourable: from 39,434,505 nouns in the corpus, only 11,862,221 is in the dictionary.

\subsubsection{Experiments}

Despite the general un-maintainability of the RUSLAN code and despite the small dictionary, we tried to run the system on our test data.

However, all our experiments ended in some sort of error.

Because I am not able to code in neither Systems-Q nor FORTRAN (in which the Systems-Q interpreter is coded), I gave up on this experiment. (?????)

\end{comment}

\subsection{Česílko 1.0}

\uv{Česílko} is a name for two entirely different machine translation systems with slightly different goals and, more importantly, slightly different structure. Both were originally intended for Czech-to-Slovak translation.

Česílko 1.0 was a system, developed in 2000, and was aimed for direct translation between Czech and Slovak and intended to assist a translation memory\footnote{See \cite{cesilko1}.}. The translation works lemma-by-lemma in a following fashion:
\begin{pitemize}
\item morphological analysis of source (Czech) language
\item disambiguation
\item direct translation, lemma-by-lemma
\item morphological synthesis
\end{pitemize}



The system is written in a mixture of C, C++ and Flex (fast lexical analyser generator). The code itself is not really well documented and modular, but that can be attributed to the age of some of the components -- despite the whole system being developed in 2000, some files seem to be as old as 1991.

This system itself is unfortunately not very extendable from Slovak to Russian (as the target language). Partly because of the design itself, partly because translations from Czech to Russians are not doable only word-by-word basis.
%-- as we can see on the examples in the section \ref{sec:experiments} -- the sentences are not really translatable word-by-word.

%For that reason I decided not to further experiment with Česílko 1.0 for Czech-to-Russian machine translation.

\subsection{Česílko 2.0}
Česílko 2.0 is a different project with similar goals, but using different frameworks and adding more transfer rules\footnote{See \cite{cesilko2}}. 
%However, it has its own shares of problems, that prevented us to use it.

The system works in a following fashion:
\begin{pitemize}
\item \textbf{non-deterministic} morphological analysis of source Czech language
\item translation of lemmas
\item applying transfer rules by changing syntactic tree
\item morphological synthesis
\item ranking of all the generated sentences
\end{pitemize}

Unlike Česílko 1.0, Česílko 2.0 uses a non-deterministic parser and explores all the possi\-bi\-li\-ties in parallel. 

Česílko 2.0 uses more advanced and more clearly defined tranfer rules. This advanced transfer would, in an ideal world, make the system more modular and extensionable for our purposes. 

%However, technical problems -- more described in the section TODO -- prevented us from doing any significant work.

Česílko 2.0 is written in the language Objective-C. Because Objective-C might not be known to the reader, I will describe it a slightly more detailed manner.

%However, the implementation details prevented us from doing any significant work on Česílko 2.0.
%To illuminate why, let me focus a little on the technical details.

\subsubsection{Objective-C}
Objective-C is a very simple and elegant extension of C language, developed by Brad Cox in 1980s by adding Smalltalk features to C\footnote{See \cite{cocoa4}}. 

Objective-C is, in my opinion, very easy to learn and understand, at least compared to C++, its more popular counterpart.

Objective-C is not a proprietary language and is possible to compile with either gcc or Clang/LLVM compilers. However, what is proprietary is its most used standard library, Cocoa.
I will describe it here, since it will be important in further sections.

\subsubsection{Cocoa}
When Steve Jobs left Apple, he made a smaller company called NeXT. Among other things, they produced a proprietary operating system called NeXTSTEP, based on Unix.\footnote{For a more detailed history, see \url{https://developer.apple.com/legacy/library/documentation/Cocoa/Conceptual/CocoaFundamentals/WhatIsCocoa/WhatIsCocoa.html\#//apple\_ref/doc/uid/TP40002974-CH3-SW12}.}

This operating system used Objective-C as its standard language, and proprietary libraries, called OpenStep.\footnote{Despite the name, OpenStep is not open source -- the Open allude to the fact that its API specification was open.}

Several years later, Apple (now merged with NeXT) made its new version of Mac OS, called Mac OS X; this operating system was partially based on NeXTSTEP and also used some of its proprietary libraries, now renamed Cocoa.\footnote{The kernel of Mac OS X is open source, as is its \uv{underlying} operating system called Darwin -- however, this system does not contain Cocoa libraries.}

Cocoa is not the only library for Objective-C, but because Apple is the main investor in Objective-C-based systems, it's a de-facto standard library. Cocoa is nowadays found in every Mac PC, iPhone and iPad and maybe other Apple's products.

\begin{comment}
\subsubsection{GNUstep}
GNUstep is a free re-implementation of OpenStep/Cocoa.\footnote{See \url{http://www.gnustep.org/}}.

Its development started in the NeXTSTEP days; however, it still hasn't met feature parity with Cocoa's OS X.

Aaron Hillegass in 2nd edition of his popular book \emph{Cocoa Programming on Mac OS X} discouraged people from using GNUStep. He redacted this note in later versions of the book, perhaps because of protests from GNUstep developers\footnote{\url{http://www.gnustep.org/resources/BookClarifications.html}}, but in my opinion, his notes are still valid.

GNUstep implementations are very often buggy and not feature-complete with Cocoa and, most unfortunately, unpredictable. This is what hurt us with Česílko 2.0.

\subsubsection{Cocoa and Česílko}
When Petr Homola was writing Česílko 2.0, he decided to use Cocoa and Objective-C for development.

On Mac OS X, this configuration is just fine; however, on Linux, where we wanted to run the MT systems (and where only GNUstep is available), this creates unpredictable results.

In my experiments with Czech-to-Slovak translations, I noticed that on Mac OS X, there are about 5-times more sentences generated, than on Linux -- while the program was compiled from the same sources.

After thorough inspection, I found out the error was in GNUstep implementation of NSDictionary -- Cocoa's implementation of associative array\footnote{\url{https://developer.apple.com/library/mac/documentation/Cocoa/Reference/Foundation/Classes/NSDictionary\_Class/Reference/Reference.html}} -- in some unpredictable cases, NSDictionary returns two different values for two equal NSString keys\footnote{it might have to do something with Unicode; however, NSStrings are supposed to be UTF-8 by default}. As a result, one of the modules returned wrong inflection patterns for a number of words and the morphological analyzer then returned only a fraction of the results.

After a \uv{hacky}, but working workaround for this issue, the system returned same correct results on both OS X and Linux. However, I am not at all confident there aren't more similar issues in GNUstep to further develop the system for Russian;
fixing the issues of the standard frameworks, copying API of a closed-source library, that's normally very rarely used, is way beyond the scope of this thesis.

%when the very basic frameworks themselves are unstable and unreliable, the development ceases to make sense.

Reading the paper \cite{evalquality_cesilko}, that presents Česílko 2.0 with a very low BLEU, I think the same issue plagued the authors of that paper -- it's unprobable the BLEU of the correctly working system would be that low, when in \cite{cesilko2}, the results of Česílko 2.0 were slightly better than of Česílko 1.0.

\section{Black-box systems}
\label{blackbox}
In this chapter, I am describing all the \uv{black-box} systems -- that is, without any access to the source code -- that we successfully tried.
\end{comment}

\subsection{PC Translator}
\label{langsoft}

PC Translator is a commercial translation system from a Czech company LangSoft (\url{http://www.langsoft.cz/translator.htm}). PC Translator works with several language pairs, all with Czech on either source or target side.

Authors of PC Translator don't publish any papers or other literature about the system -- what can we tell about its functionality is gathered only from its promotional website and from the experiments with the software itself.

PC translator seems to be purely rule-based. The system seems to work in following steps:

\begin{pitemize}
\item some (probably rule-based) morphological analysis of the source language
\item translation of the lemmas from source language to target language by searching in a large dictionary
\item some synthesis of morphological information and the translated lemma
\end{pitemize}

The system doesn't seem to do any kind of reordering. It also doesn't seem to do any analysis on a deeper level, like sentence constituents. Some of the phrases in the dictionary are longer than one word, but most of them seem to be one-word only.

One of the advantages of PC Translator is its large dictionary -- however, the dictionary is sometimes choosing very odd and inprobable choices when disambiguating between more possible translations. For example, the English sentence \uv{I like dogs} is translated as \uv{Mám rád kleště}, because the term \uv{dog} can be also translated as \uv{kleště}\footnote{from Collins' Dictionary: \uv{dog -- 5. a mechanical device for gripping or holding, esp one of the axial slots by which gear wheels or shafts are engaged to transmit torque}}. This can be seen as a proof that PC Translator is a purely rule-based system.

According to its marketing materials, PC Translator v14 uses a Czech-Russian dictionary with above 650.000 words.

\begin{comment}

\subsubsection{Experiments}
We found out it's not easy to automate translating with PC Translator. Its GUI is suited for translating by hand, sentence-by-sentence, but not for automated translation of thousands of sentences. Also, by definition, Windows GUI is harder to automate on Linux machine from a script.

However, we were able to work around that, with the help of VMWare Player virtualization software (\url{http://www.vmware.com/cz/products/player}) and Au\-to\-Hot\-key GUI scripting software, that allows us to emulate screen clicking (\url{http://www.autohotkey.com/}). Our workflow therefore is:

\begin{pitemize}
\item on Linux machine, encode the source from UTF-8 to windows-friendly encoding
\item encode the source as HTML code
\item start a virtual machine with PC Translator pre-installed
\item on the start of the virtual machine, run AutoHotkey script from an outer-machine folder (thanks to VMWare shared folders and Windows Startup scripts)
\item via this AutoHotkey script, run PC Translator and click on \uv{translate file} feature 
\item translate the HTML file (also shared in the VMWare shared folder)
\item turn off the virtual machine
\item turn the file back from HTML and Windows encodings back to UTF-8
\end{pitemize}

The HTML part is needed because PC Translator had some problems with translating ordinary text files, plus we can pair the translated sentences better thanks to \texttt{id} parameters in \texttt{div} tags.

We used the newest version of PC Translator available at the time, which is PC Translator v14.

\end{comment}

\section{Statistical systems}

\subsection{Google Translate}
\label{google}
Google Translate is a popular free online translation service by Google, an American web search giant (\url{http://translate.google.com}). 
Although Google is producing many academic papers on machine translation, the whole system is still proprietary and we cannot fully inspect it, as in the case of PC Translator, and we can only state our conjectures.

According to Google's own papers\footnote{for example \cite{och} -- F. J. Och is a head of Google Translate group in Google}, Google Translate uses mostly statistical approach to machine translation.

However, because of its purely statistical approach, it either needs huge amounts of data for every language pair, or it needs to use so-called \uv{pivot languages}\footnote{See for example \cite{koehn2010statistical}} -- in the case of Google Translate, it's usually English; specific English word order and English idioms are then re-translated into the target language and sometimes introduce downright wrong translations.

\subsubsection{API}
Google Translate, apart from being a website, has a paid translation API\footnote{\url{https://developers.google.com/translate/?hl=en}}. The API is a REST-based API which returns the translation in standard JSON; however, it also needs fairly complicated OAuth authentication.

Some unofficial libraries remove this complexity and abstracts it away from the user.
One of them is called prosaically \uv{google-api-translate-java} (\url{https://code.google.com/p/google-api-translate-java/}) and is, not very unexpectedtedly, Java-based.


\begin{comment}
We figured out it's not too expensive for our testing purposes, so we ended up paying for the API.\footnote{The cost is measured per character on the source side. We used about 3 million characters and paid about 60 dollars. This is rather high for any repeated experiments, but not that high for one-time translation.}

To automate Google Translate, we cannot use the website itself, simply because pasting tens of thousands of lines into a browser window usually crashes the browser and is probably against Google Translate's Terms of Use.

There are some workarounds around this, such as \uv{faking} browser environment using some automation tools and/or libraries, but we used more stable option.


%We used an unofficial Java library for Google Translate API, called prosaically \uv{google-api-translate-java} (\url{https://code.google.com/p/google-api-translate-java/}).
We used a Java library for Google Translate API, called prosaically \uv{google-api-translate-java} (\url{http://code.google.com/p/google-api-translate-java}).

The tests were done on 3rd May, 2014.\footnote{I think it's important to note the date of the tests, because the quality of online services might change overtime.}
\end{comment}

\subsection{Bing Translator}
\label{bing}
Another available online translation service is Microsoft Translator / Bing Translator. (In Microsoft's own materials, the system is usually called Bing Translator when referring to the website and Microsoft Translator when referring to the API, however it's not very consistent. I will call the whole system Bing Translator, even when referring to the API that's called just \uv{Microsoft Translator} in the documentation.)

Bing Translator is very similar to Google Translate -- it is an online website with an easy GUI and an additional paid API. Again, the team occasionally publishes some scientific papers, but the system is again  proprietary as a whole.

In non-related experiments, we found out that for some language pairs, Bing Translator does more rule-based-looking post-editation (for example, better verb separation in English-to-German translation). 
However the system as a whole seems statistical, similarly to Google Translate.


\subsubsection{API}
Again, Microsoft offers paid Bing Translator API (confusingly marketed as a \uv{dataset} inside Windows Azure platform).

The API is slightly more complex than Google's API because of the auto-expiring token, but Microsoft itself offers some abstracting code as an example in its documentation\footnote{\url{http://msdn.microsoft.com/en-us/library/hh454950.aspx}} in C\# and PHP.

%The pricing is slightly different in Microsoft Translator than in Google Translate, but in general is slightly cheaper. First 2 million letters are for free, next 2 million are for about 40 US dollars.

\subsection{Yandex Translate}
\label{yandex}
Yandex (\url{http://www.yandex.ru}) is a Russian search portal that, according to its website\footnote{\url{http://company.yandex.com/}}, generates 61 percent of web search traffic in Russia.

Apart from being a search engine, Yandex offers a variety of other services. One of them is Yandex Translate (\url{http://translate.yandex.com})\footnote{Or \url{http://translate.yandex.ru} for Russian version} -- again, a simple website for automatized translations, similar to aforementioned Google Translate or Bing Translator.

%I wanted to include Yandex Translate, because as a Russian service, it could have better Russian language models and better Russian support in general.

\subsubsection{API}
%From the three online services, Yandex API is probably the simpliest to use. It uses a simple JSON interface, which requires an API key.
Yandex Translate also has a translation API. 
The API itself is absolutely free, unlike the other two translation systems, and is probably the easiest of the three online services to implement; however, it has strange and vaguely defined usage limits with no way of checking the actual usage.

%In our experiments, the API simply stopped returning sentences after approximately 1 million characters per 24 hours. After 24 hour period, the API became usable again.

\subsection{Moses}
\label{moses}
Moses is an open-source machine translation toolkit with GPL licence, developed as a successor to a closed-source Pharaoh system.\footnote{See for example \cite{mosespaper} or \cite{moseslink}}

The system is very modular and very customizable, which makes it a bit harder to describe. What makes it also harder is that the term \uv{Moses} is usually applied for both the \uv{core} Moses decoder and the phrase extractor, and the whole toolkit that's bundled with it. I will try to describe it from the point of view that's relevant to our task and write only about the modules that we actually used and about our Moses use-case in general.

%In this section, I will try to describe our Moses set-up; first, I describe the overview of the entire system, and then I further describe some of the elements and our contributions.
%; bear in mind, however, that a completely different set-up is also possible.


\subsubsection{Pipeline overview}
In a very broad view on Moses pipeline, we have some corpus of texts, either parallel or monolingual, and we want to somehow learn a \emph{model} for the translation task. We can then use this model for translating any other sentences in the source language.

This is still a fairly broad definition. For our purposes, let's assume we have a bilingual corpora of a given language pair and a different, usually bigger, monolingual corpora of the target language. We can then learn \emph{translation model} from the bilignual corpora, which is responsible for the \uv{precision} of the translation; and then \emph{language model}, responsible for \uv{fluency}. The actual translation is then \uv{combining} those two factors.

The translation model is called \emph{phrase-based}, because it contains whole phrases, and it contains probabilities of their possible translation, inferred from the corpus. 
Similarly, language model contains probabilities of various word n-grams. 

%At the start, we have a bilingual corpora of a given language pair, and bigger monolingual corpora of the target language.
Now we can look a little closer to what is actually hapenning and what are the actual needed steps.

The bilingual corpora have to be first prepared by aligning the sentences, so every sentence has exactly one translation. (Almost every corpus, available online, is already sentence-aligned.)
%We describe our corpora in the part TODO.

The sentences are then word-aligned, which means pairing words to their translations. We are using MGIZA++\footnote{See \cite{mgiza}}. From this word alignment, Moses learns a so-called \emph{phrase-based translation model}. From the monolingual corpora, we then learn a statistical \emph{language model} -- using, for example, SRILM language model\footnote{See \cite{srilm}}.

Moses is then used for so-called \emph{decoding} of the information from both the language model and the translation model, which choses the best possible translation, using algorithms like beam-search.

However, for the best translation, we need to tune Moses parameters for optimal results. This is done using so-called \emph{minimum rate error rating} -- or MERT for short, which is tuning the parameters on a small separate development set.



After MERT tuning, we finally have working language model, translation model and Moses parameters, which is our complete translation system.

To reiterate, this is our Moses pipeline
\begin{pitemize}
\item getting sentence-aligned parallel corpus, plus bigger monolingual corporus
\item world-alignment on parallel corpus
\item creating phrase-based translation model
\item creating language model
\item tuning the parameters for Moses decoder
\end{pitemize}

%For managing input and output from the various steps, we use \emph{eman} system, which we transformed a little.

\subsubsection{Managing experiments}
The crucial part of Moses is its decoder and phrase extractor. However, we also need some overaching system for managing all the described steps (model training, etc.) -- steps variously fail, don't compile, don't fit in memory, etc. We would also like to reuse partial results in more experiments.

Moses itself has built-in perl-based experiment managment system, called prosaically Experiment Management System (EMS). However, this system is not very widely used on UFAL and we decided to not use it either.

Instead of EMS, we use another perl-based tool called eman (experiment manager). Eman is described well in \cite{eman} or at its website, \url{http://ufal.mff.cuni.cz/eman}.  

Eman breaks down experiment into so-called \uv{steps}. Step encapsulates an atomic part of an experiment and can be in one of a few various states. More importantly, step can be dependent on various other states; if a step fails, all steps dependent on it automatically fail. The whole experiment is then just another step, dependent on all the necessary substeps.

Step is represented by a directory in a playground directory. Step is created by copying a script, called \uv{seed}, from a library of seeds, to a new directory.

%In my opinion, while eman itself is well written, I found the seeds themselves hard to read, too repetitive, and with large amount of code copied and pasted over. 

%For that reason, I tried to rewrite the seeds as perl modules instead of bash scripts for more clarity and reusability. I am, however, not personally sure if my effort in this regard was successful. I decided to use the module \texttt{MooseX::Declare}\footnote{\url{http://search.cpan.org/~ether/MooseX-Declare-0.38/lib/MooseX/Declare.pm}}, which seemed to us like a modern way to write modules in perl. 

%Unfortunately, that module is using very difficult-to-understand perl concepts and source code transformations through \texttt{Devel::Declare}, and as a result, it takes long to run and, perhaps worse, returns very confusing and undecypherable errors. 
%So as a result of my rewrite, I have seeds with code that's probably easier to read and refractor, but on the other hand, it's slow and produces very opaque errors.

%Author of \texttt{MooseX::Declare} is now recommending \texttt{Moops} module instead for declarative syntax; this module is, however, requiring perl version 14 and above, while on UFAL's network, only perl 10 is installed.

\subsubsection{Word alignment}
For word alignment, we are using MGIZA++\footnote{See \cite{mgiza}}, which is a GPL toolkit based on GIZA++\footnote{See \cite{giza}}, which is itself based on models, sometimes called IBM Model 1 to IBM Model 5\footnote{See \cite{ibm}}, which are themselves based on expectation–maximization algorithm (EM).

IBM Models and the underlying EM algorithms are explained perfectly in Chapter~4 of \cite{koehn2010statistical} or in those slides by the same author -- \url{http://www.inf.ed.ac.uk/teaching/courses/mt/lectures/ibm-model1.pdf}.

GIZA++ is an implementation of those models. MGIZA++ is just its mu\-lti-thre\-aded variant, which makes the word alignment slightly faster.

\subsubsection{Phrase-extraction}
In this step, Moses takes the word alignment from the previous step and learns a so-called \uv{phrase table}.
Unlike word alignment, phrase extraction spans multiple words on every side in so-called \uv{phrases}.

Phrase table consists of list of phrases, their probabilities in both ways of translation, and their lexical weighting -- lexical weighting is the probability of the translated phrase counted by individual word pairs. The exact meaning of the numbers is well explained in \cite{koehn2003}.

The phrase-table defines a so-called \uv{translation model}. 

\subsubsection{Language model}
Language model is a part of the system, that tries to model the probability of a target language sentence alone. It's trained on a monolingual corpus.

We use SRILM, which is an open source language modeling toolkit. (Although it's open-source, it uses its own license, that allows free use only for non-commercial and educational purposes.) Current status of SRILM is described in \cite{srilm}, original design is described in \cite{srilm_old}. 

SRILM uses several models, one of them is n-gram word model, described well in \cite{koehn2010statistical}\footnote{chapter 7}. We use n-grams model to the order 3 with words and order 5 with tags (see section TODO). We smooth the models with Kreser-Ney smoothing with Chen and Goodman's modification\footnote{See \cite{chen} and \url{http://www.speech.sri.com/projects/srilm/manpages/ngram-discount.7.html}}. 


\subsubsection{Language model interpolation}
\label{interpol}
If we have more than one monolingual corpora (as we have, as described in TODO), but we are not sure how helpful each of them are, we can use so-called 
%As described in the section XX, we had more than one monolingual Russian corpora, but we weren't sure of how high quality each of them were and how helpful it would be. For this reason, we used so-call 
interpolation (also called mixing).

Linear interpolation in general is described for example in \cite{gutkin}. On a separate heldout data, set of \emph{lambdas} are trained -- the resulting probabilities are then just the individual probabilities, multiplied by the lambdas and summed.

Linear interpolation is supported by Moses by undocumented script in the codebase, called \texttt{interpolate-lm.perl}, which in turn uses SRILM's undocumented AWK script \texttt{compute-best-mix.gawk} and SRILM's \texttt{ngram} with \texttt{-mix-lm} option\footnote{See \url{http://www.speech.sri.com/projects/srilm/manpages/ngram.1.html}}. 
Eman manager then uses these scripts in the \texttt{mixlm} seed.

%We used a linear interpolation instead of log-linear interpolation simply because we didn't notice the option until later in the project.

\subsubsection{Factored translation}

The pipeline, described in the previous sections, translates phrases from the source language to the target language \uv{as is}. Only the exact phrases, found on the source side, can be translated to the exact phrases on the target side; and as they are decoded by Moses, only the phrases themselves are taken into account.

However, with morphologically rich languages such as Russian or Czech, this can result in worse translations because of the number of word forms and resulting data sparsity.
With so-called factored translation, we can add some morphological information while still keeping the main ideas of phrase-based translation. Factored translation was introduced in \cite{factored}.

With factored translation, 
phrased-based approach is extended with 
morphological (or other) information\footnote{Paraphrased from \cite{factored}. The exact quote is \uv{Therefore, we extended the phrase-based approach to statistical translation to tightly integrate
additional information.}}. We can add additional information (for example, lemma or morphological tag) to either side of the translation, on a word level -- this is called a \emph{factor}. Then, instead of training language models and/or translation models on the words alone, we train them on some combination of these factors and then, with the help of Moses that supports factored translation models, combine them together.

\begin{comment}

\subsubsection{Our factored translation experiments}
\label{factors}
In a separate set of experiments only on UMC data (this dataset is described in the section ??), we realized our Moses results have a high OOV rate\footnote{Out Of Vocabulary; how many words were untranslated due to not being found in the phrase table}; this is easily recognizable by Latin script appearing in Czech-to-Russian translation (or Cyrillics in the opposite direction). We then tried to compare several set-ups for factored tranlation to get lower OOV rate and higher BLEU scores.

\grafff{backoff}{Backoff model}{60}

We used a modified version of a set-up described for example in \cite{backoff} as \texttt{lemma backoff}. The set-up is illustrated on Figure~\ref{graf:backoff}, on the left.

The primary translation model is from full word on source side to the full word and morphological tag on target side. The backoff translation model is from lemma on source side to the full word and morphological tag on target side. Then we are using two language models, one for tags and one for words (both separately interpolated, as described in \ref{interpol}).

We were not using interpolated backoff, simply because regular backoff is easier to use with Moses. We were not using models that generate the words from lemma+tag, because we didn't have a working module for Russian morphological generation -- as a result, we can only get the word-forms found on the target side of the parallel training data.

%Primary, we are translating from full word to fill word and morphological tag; only as a backoff, we are translating from 


%However, since we did not have Russian morhpology fully working, we used only the system described as \texttt{lemma backoff} -- with the exception of not translating to lemma. 
%We were not using interpolated backoff, simply because regular backoff is easier to use with Moses.

%The main model translates from a word form on the source side to word form and tag on the target side. The backoff model translates from a lemma (or a stem -- see below) to form and tag on the target side.

For tagging Russian, we used TreeTagger software\footnote{\url{http://www.cis.uni-muenchen.de/~schmid/tools/TreeTagger/}, also see \cite{treetagger1} and \cite{treetagger2}} with a Russian parameter file\footnote{trained on a corpus created by Serge Sharoff, see \url{http://corpus.leeds.ac.uk/mocky/}}. TreeTagger is a closed-source software with a restrictive license, but for free for research purposes.

For Czech, we used tokenizer from UFAL project Treex (described further in section XXX) and for lemmatizing, we used morphological analyzer Morče (\url{http://ufal.mff.cuni.cz/morce/references.php}); however, as described further, in the final system we didn't actually use its output.

With further experimenting, we discovered that using not lemma on the source side, but a \emph{very crude} stem -- just using the first $n$ letters of a word -- gets better results.\footnote{Using stems instead of lemmas is suggested for example in \cite{stemy}. However, their stems are more linguistically motivated, while we just crudely take first few letters. It's actually debatable if our \uv{stems} can be called stems at all.} 
The model is illustrated on Figure~\ref{graf:backoff}, on the right.

\grafff{stem-plot-csru}{Comparison of various set-ups}{100}

The results of our experiment are seen on Figure~\ref{graf:stem-plot-csru} -- \emph{baseline} is original moses with no factors, \emph{1-lemma} and \emph{1-stem} are the \uv{backoff} models without the main model, and \emph{2-stem} and \emph{2-lemma} are the whole models with backoff.
 
We can see that stem with length 6 gets the best results. So, we used stemma with the length 6 in further experiments, such as the WMT submission \cite{mujpaper}.

\end{comment}

\subsubsection{Recasing}
If the language and translation models are all trained on lowercased corpora (like ours are), we need to train a recaser that will convert the translated text from lower case back to upper case.

We could make a rule-based recaser, such as the ones that are included in Moses; however, we can also train a statistical recaser. 
The recaser is basically a complete Moses model, trained as a translation from lower-cased corpus to a cased corpus, where any (case-sensitive) monolingual corpus can serve as a source for the language model -- where source language is the lowercased corpus and the target language is the original corpus. 

\section{Hybrid systems}
\subsection{TectoMT}
\label{tecto}
TectoMT is a different translation system, developed almost exclusively at ÚFAL (\url{http://ufal.mff.cuni.cz/tectomt}; one of the descriptions in \cite{tmt_desc}). While it's partly based on more linguistically motivated theory, it still has many individual parts based on statistical approach -- therefore, I think it's appropriate to put it somewhere in the middle on the axis from the section \ref{axis}. Similarly to Moses in section \ref{moses}, I will try to describe the general structure of the system, but only as relevant to our experiments.

TectoMT is built on the Treex platform, which used to be developed together with TectoMT under the same name, but later split as its own project and is nowadays used for other applications (Depfix, HamleDT). (\url{http://ufal.mff.cuni.cz/treex}). 
Still, because of the long coupled development, Treex source code and its inner structures are based on the needs of TectoMT, and even today it's sometimes difficult to say where exactly the framework ends and application begins. For example, while Treex is downloadable from CPAN perl repository, the version on CPAN is outdated and doesn't work with TectoMT; all TectoMT blocks exist under Treex package; the only way to get newest Treex sources is to install the whole TectoMT framework.

TectoMT is available for a download from UFAL's public SVN repository, with the instructions on UFAL's public wiki (\url{https://wiki.ufal.ms.mff.cuni.cz/external:tectomt:tutorial}). Even more than Moses, TectoMT is an experimental software for academic usage with constant changes from many participants and it takes a while to learn to use it.

Treex and TectoMT are free software. Treex is dual-licensed under Artistic License 1.0 and GPLv2, as most CPAN packages are. TectoMT is licensed under GPLv2 outright. However, there are modules in TectoMT with more restrictive licencing -- some of them can be used only non-commercially -- and some models are trained from non-free sources and probably couldn't be used outside of academia. 

TectoMT package itself also contains various scripts and tools -- one of them are Make scripts for easier running and evaluation of experiments\footnote{What needs to be said is that's it's tailored mostly for UFAL's cluster infrastructure instead of for general usage.}. The goals are different from Moses evaluation managers, described in \ref{moses}, as well as it means, but we can imagine it as being slightly similar. (Those Make scripts don't, unfortunately, have any nice name to refer to them as.)

\subsubsection{Trees and layers}
Ultimatively, TectoMT is based on a linguist theory that predates machine translation by decades.

The Functional Generative Description theory comes from Prague's Linguist Circle (and its older theories), and was described for example in \cite{sgallczech} (in Czech) or \cite{sgallenglish} (in English). It describes a system of various layers of description and the system of their representation and composition, where the layers are (from the \uv{lowest} level) tectogrammatical, phenogrammatical, morphemic, morphophonemic and phonetical\footnote{\cite{sgallenglish}, page 26.}. The concept of tectogrammar was first introduced in \cite{curry}.

Some of the layers would use dependency trees, which are inspired by Czech \emph{sentence analysis} (described for example in \cite{smilauer}).

While Functional Generative Description is a theory, Prague Dependency Treebank project \footnote{\cite{pdt_soft}, the latest description in \cite{pdt_desc} and more detailed in \cite{pdt_manual_a}, \cite{pdt_manual_m} and \cite{pdt_manual_t}} is an application of this theory on an actual treebank. Its data format and software tools are used directly in TectoMT.

PDT uses several layers, with an inspiration from FGD theory. However, instead of the many FGD layers, PDT uses the following ones:
\begin{itemize}
\item w-layer (word layer) for segmented words
\item m-layer (morphological layer), where every word has been transformed into a combination of lemma and tag; but there is still so relation between words,
\item a-layer (analytical layer), where the sentence is tranformed into a dependency tree, where edges represent constituent dependency (or some other relation) and the edge is marked with one of 28 analytical functions (\emph{afun}).\footnote{Technically, the edge is not marked with the function, only the dependent node.}
\item t-layer (tectogrammatical layer), that tries to express semantic structure of a sentence, again with a dependency tree. Nodes on this layer sometimes correspond to nodes on a-layer, but sometimes some artificial nodes are added and, on the other hand, auxiliary words are removed. In addition to \emph{t-lemma} (corresponding with morphological lemma), each node has a \emph{functor}, that tries to somehow convey a semantic function of a relation to node's head (for example, \texttt{AIM} as adjunct expressing purpose). Morphological categories are represented by \emph{grammatemes} (for example, \texttt{number=sg} for singular).
\end{itemize}

TectoMT keeps this distinction into layers. 
The idea of TectoMT is to first convert source sentences through all the layers to t-layer (\emph{analysis}), translating the t-layer to the target language (\emph{transfer}) and converting it back to full sentences (\emph{synthesis}).
%Originally, the plan was to convert source sentence through all the layers to t-layer, transfer the semantics and generate the target sentence.

However, TectoMT modifies the PDT model with the addition of \emph{formemes} (described for example in \cite{zabokrtsky_hab}). Formemes are added to nodes on t-layer, and represent \uv{in which morphosyntactic form the t-node was (in the case of analysis) or will be (in the case of synthesis) expressed in the surface sentence shape}\footnote{\cite{zabokrtsky_hab}}. Theoretically, they should be seen as something \uv{between} the t-layer and the layers above.Example formeme is \texttt{n:since+X} for English expression of time, translatable as \texttt{n:od+2} for Czech (2 for genitive).

Formemes are technically not \uv{correct} according to FGD description and they shouldn't be needed for analysis or synthesis, and we should be able to just transfer the functors. However, the motivation for formemes (at least my understanding of it) is, that \emph{we are not that far}, \uv{pure} semantic translation is not that powerful, and it's better to transfer t-lemmas, formemes and grammatemes, and generate the more surface layers from that.

%Interesting techniques of using Hidden T
\subsubsection{Blocks}



\chapter{Data}
\label{chapter:data}
In this chapter, I am describing the datasets that I used for my experiments.


\section{Parallel data}
\label{corpora:parallel}

\subsection{WMT test sets}
Two of my sets are WMT test data -- WMT 2012 and WMT 2013.

WMT (short for Workshop on Statistical Machine Translation) is, as the name suggests, an annual workshop about statistical machine translation. One of the recurring activities is \emph{shared translation task}, where various teams compete on translation of a shared test data, with a given set of languages. (See for example \cite{wmt_findings_2013}, or the rich history on \url{http://www.statmt.org}.) 


In 2012 and 2013, for all the available languages, one multi-lingual parallel testset was created.
As noted in \cite{wmt_findings_2013}, in 2013, Russian was added as one of the languages. 

In the year 2012, Russian was not one of the languages. However, in the year 2013, WMT released data called \texttt{news-test2012} which \emph{does} retroactively include Russian, additionally to other languages from the year 2012, so I decided to use that, too.

The sentences in the training set are manually translated; for the year 2013, the set is described in \cite{wmt_findings_2013} and available on \url{http://www.statmt.org/wmt13/translation-task.html}. For each of the languages, a fixed number of (different) sentences is taken and then translated to all the other languages.\footnote{In the year 2014, the testset was created slightly differently and I could not extract Czech-Russian pair from it.}


Therefore, most Czech and Russian sentences in this set are not a direct translation of each other, but they are different translations of the same source sentences from various languages -- except for sentences that are originally from either Czech or Russian sources. 

It can be argued, that because the Czech and Russian side are translated separatedly from different languages, the advantage of similarity of the two languages is lost -- different idoms and different word order will be used. 
However, if I used only the directly translated sentences, the data would be significantly smaller. 
%We extracted pairs of Czech and Russian from the 2013 test dataset.

%for the year 2013 (WMT2013) and ??? sentences for the year 2014 (WMT2014). We use WMT2014 as a final test set, and WMT2013 as a development set, for fine-tuning the results.

To reiterate -- I extracted the Czech and Russian sentences from WMT 2013 test set and from WMT 2012 test set.

\subsection{Intercorp}
\label{intercorp_p1}
Intercorp\footnote{Stylized as InterCorp in some materials.} is a parallel corpus for many language pairs, each including Czech. The history and other information is thoroughly described in \cite{intercorp}. One of the language pairs in Intercorp is Czech-Russían.

I am using two separate Intercorp corpora for technical reasons. 

\subsubsection{Mixed Intercorp}
The first Intercorp corpus was used for some of the Moses models (section \ref{experiments:mosesfull}) and was created by my colleague Natalia Klyueva. 
The source data are both from direct Russian-Czech translation and translations from third language, and the sources are not marked clearly.

In this thesis I will call the corpus \uv{Mixed Intercorp}, because all the various sources are mixed together.

\subsubsection{Original Intercorp}

Because I wanted some additional data for testing purposes, I decided to ask for more data from Intercorp. 
I was  given access to \uv{raw} Intercorp data (by Institute of the Czech National Corpus) for non-commercial, academic purposes.

The data itself is organized by source, and each data source is given an information of the original language; even in the Czech-Russian part of Intercorp, there are texts with an English source (for example, Czech and Russian translation of Harry Potter novels). 

The data are in a strange, XML-like format, that's apparently used by a Manatee corpus management system.\footnote{\url{https://www.sketchengine.co.uk/documentation/wiki/SkE/PreparingCorpusOverview}}

\subsubsection{Filtered Intercorp}
\label{corpora:filteredintercorp}
I was able to extract just the data, that are either direct translations from Russian to Czech or vice versa, thanks to the metadata in the corpus.

To have a separate set, I removed the data already present in the \uv{mixed} Intercorp.

The resulting data is purely from fictional novels, except for Jiří Levý's Art of Translation, which is (as the name suggests) a translation theory book.

All of the data are translations from Russian to Czech, except, again, Jiří Levý's Art of Translation.
%Interestingly, this is also the only book that has been translated from Czech to Russian and not the other way.

\jednatabulkan{icorpdata} { |l|l |l | l | }
{
         \hline
\textbf{Author}
&
\textbf{English name}
&
\textbf{Year}
&
\textbf{Sent.}

\\ \hline
Nikolai Ostrovsky &
How the Steel Was Tempered &
1936 &
9844
\\ \hline
Ilya Ilf, Yevgeni Petrov &
The Twelve Chairs &
1928 &
8525

\\ \hline

Mikhail Bulgakov &
The Master and Margarita &
1967 &
7124 
\\ \hline

Nikolai Nikolaevich Nosov &
 The Adventures of Neznaika and His Friends 
&1953-1954 &
3523




\\ \hline

Jiří Levý &
The Art of Translation &
1957 &
3149

\\ \hline

Aleksandr Solzhenitsyn
&
One Day in the Life of Ivan Denisovich
&
1962
&
3090

\\ \hline

Alexander Pushkin &
The Captain's Daughter 
&1863&
2984 
\\ \hline

Aleksandr Solzhenitsyn &
An Incident at Krechetovka Station &
1963&
1467 

\\ \hline
Aleksandr Solzhenitsyn &
Matryona's Place  
&1963
&
880

\\ \hline

} {Filtered Intercorp data} 


All the used novels are in the Table~\ref{tabulka:icorpdata},
sorted by the sentenced count. (English transcriptions, English title translations and years of publication are taken from English Wikipedia.)

%As the reader can probably see, this dataset is markedly different from the first dataset. The data are bigger and are translated directly, on the other hand, the youngest book is from 1967 and the language itself is -- as a prosaic text -- harder to translate in general. Because the two corpora are too different, we did not join the WMT and the Intercorp sets and instead test the systems on each set separately.
\subsection{Subtitles}
\label{corpora:subtitles}
Another set of data that I used were subtitles from OpenSubtitles database.
\subsubsection{FilmTit}

In a separate FilmTit project (\cite{filmtit}), me and my colleagues  tried to make a project for subtitle translation from English to Czech, working simultaneously as a translation memory and a machine translation system.
\subsubsection{OpenSubtitles}

For that project, we were given access to the set of subtitles from the server OpenSubtitles (\url{http://opensubtitles.org}). 

This dataset was, however, not a pair of aligned sentences. It was not even a pair of aligned \emph{files}; we were given just a set of SRT files, and a table which paired each of those files with a movie (identified by IMDB number) -- each movie usually has more subtitle files, and there are usually more errors in the data. 

Subtitle files have the sentences paired with timestamps. (We described the format more thoroughly in \cite{filmtit}.)

From a set of SRT files paired with a movie, we selected just one Czech and just one English SRT file which we found most similar, based on the timestamps.
From the pair of the files, we then extracted the sentences that have the most similar timestamps.
\subsubsection{Tolerance}

These two pairings -- pairings of subtitle files and pairing of the actual lines -- are non-trivial tasks, and require a \uv{tolerance} -- how different can the time marks of a sentence be to be still paired together.

Higher tolerance produces bigger corpus with more errors, while lower tolerance produces smaller, but more correct corpus.

When experimenting on the FilmTit project (as, again, described more thoroughly in \cite{filmtit}), we found out, that the best results (tested both on another movie subtitles and on a different corpus) are -- without exception -- with \emph{bigger corpus} and \emph{higher tolerance}. Even when that introduced a lot of incorrect sentence pairs, the overall results were still better with the biggest possible corpus.
\subsubsection{Czech-Russian subtitles}

I asked OpenSubtitle maintainers, again, for another set of data, this time with Czech and Russian.
Because it contains only movies, that have both Czech and Russian subtitles, the set was much smaller than with English and Czech. I was still able to use the same algorithms from FilmTit to build a parallel corpora, since the raw files had essentially the same format.

I again used the highest possible tolerance, and therefore surely introduced a lot of errors. Unfortunately, for a lack of time, I was't able to replicate the experiments for the ideal tolerance here. However, I hope that the results would be similar than in English-to-Czech translation -- that is, the biggest possible corpus will result in the best translation.

%Due to a technical error\footnote{The raw subtitle file was too big to hold on school servers, and the only copy on a physical disk got corrupted.}, I unfortunately no longer have the original raw data from OpenSubtitles.org, only the extracted pairs. 


\subsection{UMC corpus}
\label{corpora:umc}
UMC (ÚFAL Multilingual Corpus) is a Czech-English-Russian corpus (see \cite{umc}\footnote{This paper talks about UMC 0.1. I haven't been able to find any paper about 0.3, but there is some information on the website -- \url{http://ufal.mff.cuni.cz/legacy/umc/cer/}}). The data were given to UMC creators by Project Syndicate, Prague-based non-profit news organization, translating news and opinions from around the world.

UMC has two versions -- UMC 0.3 and UMC 0.1. UMC 0.3 is then strangely divided into \emph{all}, \emph{test} and \emph{devel} -- however, the parts are strangely mixed together and (only in some files) lowercased.

I decided to use UMC 0.1 corpus as \uv{umc-train}, and from 0.3 I use the test part as \uv{umc-test} and the devel part as \uv{umc-devel}\footnote{However, the texts in the folders test and devel was actually lowercased, so I had to take the data from the all folder.}.


\subsection{Wiki titles}
\label{corpora:wiki}
I also extracted all of the titles from the Czech and Russian Wikipedia, that correspond to each other.\footnote{There is a corpus called \emph{Wiki Headlines} on WMT2013 website, for English and Russian. I am not sure how that got created, but it has nothing to do with my corpus.} 
Wikimedia Foundation (parent organization of Wikipedia) produces complete dumps of Wikipedia in XML; I used one of those dumps and derived pairs of Czech and Russian titles that are translations of each other.\footnote{Unfortunately, at some point in 2013, Wikipedia changed the way interlanguage links work, so my old script no longer works; however, the new way of saving interlanguage links should be even easier to exploit.}

Those are usually only noun phrases and the main word is usually in nominative singular, so the morphology isn't that rich -- however, I hoped that the model will learn some phrases needed for the translation of the named entities.
\section{Monolingual Russian data}
\label{corpora:monolingual}
\subsection{News Crawl}
The largest part of my monolingual data is corpus from WMT workshops that they call \uv{News Crawl}.

According to \cite{wmt_findings_2009}, WMT workshop has been continously crawling web articles since 2007 for making test sets. This allowed them to make a big, randomized corpus from all these sources.

The corpus is categorized by year, and I treat each year as its own corpus for the interpolation (as described in \ref{systems:interpol}).%\ref{interpol}).
\subsection{Common Crawl}
Common Crawl is a publicly available web crawl\footnote{From Wikipedia -- \uv{A Web crawler is an Internet bot that systematically browses the World Wide Web, typically for the purpose of Web indexing. }; web crawl is then a result of such a web crawler} -- \url{http://commoncrawl.org/}.

As described in \cite{commoncrawl}, group of researchers tried to extract parallel data from this web crawl. One of the language pairs was English-Russian and the result is publicly available on WMT site. I used the Russian side of the corpus.

However, the quality of this corpus is very discutable. Because it contains data downloaded from the \uv{raw web}, it often has sentences in different languages, sentences in machine-translated Russian, random UTF-8 symbols, random HTML data, some code, and so on.

This was one of the reasons why I decided to use linear interpolation as discussed in 
\ref{systems:interpol} 
-- hoping, that the tuning algorithm will automatically \uv{find the right balance} between the language models.
\subsection{Yandex}
%Yandex is a Russian portal and search service, that can be described like a Russian mix between Google and Yahoo!. One of its services is a free translation service
Yandex was already described in 
\ref{systems:yandex}. 
Apart from providing free translation API, Yandex also provides an English-Russian parallel corpus (\url{https://translate.yandex.ru/corpus?lang=en}). I used the Russian part of this corpus as a monolingual corpus.

%The interested thing to note is that Yandex parallel corpus is lowercased.
The version of Yandex corpus that I used was originally lowercased. Since then, Yandex already made a new version with the correct cases; I did not use the new version for any experiments for time constrains.

\section{Statistics}
\jednatabulkan{statcs} { |r|r |r | r |r| }
{
         \hline
         \textbf{Corpus} &
\textbf{Lines}
&
\textbf{Tokens}
&
\textbf{\emph{per line}}
&
\textbf{Types}

\\ \hline

UMC dev & 765 & 11,870 & \emph{15.52} & 5,764  \\ \hline 
UMC test & 2,000 & 30,884 & \emph{15.44} & 11,575  \\ \hline 
WMT 2013 & 3,000 & 48,268 & \emph{16.09} & 15,255 \\ \hline 
WMT 2012 & 3,003 & 54,569 & \emph{18.17} & 17,258  \\ \hline 
Wikinames & 114,742 & 244,539 & \emph{2.13} & 91,766  \\ \hline 
InterCorp filtered & 37,586 & 379,432 & \emph{10.1} & 62,479 \\ \hline 
InterCorp mixed & 148,847 & 1,595,531 & \emph{10.72} & 149,052  \\ \hline 
UMC train & 93,395 & 1,741,892 & \emph{18.65} & 111,107  \\ \hline 
Subtitles & 2,324,373 & 11,971,542 & \emph{5.15} & 333,166  \\ \hline 

}{Statistics of Czech side of parallel corpora}

\jednatabulkan{statru} { |r|r |r | r |r| }
{
         \hline
         \textbf{Corpus} &
\textbf{Lines}
&
\textbf{Tokens}
&
\textbf{\emph{per line}}
&
\textbf{Types}

\\ \hline
UMC dev & 765 & 11,936 & \emph{15.6} & 5,622  \\ \hline
UMC test & 2,000 & 31,884 & \emph{15.94} & 11,296  \\ \hline
WMT 2013 & 3,000 & 48,080 & \emph{16.03} & 15,691 \\ \hline
WMT 2012 & 3,003 & 53,499 & \emph{17.82} & 16,473 \\ \hline

Wikinames & 114,742 & 253,128 & \emph{2.21} & 93,168  \\ \hline
InterCorp filtered & 37,586 & 367,838 & \emph{9.79} & 67,666 \\ \hline

InterCorp mixed & 148,847 & 1,508,591 & \emph{10.14} & 144,884  \\ \hline
UMC train & 93,395 & 1,750,475 & \emph{18.74} & 107,756  \\ \hline
Subtitles & 2,324,373 & 11,897,564 & \emph{5.12} & 327,510  \\ \hline

}{Statistics of Russian side of parallel corpora}
\jednatabulkan{statrumono} { |r|r |r | r |r| }
{
         \hline
         \textbf{Corpus} &
\textbf{Lines}
&
\textbf{Tokens}
&
\textbf{\emph{per line}}
&
\textbf{Types}

\\ \hline

News Crawl 2008 & 38,195 & 580,308 & \emph{15.19} & 63,003  \\ \hline
News Crawl 2010 & 47,818 & 643,363 & \emph{13.45} & 70,430  \\ \hline
News Crawl 2009 & 91,119 & 1,315,794 & \emph{14.44} & 98,901  \\ \hline
Common Crawl & 878,386 & 16,837,812 & \emph{19.17} & 665,385  \\ \hline
Yandex & 997,000 & 19,942,195 & \emph{20} & 694,787 \\ \hline
News Crawl 2011 & 9,945,918 & 140,041,123 & \emph{14.08} & 1,569,963  \\ \hline
News Crawl 2012 & 9,789,861 & 140,914,399 & \emph{14.39} & 1,450,003 \\ \hline

}{Statistics of Russian side of monolingual corpora }

In the tables \ref{tabulka:statcs}, \ref{tabulka:statru} and \ref{tabulka:statrumono}, I am presenting some basic statistics about my corpora, sorted by token count.\footnote{The tokenization for the task of counting tokens and types is very rudimentary and just breaks words on every punctuation mark -- in my opinion, it doesn't matter, since the table is only for orientation anyway. Words were converted to lower-case before type counting.}

\section{Unused data}
\subsection{Lib.ru}
\label{corpora:libru}

My colleague Natalia Klyueva downloaded in the year 2012 large amount of fiction books from Russian online library \url{http://lib.ru}.

Unfortunately, I neglected this source and I forgot to include it in any models; I noticed it only at a very late stage and too late for further inclusion in the models described in the section \ref{experiments:mosesfull}.

\chapter{Experiments}
In this chapter, I am describing the experiments with the systems from the chapter TODO.

In general, our goal was to 
\begin{itemize}
\item try to run the historical systems on our data,
\item automate the black-box systems, so we can at least reliably run them from the command line,
\item explore the more open frameworks and try to use them for Czech-to-Russian machine translation and eventually identify possible future work,
\item and finally, compare all the systems on the same set of data.
\end{itemize}

The last goal will be explored in further details in the next section.

%The first goal was largely unsuccessful, as we were not able to run either of the more historical systems. 
%We were able to automate the black-box systems like Google Translate and PC Translator, even when we had to pay for the official API.
%We were successfully able to train a working Moses system, and we were able to improve on an existing TectoMT 
\section{Unsuccessful historical systems}

\subsection{RUSLAN}
\subsubsection{Dictionary coverage of WebColl}

RUSLAN dictionary contains about 8,000 lexical items. The domain of the translation and, therefore, the domain of the dictionary itself, was manuals for old computers from 1980's. We decided to try, how much is this dictionary applicable to a current corpus.

In a set of experiments (\cite{florida}), we tried to measure how many nouns from the RUSLAN dictionary appear at all in a modern text.\footnote{The goal of the experiment described in the paper was actually even broader -- we tried to work on machine learning and assignment of semantic categories. However, the dictionary coverage was a sub-experiment.} 

For that, we used a monolingual Czech corpus WebColl (\cite{webcoll}), consisting of roughly 7~million sentences (114~million tokens).

RUSLAN dictionary has 2,783 nouns. In the WebColl corpus, from those nouns, 611 appear less than 10 times -- and from those, 412 don't appear \emph{at all}.

The reverse is similarly infavourable: from 39,434,505 noun tokens in the corpus, only 11,862,221 are represented in the dictionary.

This means that about two third of nouns would never be translated.

\subsubsection{Experiments}

Despite the general un-maintainability of the RUSLAN code and despite the unfavourable results described in the last section, we tried to run the system on our test data.

However, all our experiments ended in some sort of error.

TODO: vymyslet něco.

Because I am not able to code in neither Systems-Q nor FORTRAN (in which the Systems-Q interpreter is coded), I gave up on this experiment. (?????)

\subsection{Česílko 1.0}

%This system itself is unfortunately not very extendable from Slovak to Russian (as the target language). Partly because of the design itself, partly because translations from Czech to Russians are not doable only word-by-word basis.

As described in the section \ref{cesilko10}, the system is not very extendable for Russian as the target language. Extending the system for Russian would mean significant addition to the code, which is not very mantainable by today's standards. 

Even then, the code in general assumes that the languages are directly translatable word-for-word. As we will see in the PC-Translator results, word-for-word translation from Czech to Russian doesn't give very good results anyway.

\subsection{Česílko 2.0}

When Petr Homola was writing Česílko 2.0, he decided to use Cocoa and Objective-C for development. In the section \ref{cesilko20}, I tried to describe those two.

However, we planned on running Česílko 2.0 on Linux environment. Cocoa is not available on other systems than iOS and Max OS X. For running Česílko 2.0 on Linux, we need another library, called GNUStep -- and there lies our problem.

\subsubsection{GNUstep}
GNUstep is a free re-implementation of OpenStep/Cocoa. (See \url{http://www.gnustep.org/}).

Its development started in the NeXTSTEP days; however, it still hasn't met feature parity with Cocoa's OS X.

Aaron Hillegass in 2nd edition of his popular book \emph{Cocoa Programming on Mac OS X} discouraged people from using GNUStep. He redacted this note in later versions of the book, perhaps because of protests from GNUstep developers\footnote{\url{http://www.gnustep.org/resources/BookClarifications.html}}, but in my opinion, his notes are still valid.

GNUstep implementations are very often buggy and not feature-complete with Cocoa and, most unfortunately, unpredictable. This is what hurt us with Česílko 2.0.

\subsubsection{GNUStep and Česílko}

On Mac OS X, Česílko seems to run fine.
However, on Linux, where we wanted to run the MT systems (and where only GNUstep is available), GNUStep bugs create unpredictable results.

In my experiments with Czech-to-Slovak translations, I noticed that on Mac OS X, there are about 5-times more sentences generated, than on Linux -- while the program was compiled from the same sources.\footnote{Of course linking to Cocoa instead of GNUStep on Mac OS X.}

After thorough inspection, I found out the error was in GNUstep implementation of NSDictionary -- Cocoa's implementation of associative array\footnote{\url{https://developer.apple.com/library/mac/documentation/Cocoa/Reference/Foundation/Classes/NSDictionary\_Class/Reference/Reference.html}} -- in some unpredictable cases, NSDictionary returns two different values for two equal NSString keys. It might have to do something with Unicode; however, NSStrings are supposed to be UTF-8 by default. As a result, one of the modules returned wrong inflection patterns for a number of words and the morphological analyzer then returned only a fraction of the results.

After a \uv{hacky}, but working workaround for this issue, the system returned same correct results on both OS X and Linux. (The hack involved concatenating a space to the original NSString.) 

However, I am not at all confident there aren't more similar issues in GNUstep to further develop the system for Russian.
I believe fixing the issues of frameworks, copying API of a closed-source library, that's normally very rarely used, is way beyond the scope of this thesis.

%when the very basic frameworks themselves are unstable and unreliable, the development ceases to make sense.

Reading the paper \cite{evalquality_cesilko}, that presents Česílko 2.0 with a very low BLEU, I think the same issue plagued the authors of that paper -- it's unprobable the BLEU of the correctly working system would be that low, especially when compared with \cite{cesilko2}, where the results of Česílko 2.0 were slightly better than of Česílko 1.0.


\section{PC Translator}
We found out it's not easy to automate translating with PC Translator, especially when our goal is to be able to run it from a Linux command-line.

Its GUI is suited for translating by hand, sentence-by-sentence, but not for automated translation of thousands of sentences. Also, by definition, Windows GUI is harder to automate on Linux machine from a script.

However, we were able to work around that, with the help of VMWare Player virtualization software (\url{http://www.vmware.com/cz/products/player}) and Au\-to\-Hot\-key GUI scripting software, that allows us to emulate screen clicking (\url{http://www.autohotkey.com/}). Our workflow therefore is:

\begin{pitemize}
\item on Linux machine, encode the source from UTF-8 to windows-friendly encoding
\item encode the source as HTML code
\item start a virtual machine with PC Translator pre-installed
\item on the start of the virtual machine, run AutoHotkey script from an outer-machine folder (thanks to VMWare shared folders and Windows Startup scripts)
\item via this AutoHotkey script, run PC Translator and click on \uv{translate file} feature 
\item translate the HTML file (also shared in the VMWare shared folder)
\item turn off the virtual machine
\item turn the file back from HTML and Windows encodings back to UTF-8
\end{pitemize}

The HTML part is needed because PC Translator had some problems with translating ordinary text files, plus we can pair the translated sentences better thanks to \texttt{id} parameters in \texttt{div} tags.

We used the newest version of PC Translator available at the time, which is PC Translator v14.

\section{Web translators}

\subsection{Google Translate}
\label{gtranslate_ex}

To automate Google Translate, we cannot use the website itself, simply because pasting tens of thousands of lines into a browser window usually crashes the browser and is probably against Google Translate's Terms of Use.

There are some workarounds around this, such as \uv{faking} browser environment using some automation tools and/or libraries, but we used more stable option, which is the paid Google Translate API, as described in the section TODO.


We figured out using the paid API is not too expensive for our testing purposes, so we ended up paying for it, and using it with the library described in TODO.

The cost is measured per character on the source side. We used about 3 million characters and paid about 60 dollars. This is rather high for any repeated experiments, but not that high for a one-time translation.

The tests were done on 3rd May, 2014.\footnote{I think it's important to note the date of the tests, because the quality of online services might change overtime.}
\subsection{Bing Translator}
With similar reasoning as described in \ref{gtranslate_ex}, we decided to use Bing Translator paid API, with the PHP script described in TODO.

The pricing is slightly different in Microsoft Translator than in Google Translate, but in general is slightly cheaper. First 2 million letters are for free, next 2 million are for about 40 US dollars.

\subsection{Yandex Translate}
Yandex Translate API is at the same time easiest and hardest to use from the online APIs.

On one hand, its API is trivial and it's trully free to access, with no charges whatsoever.

On the other hand, the API limits are very vague and majorly slow the experiments down. In our experiments, the API simply stopped returning sentences after approximately 1 million characters per 24 hours. After 24 hour waiting period, the API became usable again.

This actually means the experiments have to be regularly stopped for 24 hour \uv{cool-offs}.

\section{Moses experiments}
We trained the whole Moses model from scratch, using data, described in chapted TODO. I will try to describe our experiments in this section.
\subsection{Alternative eman seeds}

In my opinion, while eman itself is well written, I found the seeds themselves hard to read, too repetitive, and with large amount of code copied and pasted over. 

For that reason, I tried to rewrite the seeds as perl modules instead of bash scripts for more clarity and reusability. I am, however, not personally sure if my effort in this regard was successful. I decided to use the module \texttt{MooseX::Declare}\footnote{\url{http://search.cpan.org/~ether/MooseX-Declare-0.38/lib/MooseX/Declare.pm}}, which seemed to me at that time like a modern way to write modules in perl. 

Unfortunately, that module is using very difficult-to-understand perl concepts and source code transformations through \texttt{Devel::Declare}, and as a result, it takes long to run and, perhaps worse, returns very confusing and undecypherable errors. 
So as a result of my rewrite, I have seeds with code that's probably easier to read and refractor, but on the other hand, it's slow and produces very opaque errors.

Author of \texttt{MooseX::Declare} is now recommending \texttt{Moops} module instead for declarative syntax; this module is, however, requiring perl version 14 and above, while on UFAL's network, only perl 10 is installed.

\subsection{Factored translation experiments}
\label{factors}
We first experimented only on UMC data -- that is, we used dataset UMC-train for training both language and translation model, UMC-dev for MERT tuning and UMC-test for BLEU testing.

At that point, we realized our Moses results have a high OOV rate\footnote{Out Of Vocabulary; how many words were untranslated due to not being found in the phrase table} -- this is easily recognizable by Latin script appearing in Czech-to-Russian translation (or Cyrillics in the opposite direction). We then tried to compare several set-ups for factored tranlation to get lower OOV rate and higher BLEU scores.

\grafff{backoff}{Backoff model}{60}

We used a modified version of a set-up described for example in \cite{backoff} as \texttt{lemma backoff}. The set-up is illustrated on Figure~\ref{graf:backoff}, on the left.

The primary translation model is from full word on source side to the full word and morphological tag on target side. The backoff translation model is from lemma on source side to the full word and morphological tag on target side. So, we are then using two language models, one for tags and one for words (both separately interpolated, as described in \ref{interpol}).

We were not using interpolated backoff, simply because regular backoff is easier to use with Moses. We were not using models that generate the words from lemma+tag, because we didn't have a working module for Russian morphological generation -- as a result, we can only get the word-forms found on the target side of the parallel training data.

%Primary, we are translating from full word to fill word and morphological tag; only as a backoff, we are translating from 


%However, since we did not have Russian morhpology fully working, we used only the system described as \texttt{lemma backoff} -- with the exception of not translating to lemma. 
%We were not using interpolated backoff, simply because regular backoff is easier to use with Moses.

%The main model translates from a word form on the source side to word form and tag on the target side. The backoff model translates from a lemma (or a stem -- see below) to form and tag on the target side.

For tagging Russian, we used TreeTagger software (\url{http://www.cis.uni-muenchen.de/~schmid/tools/TreeTagger/}, also see \cite{treetagger1} and \cite{treetagger2}) with a Russian parameter file\footnote{trained on a corpus created by Serge Sharoff, see \url{http://corpus.leeds.ac.uk/mocky/}}. TreeTagger is a closed-source software with a restrictive license, but for free for research purposes.

For Czech, we used tokenizer from UFAL project TectoMT (described further in section TODO) and for lemmatizing, we used morphological analyzer Morče (\url{http://ufal.mff.cuni.cz/morce/references.php}); however, as described further, in the final system we didn't actually use its output.

With further experimenting, we discovered a surprising thing -- that using not lemma on the source side, but a \emph{very crude} stem -- just using the first $n$ letters of a word -- gets better results.
The model is illustrated on Figure~\ref{graf:backoff}, on the right.

Using stems instead of lemmas is suggested for example in \cite{stemy}. However, their stems are more linguistically motivated, while we just crudely take first few letters.\footnote{It's actually debatable if our \uv{stems} can be called stems at all.}

\grafff{stem-plot-csru}{Comparison of various set-ups}{100}

The results of our experiment are seen on Figure~\ref{graf:stem-plot-csru} -- \emph{baseline} is original moses with no factors, \emph{1-lemma} and \emph{1-stem} are the \uv{backoff} models without the main model, and \emph{2-stem} and \emph{2-lemma} are the whole models with backoff.
 
We can see that stem with length 6 gets the best results. So, we used stemma with the length 6 in further experiments, such as the WMT submission \cite{mujpaper} or the one described in the next section.

\subsection{Full setup}
Our final Moses system uses the following setup:



\chapter{Results}

\section{Overview}
\label{overvieweval}
We are testing all the following systems:
\begin{pitemize}
\item PC Translator \emph{(rule-based, TODO)}
\item Google Translate \emph{(statistical, TODO)}
\item Microsoft Translator \emph{(statistical, TODO)}
\item Yandex Translate \emph{(statistical, TODO)}
\item Moses \emph{(mostly statistical, TODO)}
\item Treex \emph{(hybrid, TODO)}
\end{pitemize}

We are testing on two separate test sets: WMT 2013 and Filtered Intercorp, as described in the section TODO.

\begin{comment}
\subsection{Problem with my evaluation}
My original plan was to do both automated testing \emph{and} do some human judgement on the results, as I do have some basic Russian language skills and I am fully fluent in CZech.

However, I soon found out my level Russian understanding is not high enough for comparing various automated systems. Therefore I will talk about the translation quality only at a very surface level.

At this point, the reader might ask -- why spend time on Czech-to-Russian translation with only passing knowledge of Russian, when Russian-to-Czech translation would be easier for me to evaluate?


I agree that this question has some merit. However, I chose Czech-to-Russian translation pair for several reasons:
\begin{pitemize}
\item existing historical systems like RUSLAN, that I planned to compare with modern systems. Unfortunately, I wasn't able to make any of the systems work.
\item pre-existing Czech-to-Russian TectoMT scenario. I was actually to utilize that and build on it, instead of making the system from scratch, as it would be needed.
\end{pitemize}

%My original plan was that, apart from doing automatized testing, I would be able to look at the results, compare the translation and by comparing the original translation and the translated sentences, I would be able to give some insights on the systems -- as I can personally read Cyrillics, I have some basic understanding of a Russian language and I am a native Czech speaker. However, it turned out that for judging translation errors and classifying them, one needs to know the language on an advanced level -- especially with the phrase-based translation, that makes the sentences from feasibly looking phrases. For this reason, I will abstain for saying too much about the quality of every system.
\end{comment}

\subsection{Metrics}
\subsubsection{Automatic metrics}

We are using several automatic metrics from

First metric we test is a BLEU metric\footnote{See \cite{bleu}.}.

For a comparable BLEU metric (because, as can be seen in the section \ref{samples}, even the reference sentences themselves are inconsistently tokenized), we re-tokenize both reference and the tested system by Moses' built-in tokenizer skript.

We also normalize punctuation, using script from WMT pages. We also replace \texttt{\&quot;}, that appears often in Intercorp, with actual quotation marks.

We decided to use \emph{not} lowercased BLEU -- that means that words \emph{Кристиан} and \emph{кристиан} are two different words.

\subsubsection{Human evaluation}
TODO: moje plány s TrueSkill/Appraise, co nevyšly, protože mi to nikdo neanotoval. Za celou dobu mi tam 4 lidé odanotovali asi 13 vět.

\subsection{Sample sentences}

\subsubsection{Sample sentences - Intercorp}
\label{samples}

For illustration purposes, we have randomly selected 5 sentences from Intercorp and 5 from WMT-2013; the results are shown below. 
On those, I want to demonstrate the conrete results of various machine translation systems. 
%I am also describing some common mistakes with selected sentences, demonstrating those mistakes.

I am presenting all Russian sentences with both original Cyrillic and GOST 7.79 RUS transliteration\footnote{See \cite{gost}}, for the convenience of the reader.
%For every system, I will demonstrate the results on selected sentences from both test sets.


\priklady{


\priklad{
Je mi teprve čtyřiadvacet let a nemohu prožit celý svůj život s legitimací invalidy práce a potloukat se po nemocnicích , když vím , že je to marné~.
}{
Мне всего двадцать четыре года , и я не могу доживать свой век с кни\-же\-чкой инвалида труда , скитаться по лечебницам , зная , что это ни к чему~.
}{
Mne vsego dvadczat` chety're goda , i ya ne mogu dozhivat` svoj vek s knizhechkoj invalida truda , skitat`sya po lechebniczam , znaya , chto e`to ni k chemu~.}

\priklad{
Generál chodil po pokoji sem a tam , kouře svou pěnovku~.
}{
Генерал ходил взад и вперед по комнате , куря свою пенковую трубку~.
}{
General xodil vzad i vpered po komnate , kurya svoyu penkovuyu trubku~.}


\priklad{
" Možná že žádné brilianty neexistují ? " 
}{
- Может быть , никаких брильянтов нет ?  
}{
- Mozhet by't` , nikakix bril`yantov net ?}

\priklad{
Nesměl při tom udělat chybu , vyžadovalo to stejnou přesnost , jako když se zaměřuje dělo . 
}{
Эта работа не допускала описки - так же как прицел орудия . 
}{
E`ta rabota ne dopuskala opiski - tak zhe kak pricel orudiya .}

\priklad{
S hrdostí vzpomněl , jak snadno dobyl kdysi srdce krásné Heleny Baurové .
}{
Он с гордостью вспомнил , как легко покорил когда-то сердце прекрасной Елены Боур .
}{
On s gordost`yu vspomnil , kak legko pokoril kogda-to serdce prekrasnoj Eleny' Bour .}

}

\subsubsection{Sample sentences - WMT}

\priklady{

\priklad{
Thiago Silva, který patří k nejlepším obráncům na světě, taky umožňuje ostatním vedle sebe růst.
}{
Тьяго Силва, который является одним из лучших защитников в мире, также мотивирует всех двигаться вперед.
}{
T`yago Silva, kotory'j yavlyaetsya odnim iz luchshix zashhitnikov v mire, takzhe motiviruet vsex dvigat`sya vpered.}

\priklad{"Dávali mi pět let života a už je to sedm," říká bez emocí na svém lůžku v domě pro paliativní péči Victor-Gadbois v Beloeil, kam přijel předešlý den.
}{
"Мне давали пять лет, я прожил семь", - говорит он, между жизнью и смертью, лежа в кровати в приюте паллиативного у\-хода Виктор-Гадбуа в Белёй, куда прибыл накануне. 
}{
"Mne davali pyat` let, ya prozhil sem`", - govorit on, mezhdu zhizn`yu i smert`yu, lezha v krovati v priyute palliativnogo uxoda Viktor-Gadbua v Belyoj, kuda priby'l nakanune.}


\priklad{
Opatrnost je ovšem na místě například na některých přemostěních, kde může být povrch namrzlý a kluzký. 
}{
Однако внимательность нужна, например, на мостах, где поверхность может быть намерзшая и скользкая. 
}{
Odnako vnimatel`nost` nuzhna, naprimer, na mostax, gde poverxnost` mozhet by't` namerzshaya i skol`zkaya.}

\priklad{
Prostě je ignoruji. 
}{
Я просто не обращаю внимания. 
}{
Ya prosto ne obrashhayu vnimaniya.}


\priklad{
Podle doktorky Christiane Martelové není quebecký zdravotnický systém dostatečně výkonný, aby zajistil přístup všech osob ke kvalitní paliativní péči, než bude možno souhlasit s provedením eutanazie.
}{
По словам доктора Кристиан Мартель, система здравоохранения Квебека недостаточно эффективна, чтобы обеспечить право на паллиативный у\-ход высокого качества до того, как будет разрешен переход к эвтаназии.
}{
Po slovam doktora Kristian Martel`, sistema zdravooxraneniya Kvebeka nedostatochno e`ffektivna, chtoby' obespechit` pravo na palliativny'j uxod vy'sokogo kachestva do togo, kak budet razreshen perexod k e`vtanazii.}

}

\subsection{Baseline}


\jednatabulkan{bleubase} { |r|r|r | }
{
\hline
&
intercorp&
WMT\\ \hline
BLEU & 0.77\% $\pm$ 0.06\%
&
0.83\% $\pm$ 0.16\%

\\ \hline
}{Baseline BLEU}

As a baseline for the translation, I use an automatic transliteration GOST 7.79 RUS, mentioned at \ref{overvieweval}; to make transliteration possible, I also remove Czech diacritics.

Resulting BLEU is very low, as expected.

\section{Sample translations and commentary}
\subsection{PC Translation}
\subsubsection{Demonstration -- Intercorp}
\priklady {


\priklad{ 
Je mi teprve čtyřiadvacet let a nemohu prožit celý svůj život s legitimací invalidy práce a potloukat se po nemocnicích , když vím , že je to marné~. }{ Есть мне только двадцать четыре рейс и не могу пережив весь свой жизнь с удостоверение личности инвалидами труд и болтаться по больницах , когда знаю , что это бесполезные~. 
}{
Est` mne tol`ko dvadczat` chety're rejs i ne mogu perezhiv ves` svoj zhizn` s udostoverenie lichnosti invalidami trud i boltat`sya po bol`niczax , kogda znayu , chto e`to bespolezny'e~.
}


\priklad{ Generál chodil po pokoji sem a tam , kouře svou pěnovku~. }{ Генерал ходил по комнате туда - сюда , дыма свою пенковая трубка~. }{General xodil po komnate tuda - syuda , dy'ma svoyu penkovaya trubka~.}
\priklad{ " Možná že žádné brilianty neexistují ? " }{ " возможно никакие бриллианты отсутствовать ?  }{" vozmozhno nikakie brillianty' otsutstvovat` ? }
\priklad{ Nesměl při tom udělat chybu , vyžadovalo to stejnou přesnost , jako když se zaměřuje dělo~. }{ Обмен причём ошибиться , требовало то такой же аккуратность , как когда специализируется пушка~. }{Obmen prichyom oshibit`sya , trebovalo to takoj zhe akkuratnost` , kak kogda specializiruetsya pushka~.}
\priklad{ S hrdostí vzpomněl , jak snadno dobyl kdysi srdce krásné Heleny Baurové~. }{ С гордостью вспомнил , как ходко захватил когда - то сердце красивое Елена Баурове~. }{S gordost`yu vspomnil , kak xodko zaxvatil kogda - to serdce krasivoe Elena Baurove~.}

}
\subsubsection{Demonstration -- WMT}

\priklady{

\priklad{ Thiago Silva, který patří k nejlepším obráncům na světě, taky umožňuje ostatním vedle sebe růst. }{ Тhиаго Силва, какой принадлежать к лучшим защитник в мире, тоже даёт возможность остальным плечом к плечу рост. }{Thiago Silva, kakoj prinadlezhat` k luchshim zashhitnik v mire, tozhe dayot vozmozhnost` ostal`ny'm plechom k plechu rost.}
\priklad{ "Dávali mi pět let života a už je to sedm," říká bez emocí na svém lůžku v domě pro paliativní péči Victor-Gadbois v Beloeil, kam přijel předešlý den. }{ "давали мне пять лет жизни и уже это семь," говорит минус эмоций в своем гнезде в доме для башка опеку Victor-Гадбоис зажечься Бєлоєил, куда доехал предшествующий день. }{"davali mne pyat` let zhizni i uzhe e`to sem`," govorit minus e`mocij v svoem gnezde v dome dlya bashka opeku Victor-Gadbois zazhech`sya Bєloєil, kuda doexal predshestvuyushhij den`.}
\priklad{ Opatrnost je ovšem na místě například na některých přemostěních, kde může být povrch namrzlý a kluzký. }{ Осторожность есть конечно на месте например на некоторых перекрытие, где может быть покрытие замёрзший и сальный. }{Ostorozhnost` est` konechno na meste naprimer na nekotory'x perekry'tie, gde mozhet by't` pokry'tie zamyorzshij i sal`ny'j.}
\priklad{ Prostě je ignoruji. }{ Запросто есть игнорирую. }{Zaprosto est` ignoriruyu.}
\priklad{ Podle doktorky Christiane Martelové není quebecký zdravotnický systém dostatečně výkonný, aby zajistil přístup všech osob ke kvalitní paliativní péči, než bude možno souhlasit s provedením eutanazie. }{ По врачи Christiane Мартєлове нет qуєбєцкэ медицинский комплекс до\-ста\-то\-чно производительный, чтобы обеспечил подход всех» личностей к ка\-чес\-тве\-нный башка опеку, нежели можно будет согласиться с про\-ве\-де\-нием єутаназиє. }{Po vrachi Christiane Martєlove net quєbєczke` medicinskij kompleks dostatochno proizvoditel`ny'j, chtoby' obespechil podxod vsex» lichnostej k kachestvenny'j bashka opeku, nezheli mozhno budet soglasit`sya s provedeniem єutanaziє.}
}



\subsubsection{BLEU}

\jednatabulkan{bleutrans} { |r|r|r | }
{
\hline
&
intercorp&
WMT\\ \hline
BLEU & 5.02\% $\pm$ 0.01\%
&
6.74\% $\pm$ 0.32\%

\\ \hline
}{PC translator BLEU}

\subsubsection{Discussion}
We can see the results are not ideal. The sentences are evidently translated word-by-word, without any regard for the whole sentence, or the correct word order.

Because the languages are similar, the morphological information is sometimes transferred correctly -- for example, \uv{s hrdostí vzpomněl} -> \uv{с гордостью вспомнил} -- but sometimes, very incorrectly -- for example, genitive \uv{překrásné} is not with the correct ending as \uv{-ой}, but incorrectly as \uv{-ое}.

It's obvious the dictionary is quite big and covering a lot of the words -- most of the words have at least some translation -- but the translation is usually wrong, or used in a wrong context.


\subsection{Google Translate}
\subsubsection{Demonstration -- Intercorp}
\priklady {

\priklad{ Je mi teprve čtyřiadvacet let a nemohu prožit celý svůj život s legitimací invalidy práce a potloukat se po nemocnicích , když vím , že je to marné~. }{ Я всего двадцать четыре года, и я живу свою жизнь с купоном не\-дей\-стви\-те\-ль\-ным работы и болтаться в больнице, зная, что это бесполезно. }{Ya vsego dvadczat` chety're goda, i ya zhivu svoyu zhizn` s kuponom nedejstvitel`ny'm raboty' i boltat`sya v bol`nice, znaya, chto e`to bespolezno.}
\priklad{ Generál chodil po pokoji sem a tam , kouře svou pěnovku~. }{ Генеральный ходил по комнате взад и вперед, его дым пены. }{General`ny'j xodil po komnate vzad i vpered, ego dy'm peny'.}
\priklad{ " Možná že žádné brilianty neexistují ? " }{ "Возможно, не алмазы там?" }{"Vozmozhno, ne almazy' tam?"}
\priklad{ Nesměl při tom udělat chybu , vyžadovalo to stejnou přesnost , jako když se zaměřuje dělo~. }{ Он не мог ошибиться в этом, это требуется такой же точности, как при фокусировке пушки. }{On ne mog oshibit`sya v e`tom, e`to trebuetsya takoj zhe tochnosti, kak pri fo\-ku\-sir\-ov\-ke pushki.}
\priklad{ S hrdostí vzpomněl , jak snadno dobyl kdysi srdce krásné Heleny Baurové~. }{ Мы с гордостью вспомнил, как легко покорил когда-то сердце прекрасной Елены Баур. }{My' s gordost`yu vspomnil, kak legko pokoril kogda-to serdce prekrasnoj Eleny' Baur.}

}
\subsubsection{Demonstration -- WMT}

\priklady{

\priklad{ Thiago Silva, který patří k nejlepším obráncům na světě, taky umožňuje ostatním vedle sebe růst. }{ Тьяго Силва, один из лучших защитников в мире, также позволяет дру\-гой параллельный рост. }{T`yago Silva, odin iz luchshix zashhitnikov v mire, takzhe pozvolyaet drugoj parallel`ny'j rost.}
\priklad{ "Dávali mi pět let života a už je to sedm," říká bez emocí na svém lůžku v domě pro paliativní péči Victor-Gadbois v Beloeil, kam přijel předešlý den. }{ "Они дали мне пять лет жизни, и это семь", говорит он без эмоций на его постели у себя дома для паллиативной помощи Виктор-Гадбуа в Beloeil, куда они прибыли в предыдущий день. }{"Oni dali mne pyat` let zhizni, i e`to sem`", govorit on bez e`mocij na ego posteli u sebya doma dlya palliativnoj pomoshhi Viktor-Gadbua v Beloeil, kuda oni priby'li v predy'dushhij den`.}
\priklad{ Opatrnost je ovšem na místě například na některých přemostěních, kde může být povrch namrzlý a kluzký. }{ Внимание, однако, находится в месте, например, некоторые перемычки, где поверхность может быть ледяной и скользкий. }{Vnimanie, odnako, naxoditsya v meste, naprimer, nekotory'e peremy'chki, gde poverxnost` mozhet by't` ledyanoj i skol`zkij.}
\priklad{ Prostě je ignoruji. }{ Просто игнорируйте их. }{Prosto ignorirujte ix.}
\priklad{ Podle doktorky Christiane Martelové není quebecký zdravotnický systém dostatečně výkonný, aby zajistil přístup všech osob ke kvalitní paliativní péči, než bude možno souhlasit s provedením eutanazie. }{ По словам доктора Кристиана Martel Квебеке система здравоохранения не является достаточно мощным, чтобы обеспечить доступ для всех людей на высококачественной паллиативной помощи, прежде чем он может согласиться проводить эвтаназию. }{Po slovam doktora Kristiana Martel Kvebeke sistema zdravooxraneniya ne yavlyaetsya dostatochno moshhny'm, chtoby' obespechit` dostup dlya vsex lyudej na vy'so\-ko\-ka\-che\-stve\-nnoj palliativnoj pomoshhi, prezhde chem on mozhet soglasit`sya pro\-vo\-dit` e`vtanaziyu.}


}



\subsubsection{BLEU}
\jednatabulkan{bleugoog} { |r|r|r | }
{
\hline
&
intercorp&
WMT\\ \hline
BLEU & 8.79\% $\pm$ 0.15\%
&
17.96\% $\pm$ 0.58\%

\\ \hline
}{Google Translate BLEU}

\subsubsection{Discussion}
As I already noted, it's much harder for me to analyze the phrase-based translations by only looking at them, since they \uv{seem correct}, since they are put together from more correct sentences and are \uv{locally correct}.

However, if we look at the BLEU, we can notice an interesting phenomenon, that will be seen at other systems too, and I am discussing it further below -- the BLEUs of the two systems are significantly different. In fact, BLEU on WMT is two times bigger than on Intercorp.

I am discussing this phenomenon further in the section TODO.

\subsection{Microsoft Translator}
\subsubsection{Demonstration -- Intercorp}


\priklady{

\priklad{ Je mi teprve čtyřiadvacet let a nemohu prožit celý svůj život s legitimací invalidy práce a potloukat se po nemocnicích , když vím , že je to marné~. }{ Мне только двадцать четыре года, и я не могу prožit всю мою жизнь с доказательствами инвалидов на работу и общаться с больницы, когда я знаю, что это бесполезно. }{Mne tol`ko dvadczat` chety're goda, i ya ne mogu prožit vsyu moyu zhizn` s dokazatel`stvami invalidov na rabotu i obshhat`sya s bol`nicy', kogda ya znayu, chto e`to bespolezno.}
\priklad{ Generál chodil po pokoji sem a tam , kouře svou pěnovku~. }{ Генерал ходил вокруг комнаты здесь и там, дым ИТ. }{General xodil vokrug komnaty' zdes` i tam, dy'm IT.}
\priklad{ " Možná že žádné brilianty neexistují ? " }{ «Может есть не алмазы?» }{«Mozhet est` ne almazy'?»}
\priklad{ Nesměl při tom udělat chybu , vyžadovalo to stejnou přesnost , jako když se zaměřuje dělo~. }{ Ему не было разрешено сделать ошибку, он требует такой же точности, как цели пушки. }{Emu ne by'lo razresheno sdelat` oshibku, on trebuet takoj zhe tochnosti, kak celi pushki.}
\priklad{ S hrdostí vzpomněl , jak snadno dobyl kdysi srdce krásné Heleny Baurové~. }{ С гордостью, что он вспомнил, как легко завоевал однажды сердце кра\-си\-вой Елены Baurové. }{S gordost`yu, chto on vspomnil, kak legko zavoeval odnazhdy' serdce krasivoj Eleny' Baurové.}
} 

\subsubsection{Demonstration -- WMT}

\priklady {
\priklad{ Thiago Silva, který patří k nejlepším obráncům na světě, taky umožňuje ostatním vedle sebe růst. }{ Тиаго Силва, который является одним из лучших защитников в мире, тоже, позволяя другим расти рядом друг с другом. }{Tiago Silva, kotory'j yavlyaetsya odnim iz luchshix zashhitnikov v mire, tozhe, pozvolyaya drugim rasti ryadom drug s drugom.}
\priklad{ "Dávali mi pět let života a už je to sedm," říká bez emocí na svém lůžku v domě pro paliativní péči Victor-Gadbois v Beloeil, kam přijel předešlý den. }{ «Они дали мне пять лет жизни, и это семь,» говорит он без эмоций на его кровати в доме для паллиативной помощи, Виктор-Gadbois в Белэиле, где он имел прибыл накануне. }{«Oni dali mne pyat` let zhizni, i e`to sem`,» govorit on bez e`mocij na ego krovati v dome dlya palliativnoj pomoshhi, Viktor-Gadbois v Bele`ile, gde on imel priby'l nakanune.}
\priklad{ Opatrnost je ovšem na místě například na některých přemostěních, kde může být povrch namrzlý a kluzký. }{ Предупреждение является, однако, на месте, например, на некоторых из přemostěních, где она может быть конечно не и скользкой поверхности. }{Preduprezhdenie yavlyaetsya, odnako, na meste, naprimer, na nekotory'x iz přemostěních, gde ona mozhet by't` konechno ne i skol`zkoj poverxnosti.}
\priklad{ Prostě je ignoruji. }{ Просто игнорирует. }{Prosto ignoriruet.}
\priklad{ Podle doktorky Christiane Martelové není quebecký zdravotnický systém dostatečně výkonný, aby zajistil přístup všech osob ke kvalitní paliativní péči, než bude možno souhlasit s provedením eutanazie. }{ По словам д-р Кристиан Martelové не является достаточно мощным, чтобы обеспечить доступ всех лиц к паллиативной помощи высокого качества, чем это будет возможным согласиться выполнять эвтаназии си\-сте\-ма здра\-во\-о\-хра\-не\-ния Квебека. }{Po slovam d-r Kristian Martelové ne yavlyaetsya dostatochno moshhny'm, chtoby' obespechit` dostup vsex licz k palliativnoj pomoshhi vy'sokogo kachestva, chem e`to budet vozmozhny'm soglasit`sya vy'polnyat` e`vtanazii sistema zdravooxraneniya Kvebeka.}
}


\subsubsection{BLEU}

\jednatabulkan{bleubing} { |r|r|r | }
{
\hline
&
intercorp&
WMT\\ \hline
BLEU & 7.22\% $\pm$ 0.13\%
&
15.49\% $\pm$ 0.50\%

\\ \hline
}{Microsoft Translator BLEU}

\subsubsection{Discussion}
Again, I cannot comment on the translation quality so much as I would like to, but we can see the same BLEU disprepancy between the two corpora, as I am further discussing in TODO.

\subsection{Yandex Translate}
\subsubsection{Demonstration -- Intercorp}
\priklady {
\priklad{ Je mi teprve čtyřiadvacet let a nemohu prožit celý svůj život s legitimací invalidy práce a potloukat se po nemocnicích , když vím , že je to marné~. }{ Мне только двадцать четыре лет, и я не могу prožit всю свою жизнь с legitimací инвалидов труда и сутулиться вокруг после больницы , когда я знаю , что это бесполезно~. }{Mne tol`ko dvadczat` chety're let, i ya ne mogu prožit vsyu svoyu zhizn` s legitimací invalidov truda i sutulit`sya vokrug posle bol`nicy' , kogda ya znayu , chto e`to bespolezno~.}
\priklad{ Generál chodil po pokoji sem a tam , kouře svou pěnovku~. }{ Генерал ходил по комнате взад и вперед , дыма свою pěnovku~. }{General xodil po komnate vzad i vpered , dy'ma svoyu pěnovku~.}
\priklad{ " Možná že žádné brilianty neexistují ? " }{ " Возможно, что никаких бриллиантов нет ? " }{" Vozmozhno, chto nikakix brilliantov net ? "}
\priklad{ Nesměl při tom udělat chybu , vyžadovalo to stejnou přesnost , jako když se zaměřuje dělo~. }{ В дальнейшем при этом сделать ошибку , требуется такой же точности , как когда направлена пушка~. }{V dal`nejshem pri e`tom sdelat` oshibku , trebuetsya takoj zhe tochnosti , kak kogda napravlena pushka~.}
\priklad{ S hrdostí vzpomněl , jak snadno dobyl kdysi srdce krásné Heleny Baurové~. }{ С гордостью вспомнил , как легко когда-то завоевал сердце прекрасной Елены Baurové~. }{S gordost`yu vspomnil , kak legko kogda-to zavoeval serdce prekrasnoj Eleny' Baurové~.}
} 
\subsubsection{Demonstration -- WMT}

\priklady {
\priklad{ Thiago Silva, který patří k nejlepším obráncům na světě, taky umožňuje ostatním vedle sebe růst. }{ Тьяго Силва, который относится к лучшим усталых защитников в мире, тоже позволяет другим рядом друг с другом рост. }{T`yago Silva, kotory'j otnositsya k luchshim ustaly'x zashhitnikov v mire, tozhe pozvolyaet drugim ryadom drug s drugom rost.}
\priklad{ "Dávali mi pět let života a už je to sedm," říká bez emocí na svém lůžku v domě pro paliativní péči Victor-Gadbois v Beloeil, kam přijel předešlý den. }{ "Дали мне пять лет жизни, и уже это семь", говорит без эмоций на своей койке в доме для паллиативной помощи Виктор-Gadbois в Beloeil, куда прибыл последний день. }{"Dali mne pyat` let zhizni, i uzhe e`to sem`", govorit bez e`mocij na svoej kojke v dome dlya palliativnoj pomoshhi Viktor-Gadbois v Beloeil, kuda priby'l poslednij den`.}
\priklad{ Opatrnost je ovšem na místě například na některých přemostěních, kde může být povrch namrzlý a kluzký. }{ Осторожность, однако, на месте, например, на некоторых přemostěních, где может быть поверхность namrzlý и скользкий. }{Ostorozhnost`, odnako, na meste, naprimer, na nekotory'x přemostěních, gde mo\-zhet by't` poverxnost` namrzlý i skol`zkij.}
\priklad{ Prostě je ignoruji. }{ Просто игнорирую. }{Prosto ignoriruyu.}
\priklad{ Podle doktorky Christiane Martelové není quebecký zdravotnický systém dostatečně výkonný, aby zajistil přístup všech osob ke kvalitní paliativní péči, než bude možno souhlasit s provedením eutanazie. }{ По doktorky Кристиан Martelové не quebecký системы медицинского до\-ста\-то\-чно мощный, чтобы обеспечить доступ всех людей к качественной па\-лл\-иа\-ти\-вной по\-мо\-щи, чем можно будет согласиться с проведением эв\-та\-на\-зи\-и. }{Po doktorky Kristian Martelové ne quebecký sistemy' medicinskogo dostatochno moshhny'j, chtoby' obespechit` dostup vsex lyudej k kachestvennoj palliativnoj pomoshhi, chem mozhno budet soglasit`sya s provedeniem e`vtanazii.}
}

\subsubsection{BLEU}

\jednatabulkan{bleuyandex} { |r|r|r | }
{
\hline
&
intercorp&
WMT\\ \hline
BLEU & 11.65\% $\pm$ 0.16\%
&
19.54\% $\pm$ 0.54\%

\\ \hline
}{Yandex Translate BLEU}

\subsubsection{Discussion}
Yandex is the most interesting from the three \uv{online translators}.

Originally, we didn't plan on including it in the experimens, but we added it in the end \uv{just in case} -- and it ended up being the best translation system from the three.

And it's interesting that it ends up being the best system \emph{despite} a lot of Czech words being untranslated and ending up as-is on the Russian side.

The system is closed-source, so we can't tell what are the exact reasons for this system being the best; however, it's obvious that Yandex is a Russian company and, therefore, probably have the best available Russian language models and morphological tools.

It also seems like the translation is \emph{not} using English as a pivot language -- there doesn't seem to be any anglicism in the text. 
\subsection{Moses}
\subsubsection{Demonstration -- Intercorp}
\priklady {
\priklad{ Je mi teprve čtyřiadvacet let a nemohu prožit celý svůj život s legitimací invalidy práce a potloukat se po nemocnicích , když vím , že je to marné~. }{ Мне только 24 лет , и я не могу метаниях всю свою жизнь с легитимностью инвалидов работы и слоняться по госпиталям , когда я знаю , что это бесполезно~.  }{Mne tol`ko 24 let , i ya ne mogu metaniyax vsyu svoyu zhizn` s legitimnost`yu invalidov raboty' i slonyat`sya po gospitalyam , kogda ya znayu , chto e`to bespolezno~. }
\priklad{ Generál chodil po pokoji sem a tam , kouře svou pěnovku~. }{ Генерал ходил по комнате , свою pěnovku дыма~.  }{General xodil po komnate , svoyu pěnovku dy'ma~. }
\priklad{ " Možná že žádné brilianty neexistují ? " }{ - Возможно , что не существует никаких бриллиантами ? "  }{- Vozmozhno , chto ne sushhestvuet nikakix brilliantami ? " }
\priklad{ Nesměl při tom udělat chybu , vyžadovalo to stejnou přesnost , jako když se zaměřuje dělo~. }{ При этом нельзя было сделать ошибку , это же точности , как , например , когда происходило уничтожение~.  }{Pri e`tom nel`zya by'lo sdelat` oshibku , e`to zhe tochnosti , kak , naprimer , kogda proisxodilo unichtozhenie~. }
\priklad{ S hrdostí vzpomněl , jak snadno dobyl kdysi srdce krásné Heleny Baurové~. }{ С гордостью вспомнил , как легко он когда-то сердце прекрасной Елены baurové~.  }{S gordost`yu vspomnil , kak legko on kogda-to serdce prekrasnoj Eleny' baurové~. }
} 


\subsubsection{Demonstration -- WMT}

\priklady {
\priklad{ Thiago Silva, který patří k nejlepším obráncům na světě, taky umožňuje ostatním vedle sebe růst. }{ Thiago Сильва , который является лучшим защитником в мире , и позволяет другим рядом экономический рост~.  }{Thiago Sil`va , kotory'j yavlyaetsya luchshim zashhitnikom v mire , i pozvolyaet drugim ryadom e`konomicheskij rost~. }
\priklad{ "Dávali mi pět let života a už je to sedm," říká bez emocí na svém lůžku v domě pro paliativní péči Victor-Gadbois v Beloeil, kam přijel předešlý den. }{ " Мне давали пять лет , и это уже семь , - говорит без эмоций в своей постели в доме для обслуживания полумерой Виктор beloeil , куда он приехал в предыдущий день~.  }{" Mne davali pyat` let , i e`to uzhe sem` , - govorit bez e`mocij v svoej posteli v dome dlya obsluzhivaniya polumeroj Viktor beloeil , kuda on priexal v predy'dushhij den`~. }
\priklad{ Opatrnost je ovšem na místě například na některých přemostěních, kde může být povrch namrzlý a kluzký. }{ Осторожность , однако , на месте , например , на некоторых обводку , где может быть поверхность полосе и скользким~.  }{Ostorozhnost` , odnako , na meste , naprimer , na nekotory'x obvodku , gde mozhet by't` poverxnost` polose i skol`zkim~. }
\priklad{ Prostě je ignoruji. }{ просто игнорируют~.  }{prosto ignoriruyut~. }
\priklad{ Podle doktorky Christiane Martelové není quebecký zdravotnický systém dostatečně výkonný, aby zajistil přístup všech osob ke kvalitní paliativní péči, než bude možno souhlasit s provedením eutanazie. }{ По словам доктора Кристиан Мартелл не Квебек служба здравоохранения достаточно полезным , чтобы обеспечить доступ всех людей полумерой качества обслуживания , чем можно будет согласиться с сделали эвтаназия~.  }{Po slovam doktora Kristian Martell ne Kvebek sluzhba zdravooxraneniya dostatochno polezny'm , chtoby' obespechit` dostup vsex lyudej polumeroj kachestva obsluzhivaniya , chem mozhno budet soglasit`sya s sdelali e`vtanaziya~. }
}


\subsubsection{BLEU}

\jednatabulkan{bleumoses} { |r|r|r | }
{
\hline
&
intercorp&
WMT\\ \hline
BLEU & 11.62\% $\pm$ 0.15\%
&
17.45\% $\pm$ 0.51\%

\\ \hline
}{Moses BLEU}







\chapter*{Conclusion}
\addcontentsline{toc}{chapter}{Conclusion}

I have automated, built, improved, demonstrated and compared (both by human annotators and by automated metrics) several translation systems, both phrase-based and rule-based, between Czech and Russian.

From the systems I have tried, phrase-based translation systems are simply easier to build and give better results.

TectoMT as a more hybrid system shows promise, but with this language pair, the work is only starting; however, it is telling, that it's probably easier to build a new system based on Moses that reaches about the same translation quality as \uv{state-of-the-art} systems, than it would be with TectoMT -- and impossible with purely rule-based systems.

%However, this result can't be taken as final, because the work on the 

\subsubsection{Future work}
The first future work, as already mentioned in \ref{future:morpho}, should probably be a better Russian parser and a better Russian morphology. This would allow us to experiment more with post-editing; we could also use it in factored translation models in Moses; and of course it would allow us to build better models with TectoMT.


%%% Seznam použité literatury
\include{bibliography}

%%% Tabulky v diplomové práci, existují-li.
%\chapwithtoc{List of Tables}

%%% Použité zkratky v diplomové práci, existují-li, včetně jejich vysvětlení.
%\chapwithtoc{List of Abbreviations}

%%% Přílohy k diplomové práci, existují-li (různé dodatky jako výpisy programů,
%%% diagramy apod.). Každá příloha musí být alespoň jednou odkazována z vlastního
%%% textu práce. Přílohy se číslují.


\chapwithtoc{Attachments}

%\setcounter{section}{0}
%\setcounter{chapter}{0}

%this is hack but I don't know how to do it right so it doesn't screw up the PDF contents
\renewcommand{\thechapter}{A}


\chapter{Sample of experiment results}
\label{sampleofresults}

In this attachment, I am demonstrating the results of the six MT systems on ten randomly selected sentences.

I am presenting all Russian sentences with both original Cyrillic and GOST 7.79 RUS transliteration (\cite{gost}), for the convenience of the reader.

\section{Sample sentences}
I have randomly selected 5 sentences from Intercorp and 5 from WMT-2013. 

\subsection{Intercorp}


\priklady{

\priklad{
Je mi teprve čtyřiadvacet let a nemohu prožit celý svůj život s legitimací invalidy práce a potloukat se po nemocnicích , když vím , že je to marné~.
}{
Мне всего двадцать четыре года , и я не могу доживать свой век с кни\-же\-чкой инвалида труда , скитаться по лечебницам , зная , что это ни к чему~.
}{
Mne vsego dvadczat` chety're goda , i ya ne mogu dozhivat` svoj vek s knizhechkoj invalida truda , skitat`sya po lechebniczam , znaya , chto e`to ni k chemu~.}

\priklad{
Generál chodil po pokoji sem a tam , kouře svou pěnovku~.
}{
Генерал ходил взад и вперед по комнате , куря свою пенковую трубку~.
}{
General xodil vzad i vpered po komnate , kurya svoyu penkovuyu trubku~.}


\priklad{
" Možná že žádné brilianty neexistují ? " 
}{
- Может быть , никаких брильянтов нет ?  
}{
- Mozhet by't` , nikakix bril`yantov net ?}

\priklad{
Nesměl při tom udělat chybu , vyžadovalo to stejnou přesnost , jako když se zaměřuje dělo . 
}{
Эта работа не допускала описки - так же как прицел орудия . 
}{
E`ta rabota ne dopuskala opiski - tak zhe kak pricel orudiya .}

\priklad{
S hrdostí vzpomněl , jak snadno dobyl kdysi srdce krásné Heleny Baurové .
}{
Он с гордостью вспомнил , как легко покорил когда-то сердце прекрасной Е\-ле\-ны Боур .
}{
On s gordost`yu vspomnil , kak legko pokoril kogda-to serdce prekrasnoj Eleny' Bour .}

}

\subsection{WMT}

\priklady{

\priklad{
Thiago Silva, který patří k nejlepším obráncům na světě, taky umožňuje ostatním vedle sebe růst.
}{
Тьяго Силва, который является одним из лучших защитников в мире, также мотивирует всех двигаться вперед.
}{
T`yago Silva, kotory'j yavlyaetsya odnim iz luchshix zashhitnikov v mire, takzhe motiviruet vsex dvigat`sya vpered.}

\priklad{"Dávali mi pět let života a už je to sedm," říká bez emocí na svém lůžku v domě pro paliativní péči Victor-Gadbois v Beloeil, kam přijel předešlý den.
}{
"Мне давали пять лет, я прожил семь", - говорит он, между жизнью и смертью, лежа в кровати в приюте паллиативного у\-хода Виктор-Гадбуа в Белёй, куда при\-был накануне. 
}{
"Mne davali pyat` let, ya prozhil sem`", - govorit on, mezhdu zhizn`yu i smert`yu, lezha v krovati v priyute palliativnogo uxoda Viktor-Gadbua v Belyoj, kuda priby'l nakanune.}


\priklad{
Opatrnost je ovšem na místě například na některých přemostěních, kde může být povrch namrzlý a kluzký. 
}{
Однако внимательность нужна, например, на мостах, где поверхность может быть на\-мер\-зша\-я и скользкая. 
}{
Odnako vnimatel`nost` nuzhna, naprimer, na mostax, gde poverxnost` mozhet by't` na\-mer\-zsha\-ya i skol`zkaya.}

\priklad{
Prostě je ignoruji. 
}{
Я просто не обращаю внимания. 
}{
Ya prosto ne obrashhayu vnimaniya.}


\priklad{
Podle doktorky Christiane Martelové není quebecký zdravotnický systém dostatečně výkonný, aby zajistil přístup všech osob ke kvalitní paliativní péči, než bude možno souhlasit s provedením eutanazie.
}{
По словам доктора Кристиан Мартель, система здравоохранения Квебека недостаточно эффективна, чтобы обеспечить право на паллиативный у\-ход высокого качества до того, как будет разрешен переход к эвтаназии.
}{
Po slovam doktora Kristian Martel`, sistema zdravooxraneniya Kvebeka nedostatochno e`ffektivna, chtoby' obespechit` pravo na palliativny'j uxod vy'sokogo kachestva do togo, kak budet razreshen perexod k e`vtanazii.}

}

\section{PC Translator}
\subsection{Intercorp}
\priklady {


\priklad{ 
Je mi teprve čtyřiadvacet let a nemohu prožit celý svůj život s legitimací invalidy práce a potloukat se po nemocnicích , když vím , že je to marné~. }{ Есть мне только двадцать четыре рейс и не могу пережив весь свой жизнь с у\-до\-сто\-ве\-ре\-ние личности инвалидами труд и болтаться по больницах , когда знаю , что это бесполезные~. 
}{
Est` mne tol`ko dvadczat` chety're rejs i ne mogu perezhiv ves` svoj zhizn` s udostoverenie lichnosti invalidami trud i boltat`sya po bol`niczax , kogda znayu , chto e`to bespolezny'e~.
}


\priklad{ Generál chodil po pokoji sem a tam , kouře svou pěnovku~. }{ Генерал ходил по комнате туда - сюда , дыма свою пенковая трубка~. }{General xodil po komnate tuda - syuda , dy'ma svoyu penkovaya trubka~.}
\priklad{ " Možná že žádné brilianty neexistují ? " }{ " возможно никакие бриллианты отсутствовать ?  }{" vozmozhno nikakie brillianty' otsutstvovat` ? }
\priklad{ Nesměl při tom udělat chybu , vyžadovalo to stejnou přesnost , jako když se zaměřuje dělo~. }{ Обмен причём ошибиться , требовало то такой же аккуратность , как когда спе\-ци\-а\-ли\-зи\-ру\-е\-тся пушка~. }{Obmen prichyom oshibit`sya , trebovalo to takoj zhe akkuratnost` , kak kogda specializiruetsya pushka~.}
\priklad{ S hrdostí vzpomněl , jak snadno dobyl kdysi srdce krásné Heleny Baurové~. }{ С гордостью вспомнил , как ходко захватил когда - то сердце красивое Елена Баурове~. }{S gordost`yu vspomnil , kak xodko zaxvatil kogda - to serdce krasivoe Elena Baurove~.}

}
\subsection{WMT}

\priklady{

\priklad{ Thiago Silva, který patří k nejlepším obráncům na světě, taky umožňuje ostatním vedle sebe růst. }{ Тhиаго Силва, какой принадлежать к лучшим защитник в мире, тоже даёт возможность остальным плечом к плечу рост. }{Thiago Silva, kakoj prinadlezhat` k luchshim zashhitnik v mire, tozhe dayot vozmozhnost` ostal`ny'm plechom k plechu rost.}
\priklad{ "Dávali mi pět let života a už je to sedm," říká bez emocí na svém lůžku v domě pro paliativní péči Victor-Gadbois v Beloeil, kam přijel předešlý den. }{ "давали мне пять лет жизни и уже это семь," говорит минус эмоций в своем гнезде в доме для башка опеку Victor-Гадбоис зажечься Бєлоєил, куда доехал предшествующий день. }{"davali mne pyat` let zhizni i uzhe e`to sem`," govorit minus e`mocij v svoem gnezde v dome dlya bashka opeku Victor-Gadbois zazhech`sya Bєloєil, kuda doexal predshestvuyushhij den`.}
\priklad{ Opatrnost je ovšem na místě například na některých přemostěních, kde může být povrch namrzlý a kluzký. }{ Осторожность есть конечно на месте например на некоторых перекрытие, где может быть покрытие замёрзший и сальный. }{Ostorozhnost` est` konechno na meste naprimer na nekotory'x perekry'tie, gde mozhet by't` po\-kry'\-ti\-e zamyorzshij i sal`ny'j.}
\priklad{ Prostě je ignoruji. }{ Запросто есть игнорирую. }{Zaprosto est` ignoriruyu.}
\priklad{ Podle doktorky Christiane Martelové není quebecký zdravotnický systém dostatečně výkonný, aby zajistil přístup všech osob ke kvalitní paliativní péči, než bude možno souhlasit s provedením eutanazie. }{ По врачи Christiane Мартєлове нет qуєбєцкэ медицинский комплекс до\-ста\-то\-чно про\-и\-зво\-ди\-тель\-ный, чтобы обеспечил подход всех» личностей к ка\-чес\-тве\-нный башка опеку, нежели можно будет согласиться с про\-ве\-де\-нием єутаназиє. }{Po vrachi Christiane Martєlove net quєbєczke` medicinskij kompleks dostatochno proizvoditel`ny'j, chtoby' obespechil podxod vsex» lichnostej k kachestvenny'j bashka opeku, nezheli mozhno budet soglasit`sya s provedeniem єutanaziє.}
}




\section{Google Translate}
\subsection{Intercorp}
\priklady {

\priklad{ Je mi teprve čtyřiadvacet let a nemohu prožit celý svůj život s legitimací invalidy práce a potloukat se po nemocnicích , když vím , že je to marné~. }{ Я всего двадцать четыре года, и я живу свою жизнь с купоном не\-дей\-стви\-те\-ль\-ным работы и болтаться в больнице, зная, что это бесполезно. }{Ya vsego dvadczat` chety're goda, i ya zhivu svoyu zhizn` s kuponom nedejstvitel`ny'm raboty' i boltat`sya v bol`nice, znaya, chto e`to bespolezno.}
\priklad{ Generál chodil po pokoji sem a tam , kouře svou pěnovku~. }{ Генеральный ходил по комнате взад и вперед, его дым пены. }{General`ny'j xodil po komnate vzad i vpered, ego dy'm peny'.}
\priklad{ " Možná že žádné brilianty neexistují ? " }{ "Возможно, не алмазы там?" }{"Vozmozhno, ne almazy' tam?"}
\priklad{ Nesměl při tom udělat chybu , vyžadovalo to stejnou přesnost , jako když se zaměřuje dělo~. }{ Он не мог ошибиться в этом, это требуется такой же точности, как при фо\-ку\-си\-ро\-вке пушки. }{On ne mog oshibit`sya v e`tom, e`to trebuetsya takoj zhe tochnosti, kak pri fo\-ku\-sir\-ov\-ke pushki.}
\priklad{ S hrdostí vzpomněl , jak snadno dobyl kdysi srdce krásné Heleny Baurové~. }{ Мы с гордостью вспомнил, как легко покорил когда-то сердце прекрасной Елены Баур. }{My' s gordost`yu vspomnil, kak legko pokoril kogda-to serdce prekrasnoj Eleny' Baur.}

}
\subsection{WMT}

\priklady{

\priklad{ Thiago Silva, který patří k nejlepším obráncům na světě, taky umožňuje ostatním vedle sebe růst. }{ Тьяго Силва, один из лучших защитников в мире, также по\-зво\-ляет дру\-гой па\-ра\-ллель\-ный рост. }{T`yago Silva, odin iz luchshix zashhitnikov v mire, takzhe pozvolyaet drugoj parallel`ny'j rost.}
\priklad{ "Dávali mi pět let života a už je to sedm," říká bez emocí na svém lůžku v domě pro paliativní péči Victor-Gadbois v Beloeil, kam přijel předešlý den. }{ "Они дали мне пять лет жизни, и это семь", говорит он без эмоций на его постели у себя дома для паллиативной помощи Виктор-Гадбуа в Beloeil, куда они прибыли в предыдущий день. }{"Oni dali mne pyat` let zhizni, i e`to sem`", govorit on bez e`mocij na ego posteli u sebya doma dlya palliativnoj pomoshhi Viktor-Gadbua v Beloeil, kuda oni priby'li v predy'dushhij den`.}
\priklad{ Opatrnost je ovšem na místě například na některých přemostěních, kde může být povrch namrzlý a kluzký. }{ Внимание, однако, находится в месте, например, некоторые перемычки, где поверхность может быть ледяной и скользкий. }{Vnimanie, odnako, naxoditsya v meste, naprimer, nekotory'e peremy'chki, gde poverxnost` mozhet by't` ledyanoj i skol`zkij.}
\priklad{ Prostě je ignoruji. }{ Просто игнорируйте их. }{Prosto ignorirujte ix.}
\priklad{ Podle doktorky Christiane Martelové není quebecký zdravotnický systém dostatečně výkonný, aby zajistil přístup všech osob ke kvalitní paliativní péči, než bude možno souhlasit s provedením eutanazie. }{ По словам доктора Кристиана Martel Квебеке система здравоохранения не является до\-ста\-то\-чно мощным, чтобы обеспечить доступ для всех людей на высококачественной паллиативной помощи, прежде чем он может согласиться проводить эвтаназию. }{Po slovam doktora Kristiana Martel Kvebeke sistema zdravooxraneniya ne yavlyaetsya dostatochno moshhny'm, chtoby' obespechit` dostup dlya vsex lyudej na vy'so\-ko\-ka\-che\-stve\-nnoj palliativnoj pomoshhi, prezhde chem on mozhet soglasit`sya pro\-vo\-dit` e`vtanaziyu.}


}


\section{Bing Translator}
\subsection{Intercorp}


\priklady{

\priklad{ Je mi teprve čtyřiadvacet let a nemohu prožit celý svůj život s legitimací invalidy práce a potloukat se po nemocnicích , když vím , že je to marné~. }{ Мне только двадцать четыре года, и я не могу prožit всю мою жизнь с доказательствами инвалидов на работу и общаться с больницы, когда я знаю, что это бесполезно. }{Mne tol`ko dvadczat` chety're goda, i ya ne mogu prožit vsyu moyu zhizn` s dokazatel`stvami invalidov na rabotu i obshhat`sya s bol`nicy', kogda ya znayu, chto e`to bespolezno.}
\priklad{ Generál chodil po pokoji sem a tam , kouře svou pěnovku~. }{ Генерал ходил вокруг комнаты здесь и там, дым ИТ. }{General xodil vokrug komnaty' zdes` i tam, dy'm IT.}
\priklad{ " Možná že žádné brilianty neexistují ? " }{ «Может есть не алмазы?» }{«Mozhet est` ne almazy'?»}
\priklad{ Nesměl při tom udělat chybu , vyžadovalo to stejnou přesnost , jako když se zaměřuje dělo~. }{ Ему не было разрешено сделать ошибку, он требует такой же точности, как цели пушки. }{Emu ne by'lo razresheno sdelat` oshibku, on trebuet takoj zhe tochnosti, kak celi pushki.}
\priklad{ S hrdostí vzpomněl , jak snadno dobyl kdysi srdce krásné Heleny Baurové~. }{ С гордостью, что он вспомнил, как легко завоевал однажды сердце кра\-си\-вой Елены Baurové. }{S gordost`yu, chto on vspomnil, kak legko zavoeval odnazhdy' serdce krasivoj Eleny' Baurové.}
} 

\subsection{WMT}

\priklady {
\priklad{ Thiago Silva, který patří k nejlepším obráncům na světě, taky umožňuje ostatním vedle sebe růst. }{ Тиаго Силва, который является одним из лучших защитников в мире, тоже, позволяя другим расти рядом друг с другом. }{Tiago Silva, kotory'j yavlyaetsya odnim iz luchshix zashhitnikov v mire, tozhe, pozvolyaya drugim rasti ryadom drug s drugom.}
\priklad{ "Dávali mi pět let života a už je to sedm," říká bez emocí na svém lůžku v domě pro paliativní péči Victor-Gadbois v Beloeil, kam přijel předešlý den. }{ «Они дали мне пять лет жизни, и это семь,» говорит он без эмоций на его кровати в доме для паллиативной помощи, Виктор-Gadbois в Белэиле, где он имел прибыл накануне. }{«Oni dali mne pyat` let zhizni, i e`to sem`,» govorit on bez e`mocij na ego krovati v dome dlya palliativnoj pomoshhi, Viktor-Gadbois v Bele`ile, gde on imel priby'l nakanune.}
\priklad{ Opatrnost je ovšem na místě například na některých přemostěních, kde může být povrch namrzlý a kluzký. }{ Предупреждение является, однако, на месте, например, на некоторых из přemostěních, где она может быть конечно не и скользкой поверхности. }{Preduprezhdenie yavlyaetsya, odnako, na meste, naprimer, na nekotory'x iz přemostěních, gde ona mozhet by't` konechno ne i skol`zkoj poverxnosti.}
\priklad{ Prostě je ignoruji. }{ Просто игнорирует. }{Prosto ignoriruet.}
\priklad{ Podle doktorky Christiane Martelové není quebecký zdravotnický systém dostatečně výkonný, aby zajistil přístup všech osob ke kvalitní paliativní péči, než bude možno souhlasit s provedením eutanazie. }{ По словам д-р Кристиан Martelové не является достаточно мощным, чтобы обеспечить доступ всех лиц к паллиативной помощи высокого качества, чем это будет возможным согласиться выполнять эвтаназии си\-сте\-ма здра\-во\-о\-хра\-не\-ния Квебека. }{Po slovam d-r Kristian Martelové ne yavlyaetsya dostatochno moshhny'm, chtoby' obespechit` dostup vsex licz k palliativnoj pomoshhi vy'sokogo kachestva, chem e`to budet vozmozhny'm soglasit`sya vy'polnyat` e`vtanazii sistema zdravooxraneniya Kvebeka.}
}


\section{Yandex Translate}
\subsection{Intercorp}
\priklady {
\priklad{ Je mi teprve čtyřiadvacet let a nemohu prožit celý svůj život s legitimací invalidy práce a potloukat se po nemocnicích , když vím , že je to marné~. }{ Мне только двадцать четыре лет, и я не могу prožit всю свою жизнь с legitimací инвалидов труда и сутулиться вокруг после больницы , когда я знаю , что это бесполезно~. }{Mne tol`ko dvadczat` chety're let, i ya ne mogu prožit vsyu svoyu zhizn` s legitimací invalidov truda i sutulit`sya vokrug posle bol`nicy' , kogda ya znayu , chto e`to bespolezno~.}
\priklad{ Generál chodil po pokoji sem a tam , kouře svou pěnovku~. }{ Генерал ходил по комнате взад и вперед , дыма свою pěnovku~. }{General xodil po komnate vzad i vpered , dy'ma svoyu pěnovku~.}
\priklad{ " Možná že žádné brilianty neexistují ? " }{ " Возможно, что никаких бриллиантов нет ? " }{" Vozmozhno, chto nikakix brilliantov net ? "}
\priklad{ Nesměl při tom udělat chybu , vyžadovalo to stejnou přesnost , jako když se zaměřuje dělo~. }{ В дальнейшем при этом сделать ошибку , требуется такой же точности , как когда на\-пра\-вле\-на пушка~. }{V dal`nejshem pri e`tom sdelat` oshibku , trebuetsya takoj zhe tochnosti , kak kogda napravlena pushka~.}
\priklad{ S hrdostí vzpomněl , jak snadno dobyl kdysi srdce krásné Heleny Baurové~. }{ С гордостью вспомнил , как легко когда-то завоевал сердце прекрасной Елены Baurové~. }{S gordost`yu vspomnil , kak legko kogda-to zavoeval serdce prekrasnoj Eleny' Baurové~.}
} 
\subsection{WMT}

\priklady {
\priklad{ Thiago Silva, který patří k nejlepším obráncům na světě, taky umožňuje ostatním vedle sebe růst. }{ Тьяго Силва, который относится к лучшим усталых защитников в мире, тоже по\-зво\-ляет другим рядом друг с другом рост. }{T`yago Silva, kotory'j otnositsya k luchshim ustaly'x zashhitnikov v mire, tozhe pozvolyaet drugim ryadom drug s drugom rost.}
\priklad{ "Dávali mi pět let života a už je to sedm," říká bez emocí na svém lůžku v domě pro paliativní péči Victor-Gadbois v Beloeil, kam přijel předešlý den. }{ "Дали мне пять лет жизни, и уже это семь", говорит без эмоций на своей койке в доме для паллиативной помощи Виктор-Gadbois в Beloeil, куда прибыл последний день. }{"Dali mne pyat` let zhizni, i uzhe e`to sem`", govorit bez e`mocij na svoej kojke v dome dlya palliativnoj pomoshhi Viktor-Gadbois v Beloeil, kuda priby'l poslednij den`.}
\priklad{ Opatrnost je ovšem na místě například na některých přemostěních, kde může být povrch namrzlý a kluzký. }{ Осторожность, однако, на месте, например, на некоторых přemostěních, где может быть поверхность namrzlý и скользкий. }{Ostorozhnost`, odnako, na meste, naprimer, na nekotory'x přemostěních, gde mo\-zhet by't` po\-verx\-nost` namrzlý i skol`zkij.}
\priklad{ Prostě je ignoruji. }{ Просто игнорирую. }{Prosto ignoriruyu.}
\priklad{ Podle doktorky Christiane Martelové není quebecký zdravotnický systém dostatečně výkonný, aby zajistil přístup všech osob ke kvalitní paliativní péči, než bude možno souhlasit s provedením eutanazie. }{ По doktorky Кристиан Martelové не quebecký системы медицинского до\-ста\-то\-чно мощ\-ный, чтобы обеспечить доступ всех людей к качественной па\-лл\-иа\-ти\-вной по\-мо\-щи, чем можно будет согласиться с проведением эв\-та\-на\-зи\-и. }{Po doktorky Kristian Martelové ne quebecký sistemy' medicinskogo dostatochno moshhny'j, chtoby' obespechit` dostup vsex lyudej k kachestvennoj palliativnoj pomoshhi, chem mozhno budet soglasit`sya s provedeniem e`vtanazii.}
}


\section{Moses}
\subsection{Intercorp}
\priklady {
\priklad{ Je mi teprve čtyřiadvacet let a nemohu prožit celý svůj život s legitimací invalidy práce a potloukat se po nemocnicích , když vím , že je to marné~. }{ Мне только 24 лет , и я не могу метаниях всю свою жизнь с легитимностью инвалидов работы и слоняться по госпиталям , когда я знаю , что это бесполезно~.  }{Mne tol`ko 24 let , i ya ne mogu metaniyax vsyu svoyu zhizn` s legitimnost`yu invalidov raboty' i slonyat`sya po gospitalyam , kogda ya znayu , chto e`to bespolezno~. }
\priklad{ Generál chodil po pokoji sem a tam , kouře svou pěnovku~. }{ Генерал ходил по комнате , свою pěnovku дыма~.  }{General xodil po komnate , svoyu pěnovku dy'ma~. }
\priklad{ " Možná že žádné brilianty neexistují ? " }{ - Возможно , что не существует никаких бриллиантами ? "  }{- Vozmozhno , chto ne sushhestvuet nikakix brilliantami ? " }
\priklad{ Nesměl při tom udělat chybu , vyžadovalo to stejnou přesnost , jako když se zaměřuje dělo~. }{ При этом нельзя было сделать ошибку , это же точности , как , например , когда про\-ис\-хо\-ди\-ло уничтожение~.  }{Pri e`tom nel`zya by'lo sdelat` oshibku , e`to zhe tochnosti , kak , naprimer , kogda proisxodilo unichtozhenie~. }
\priklad{ S hrdostí vzpomněl , jak snadno dobyl kdysi srdce krásné Heleny Baurové~. }{ С гордостью вспомнил , как легко он когда-то сердце прекрасной Елены baurové~.  }{S gordost`yu vspomnil , kak legko on kogda-to serdce prekrasnoj Eleny' baurové~. }
} 


\subsection{WMT}

\priklady {
\priklad{ Thiago Silva, který patří k nejlepším obráncům na světě, taky umožňuje ostatním vedle sebe růst. }{ Thiago Сильва , который является лучшим защитником в мире , и по\-зво\-ляет другим рядом экономический рост~.  }{Thiago Sil`va , kotory'j yavlyaetsya luchshim zashhitnikom v mire , i pozvolyaet drugim ryadom e`konomicheskij rost~. }
\priklad{ "Dávali mi pět let života a už je to sedm," říká bez emocí na svém lůžku v domě pro paliativní péči Victor-Gadbois v Beloeil, kam přijel předešlý den. }{ " Мне давали пять лет , и это уже семь , - говорит без эмоций в своей постели в доме для обслуживания полумерой Виктор beloeil , куда он приехал в предыдущий день~.  }{" Mne davali pyat` let , i e`to uzhe sem` , - govorit bez e`mocij v svoej posteli v dome dlya obsluzhivaniya polumeroj Viktor beloeil , kuda on priexal v predy'dushhij den`~. }
\priklad{ Opatrnost je ovšem na místě například na některých přemostěních, kde může být povrch namrzlý a kluzký. }{ Осторожность , однако , на месте , например , на некоторых обводку , где может быть поверхность полосе и скользким~.  }{Ostorozhnost` , odnako , na meste , naprimer , na nekotory'x obvodku , gde mozhet by't` poverxnost` polose i skol`zkim~. }
\priklad{ Prostě je ignoruji. }{ просто игнорируют~.  }{prosto ignoriruyut~. }
\priklad{ Podle doktorky Christiane Martelové není quebecký zdravotnický systém dostatečně výkonný, aby zajistil přístup všech osob ke kvalitní paliativní péči, než bude možno souhlasit s provedením eutanazie. }{ По словам доктора Кристиан Мартелл не Квебек служба здравоохранения достаточно полезным , чтобы обеспечить доступ всех людей полумерой качества обслуживания , чем можно будет согласиться с сделали э\-вта\-на\-зи\-я~.  }{Po slovam doktora Kristian Martell ne Kvebek sluzhba zdravooxraneniya dostatochno polezny'm , chtoby' obespechit` dostup vsex lyudej polumeroj kachestva obsluzhivaniya , chem mozhno budet soglasit`sya s sdelali e`vtanaziya~. }
}


\section{TectoMT}
\subsection{Intercorp}
\priklady {
\priklad{ Je mi teprve čtyřiadvacet let a nemohu prožit celý svůj život s legitimací invalidy práce a potloukat se po nemocnicích , když vím , že je to marné~. }{ Он  мне только сутки годы и , я не могу пережить всю его жизнь с до\-ку\-мен\-том инвалида работы и слонялись его по больницах , когда я знаю , что  это бесполезно~. }{On  mne tol`ko sutki gody' i , ya ne mogu perezhit` vsyu ego zhizn` s dokumentom invalida raboty' i slonyalis` ego po bol`niczax , kogda ya znayu , chto  e`to bespolezno~.}
\priklad{ Generál chodil po pokoji sem a tam , kouře svou pěnovku~. }{ Генерал ходил по комната сюда , и там , курят его шумовка~. }{General xodil po komnata syuda , i tam , kuryat ego shumovka~.}
\priklad{ " Možná že žádné brilianty neexistují ? " }{ Что » может никакие бриллианты не существуют » ? }{Chto » mozhet nikakie brillianty' ne sushhestvuyut » ?}
\priklad{ Nesměl při tom udělat chybu , vyžadovalo to stejnou přesnost , jako když se zaměřuje dělo~. }{ При этом он не сделал ошибку , требовало это же точность , когда как сосредоточиться пушка~. }{Pri e`tom on ne sdelal oshibku , trebovalo e`to zhe tochnost` , kogda kak sosredotochit`sya pushka~.}
\priklad{ S hrdostí vzpomněl , jak snadno dobyl kdysi srdce krásné Heleny Baurové~. }{ С гордостью он вспомнил , как легко он завоевал когда-то сердце красивой Елены Баурова~. }{S gordost`yu on vspomnil , kak legko on zavoeval kogda-to serdce krasivoj Eleny' Baurova~.}
} 
\subsection{WMT}

\priklady {
\priklad{ Thiago Silva, který patří k nejlepším obráncům na světě, taky umožňuje ostatním vedle sebe růst. }{ Тгиаго Силва , который принадлежит к хорошим защитникам на мире , также он по\-зво\-ляет другим возле него роста~. }{Tgiago Silva , kotory'j prinadlezhit k xoroshim zashhitnikam na mire , takzhe on pozvolyaet drugim vozle nego rosta~.}
\priklad{ "Dávali mi pět let života a už je to sedm," říká bez emocí na svém lůžku v domě pro paliativní péči Victor-Gadbois v Beloeil, kam přijel předešlý den. }{ » они давали мне пять годы жизни и , уже  это семь » , он говорит без эмоций на его постели в доме для Паллиативное ухода вицтор-гадбоис в Бэлоэил , куда он приехал предыдущий день~. }{» oni davali mne pyat` gody' zhizni i , uzhe  e`to sem` » , on govorit bez e`mocij na ego posteli v dome dlya Palliativnoe uxoda vicztor-gadbois v Be`loe`il , kuda on priexal predy'dushhij den`~.}
\priklad{ Opatrnost je ovšem na místě například na některých přemostěních, kde může být povrch namrzlý a kluzký. }{ Осторожность  но на месте например на некоторых мост , где  поверхность псих и скользкая~. }{Ostorozhnost`  no na meste naprimer na nekotory'x most , gde  poverxnost` psix i skol`zkaya~.}
\priklad{ Prostě je ignoruji. }{ Просто их я игнорирую~. }{Prosto ix ya ignoriruyu~.}
\priklad{ Podle doktorky Christiane Martelové není quebecký zdravotnický systém dostatečně výkonný, aby zajistil přístup všech osob ke kvalitní paliativní péči, než bude možno souhlasit s provedením eutanazie. }{ По доктору Христиан Мартел не  qуэбэцкый здравоохранение система до\-ста\-то\-чно исполнительные , чтобы он обеспечил доступ всех людей к ка\-чес\-тве\-нный Паллиативное уходу , než он будет возможно согласен с вы\-пол\-не\-нием эвтаназии~. }{Po doktoru Xristian Martel ne  que`be`czky'j zdravooxranenie sistema dostatochno ispolnitel`ny'e , chtoby' on obespechil dostup vsex lyudej k kachestvenny'j Palliativnoe uxodu , než on budet vozmozhno soglasen s vy'polneniem e`vtanazii~.}

}



\renewcommand{\thechapter}{B}
\chapter{Data on the attached disk}

I am attaching a hard drive to this thesis.\footnote{Because of technical difficulties, I have only one copy instead of three.} For a better compatibility, the drive is formatted with NTFS.\footnote{For some reason, Ubuntu's Nautilus refuses to display some of the folders, while \texttt{ls} seems to work fine. I do not have resources to investigate this further; it might be connected with the fact some of the folders were created on Mac OS X.}

The disk has several subfolders:
\begin{itemize}
\item \texttt{corpora} for the corpora
\item \texttt{systems}  for the systems and experiments
\item \texttt{evaluation}  for the some evaluation scripts (that include the results of the translation)
\item \texttt{thesis} for \XeLaTeX source code of this thesis plus its PDF version
\end{itemize}

All the included scripts are mostly experimental and, as most of used systems and frameworks themselves (Moses, TectoMT, GNUstep...), they are not easy to run. I have \emph{not} tried to run any of the experiments anywhere else than on the ÚFAL network. \footnote{Except for the VMWare/PC Translator setup, that I have run only on my personal computer.}

For reference: ÚFAL network is made of 64-bit Ubuntu 10.04 LTS installations, with perl 5.10 and Sun Grid Engine installed.

Some of the corpora and systems have special licenses that \emph{don't allow them to be shared}; for example, in the Intercorp license agreement, I had to sign that:
\begin{quotee}The User agrees not to re-distribute or otherwise make publicly available the SCD, or any derivative work based on it\end{quotee}

I also include a VMWare virtual machine with pre-installed Microsoft Windows (that I don't have legal permission to share) and PC Translator (that I don't have legal permission to share). 

My understanding of Czech copyright law is that it's legal to share such data in academic, non-commercial purposes, such as attaching them to a thesis on a hard drive.

\section{Corpora}
The folder \texttt{corpora} has several subfolders:
\begin{itemize}
\item \texttt{original\_data} for the raw, original data, as downloaded
\item \texttt{scripts} for some of the extraction scripts
\item \texttt{cleaned\_data} for already filtered corpora
\item \texttt{unused} for the unused data
\end{itemize}

\subsection{Original data}
\subsubsection{WMT}
Both WMT test sets are in the folder \texttt{wmt}. The files were downloaded from \url{http://www.statmt.org/wmt13/translation-task.html}.

\texttt{wmt/test\_2013.tgz} has several SGML\footnote{As far as I know, SGML is a superset of XML; however the files seem like well-formed XML; not valid, because the DTDs are not present} files for every language in the competition. With every document, information about original language is included.

\texttt{wmt/test\_2012.tgz} includes more languages and even previous years.

The previously mentioned webpage is also saved in the \texttt{wmt/wmt.html} file.
\subsubsection{Intercorp}
\texttt{intercorp/mixed.gz} is a gzipped text file with all the data from the mixed corpus. Each line has both Czech and Russian text, divided by a tabulator.

\texttt{intercorp/filtered/data.tgz} is all the Intercorp data.\footnote{This data source is under a license agreement, that's in the \texttt{License\_Agreement.odt} file.}

The tarred and gzipped file includes \texttt{intercorp\_shuff\_cs} and \texttt{intercorp\allowbreak \_shuff\allowbreak \_ru}, that include the book data (both sentences and metadata) in a strange, XML-like format. The sentences in the books are shuffled.

The file \texttt{intercorp\_shuff\_ru2cs} includes the linking of the sentences.

\subsubsection{UMC}
UMC corpus is in the files \texttt{umc/umc-0.1-corpus.zip} and \texttt{umc/\allowbreak umc003-\allowbreak cs-\allowbreak en-\allowbreak ru-\allowbreak triparallel-testset.zip} zipped, as downloaded from the UMC website, that's also saved in the folder \texttt{umc/doc}.

In UMC 0.1, all that matters to us is the file \texttt{Czech-Russian.1-1.txt} with the sentences that are linked to one another.

In UMC 003, the sentences are strangely mixed (and the \texttt{README} file is not entirely accurate) and strangely lowercased. The only non-lowercased text is in the file \texttt{all/ps2009.tok.csenru.gz}.

\subsubsection{Wikipedia titles}
As I already mentioned in \ref{corpora:wiki}, wikipedia now use a different format of inter-language linking somewhere in 2013, where my old script no longer works. I do not have the original dump; I, however, have an older 2012 dump on which my script works.

The dump is in the file \texttt{wiki/cswiki-\allowbreak 20121112-\allowbreak pages-\allowbreak articles.xml.bz2}

\subsubsection{News Crawl}
All News Crawl corpora are in the folder \texttt{newscrawl}. The files are exactly as downloaded from the page already mentioned in the section WMT.

The files are tarred and gzipped in the \texttt{training-\allowbreak monolingual-\allowbreak news-\allowbreak 2008.\allowbreak tgz} (and similar for other years). There are more language files in each of them.

\subsubsection{Common Crawl}
Common Crawl is in the folder \texttt{commoncrawl}. The file is exactly as downloaded from the page already mentioned in the section WMT.

The files are tarred and gzipped in the file \texttt{training-\allowbreak parallel-\allowbreak commoncrawl.tgz}. 

We use only \texttt{commoncrawl.\allowbreak ru-\allowbreak en.\allowbreak ru}
with the Russian text, but in the file \texttt{commoncrawl.\allowbreak ru-\allowbreak en.\allowbreak annotation}, there are links to sources of all the data.
\subsubsection{Yandex}
Yandex data are in the \texttt{yandex} directory, tarred and gzipped in the \texttt{corpus.\allowbreak en\_ru.\allowbreak 1m.\allowbreak tgz} file. The file contains just two text files -- one for English (that we don't use) and one for Russian.
\subsubsection{Subtitles}
Subtitles are as given to us, in the folder \texttt{subtitles}.


\subsection{Scripts}
Most of the data require only some very easy one-liners to prepare; I have included only the more complicated scripts. 

As I mentioned before, those scripts were intended for one-time use on specific computers and I am not guaranteeing their reusability; however I am including them for completeness.

\subsubsection{Intercorp}
Scripts for extracting Filtered Intercorp data (\ref{corpora:filteredintercorp}) from the original data are in the \texttt{intercorp} directory.

Following must be done before running any of the scripts:
\begin{itemize}
\item the \texttt{intercorp\_shuff\_ru} has to be corrected to be well formed XML by enclosing with a big \texttt{<all>} tag; also some other minor corrections (like replacing \texttt{\&} with \texttt{\&amp;} and so on), and saved as \texttt{intercorp\_shuff\_ru.corrected}, similar with the Czech data
\item the \texttt{intercorp\_shuff\_ru2cs} data has to be split into separate XML files (for example by using UNIX \texttt{split}) and saved as \texttt{xx01} to \texttt{xx86}
\end{itemize}

The scripts then do the following:
\begin{itemize}
\item \texttt{make\_info.pl} parses the corrected XML and prints metadata about the books in YAML into \texttt{info.yaml}
\item \texttt{extract\_text.pl} extracts the text from one of the \texttt{xx} books and prints it in \texttt{splitbooks} directory; first argument to the script is the number of the book.  
\item \texttt{are\_used.pl} takes the data in \texttt{info.yaml}, the directory \texttt{splitbooks} and a sorted corpus (its address is the first argument) and detects which books were and which weren't used in the corpus
\item \texttt{direct.pl} takes the info from the last script and determines, which books were not used and were direct translations of each other, instead of translation from third language, and prints this information
\end{itemize}

\subsubsection{Subtitles}
%This script is not that useful, since we need the raw subtitle files that I no longer have (\ref{corpora:subtitles}), but again, I am providing it for completeness.

The \uv{script} is actually the whole FilmTit project (\ref{filmtit}) with a lot of unrelated modules, but also with a subtitle alignment module. For running it, we need correctly set-up Maven.\footnote{Scala is probably not needed as Maven installs it correctly based on the pom.xml file. The project is tested only with Maven 2, see next footnote.} 

\texttt{align\_files.sh} runs the correct module; the input and output directories are defined as variables at the top of the script.\footnote{This script
hasn't been run in two years, so it is highly untested.}

\subsubsection{Wikipedia}
\texttt{extract.pl} is a perl script, that extracts the title pairs from a Czech wikipedia dump.\footnote{As noted elsewhere, the dump has to be in an old, pre-2013 format, like the file that's on the disk.}

\section{Systems}
In the folder \texttt{systems}, there are several systems available.
\subsection{RUSLAN}
I did not find a working RUSLAN copy, so I am not providing that; I am, however, including a RUSLAN dictionary plus the experiments, described in \ref{experiments:webcoll}.

\subsection{Česílko 1.0}
Česílko 1.0 is in the folder \texttt{cesilko10}. I haven't done any experiments with the code, as noted in \ref{experiments:cesilko10}.

\subsection{Česílko 2.0}
My slightly fixed version of Česílko 2.0 is in the folder \texttt{cesilko20}; the difference between the original code and my slightly fixed code is in the file \texttt{code\_diff}.

\subsection{Online systems}
Scripts for online systems are in the \texttt{online\_systems} directory; the java library for Google Translate, the PHP script for Bing Translator and a simple shell script for Yandex Translate.

\subsection{PC Translator}
PC Translator virtual machine and helper script are included in the directory \texttt{pc\_translator}. 
\begin{itemize}
\item \texttt{vmware} is the VMWare machine itself, with working Windows XP and PC Translator v14.
\item \texttt{sharedfolder} is a folder that has to be set up in VMWare Tools on the Windows XP machine correctly as \texttt{sharedfolder}
\item \texttt{run/do\_csru.pl} is the script, that tries to start to convert STDIN into PC Translator-friendly format, then moves the AutoHotkey script into the shared folder, then starts the machine that -- if set up correctly -- will start PC Translator, translate the file and turn itself off.
\end{itemize}

I have not tested this script outside of my own computer, so I am not sure how well it works.

\subsection{Moses}
I am providing a standard \texttt{eman} playground (together with \texttt{eman} version I have been using). The experiments relevant to this thesis are written in the \texttt{relevant\_experiments} file.

\subsection{TectoMT}
I am also providing the latest TectoMT version, together with the \texttt{share} folder that actually includes my models.

Numbers of revisions in TectoMT SVN, that are my improvements, are written in the \texttt{my\_revisions} file.


\section{Evaluation}
In the folder \texttt{evaluation}, I include
\begin{itemize}
\item the actual results of the translation, plus my small perl script for BLEU evaluation (and generating the examples for section A )
\item the PHP script for human evaluation (plus SQL dump of the server)
\item slightly modified TrueSkill
\end{itemize}








\openright
\end{document}

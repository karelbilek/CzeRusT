\chapter*{Introduction}
\addcontentsline{toc}{chapter}{Introduction}
Slavic  family of languages consists of around 12 languages, de\-pen\-ding on the exact definition.\footnote{See for example \cite{sussex2011slavic} or \cite{siewierska1998overview} --  the final language count depends on whether Upper and Lower Sorbian are taken as different languages or not; whether Kashubian is a separate language, or just a variety of Polish; whether Serbian and Croatian are treated as separate languages or not}
They are usually divided into three sub-groups - South Slavic, West Slavic and East Slavic. While languages inside each of those groups are mutually intelligible, intelligibility accross those groups is lower; still, the Slavic languages in general have very similar morphology, syntax and vocabulary.

Czech language belongs to the West Slavic group, while Russian belongs to the East Slavic group.  Therefore, they are not fully mutually intelligible -- but, as noted, as Slavic languages, they are in some respects still very similar. 
This creates a great environment for building machine translations tools -- the languages are not too close to make the task meaningless, while we can still take an advantage of the closeness.

For political reasons, significant efforts were directed towards building an automatic translation system between Russian and Czech, mainly during mid-eighties with the RUSLAN system. The efforts were terminated in 1990 in the final phases of development.\footnote{See for example \cite{recycled}, \cite{hajic1987}, \cite{olivaruslan}} 

After the velvet revolution, political dependence on Russian-speaking countries were weakened, while the economical and cultural dependence on English-speaking countries were strengthened. Therefore, the efforts of automatic translation were more focused on English. 

In the process, general direction of machine translation efforts shifted from rule-based systems to more statistical-based systems. This allows us to compare older, rule-based approaches to the more modern statistical approaches.

In this thesis, I am trying to describe the rich history of Czech-to-Russian machine translation, experiment with the available tools -- both historical and more recent ones -- on a given set of data, and describe and compare the results.

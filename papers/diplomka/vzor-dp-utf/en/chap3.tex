\chapter{Unrunnable systems}
\section{RUSLAN}

RUSLAN is a machine-translation system, developed between 1984 and 1988 at several departments of Charles University, Prague.

\subsection{Q-Systems}
Q-Systems (sometimes also Systems Q) -- Q stands for \uv{Quebec} -- are a tool for machine translation, developed at Montreal University by Alain Colmerauer, also the creator of Prolog.\footnote{See \cite{qsystems}.} 

Q-Systems are similarly declarative as Prolog, focusing more on the result than the procedural order of analysis. If there is any ambiguity, all the possibilities are explored \emph{in parallel}. In theory, this could make writing the lexical rules better; in reality, the rules are still quite unreadable, as will be seen later.

Q-Systems are not very widely used or widely worked with. One of the reasons might be the fact that all documentation is in French. However, \cite{nguyen2009systemes} describes modern-day experiments with Systems Q, also mentioning, that all Fortran implementations has been lost\footnote{Which would make UFAL's version, if it was working, quite unique.} and reimplementing it in C. I haven't been able to try this version.


\subsection{RUSLAN components}

Description of the system can be found in \cite{olivaruslan} or \cite{hajic1987} -- however, the reader has to bear in mind that not only the systems are severly dated, but so is their manual and description. \footnote{At least for me personally, the book \cite{olivaruslan} especially was hard to read and hard to find needed information in.} Contemporary (but not as detailed) description of the system can be found in \cite{recycled}.

The whole RUSLAN system has several components:
\begin{itemize}
\item preprocessing, written in Pascal
\item morphological analysis, using dictionary, written in Q-Systems and interpreted in Fortran IV
\item syntactico-semantical analysis, using morphology, also written in Q-Systems; this component uses FGD as its theoretical starting point
\item generation, also using Q-Systems
\item morphological synthesis, using Pascal
\end{itemize}

RUSLAN uses a Czech-to-Russian dictionary, written by hand in afforementioned Q-Systems. Dictionary item looks like this:

\begin{verbatim}
DLOUH==M(RS(+(*INT)),MI2289,DLINNYJ).
DLOUH==M(RS(-(*INT)),MI2276,DOLGIJ).
\end{verbatim}

This describes two possibilites of the translation of the word \uv{dlouhý} to Russian: the first is \uv{длинный} and the second is \uv{долгий}. They differ by the semantic feature INT (TODO: find out) they require or forbid from the word they depend on. 

More complex dictionary item looks like this:
\begin{verbatim}
C3ES3TIN
  ==Z(@(*A), MIO109, $(JAZYK), 
     2(POS, #($), &, $(MI28), $(C2ES2KIJ),
       1(=,@($), #($),$($))),
     1(=,@($),#($),$($))).
\end{verbatim}

This item translates the word \uv{čeština} to Russian words \uv{чешский язык} and also describes their relationship.

Maybe because memory was more expensive than today, all similar rules are in the dictionary without any comments, leaving only very difficult-to-decypher rules.

The rules for analysis are even less readable. Random examplo of two such rules are as follows:

\begin{verbatim}
1(B*, X*1, /, X*2, F*1(C*), X*3, /, X*4, @(V*), X*5, %(X*),
 I(*), 1(X*6, $($)), X*7)

1Z(A*9), (Z*2)
  == 1(D*, /, X*2, F*1/X*),/,@(V*),X*5,%(X*),1*,1(X*6,$($)),X*7,A*B,
     5(U*1, @(U*2), U*3, $(U*), 3(E*(Y*1), B*(C*), W*1, W*3, %(X*), 
        $(W*)),
     +1Z(A*9, Z*2)
       / -NON- (, + -DANS- X*9 -ET- +(V*) -HORS- X*9, +(VZT)
        -ET- -(V*) -HORS- X*9, *
        -ET- C* = S
        -ET- X*3,* -HORS- /,N(S), S(S), D(S), A(S), L(S), I(S)
        -ET- / -HORS- X*2, 2
        -ET- (, Y%1 = -NUL-
             -OU- E*(Y*1) -HORS- U*2,*
             -OU- E*(+(V*, *)) -HORS- U*2, *
             -OU- -NON- E*(-(V*)) -HORS- U*2, * .)
        -ET- (. E*(Y*1) *N
             -OU- H(B*(C*)) -DANS- U*1 .) .
\end{verbatim}

Of course this is all left without any comments, and the only comment for these two rules and about 20 more is \uv{RELATIVE CLAUSES ADJOINED TO THEIR HEADS}.

\subsection{Dictionary coverage of WebColl}

The dictionary contains about 8,000 lexical items. However, the domain of the translation and, therefore, the dictionary, was manuals for old computers from 1980's. 

In different experiments (\cite{florida}), we tried to measure how many nouns from the RUSLAN dictionary appear at all in a modern text.

For that, we used a monolingual Czech corpus WebColl, consisting of roughly 7 million sentences (114 million tokens)\footnote{See \cite{webcoll}; the corpus is also more thoroughly described in the chapter ????}.

RUSLAN dictionary has 2,783 nouns. In the WebColl corpus, from those nouns, 611 appear less than 10 times -- and from those, 412 don't appear \emph{at all}.

The reverse is similarly infavourable: from 39,434,505 nouns in the corpus, only 11,862,221 is in the dictionary.

\subsection{Experiments}

Despite the general un-maintainability of the RUSLAN code and despite the small dictionary, we tried to run the system on our test data.

However, all our experiments ended in some sort of error.

Because I am not able to code in neither Systems-Q nor FORTRAN (in which the Systems-Q interpreter is coded), I gave up on this experiment.


\section{Česílko}

Česílko is actually a name for two different machine translation systems with slightly different goals and, more importantly, slightly different structure.

\subsection{Česílko 1.0}
Česílko 1.0 was a system, developed in 2000, and was aimed for direct translation between Czech and Slovak and was intended to assist a translation memory\footnote{see \cite{cesilko1}}. The translation works lemma-by-lemma in a following fashion:
\begin{itemize}
\item morphological analysis of source Czech language
\item disambiguation
\item direct translation, lemma-by-lemma
\item morphological synthesis
\end{itemize}



The system is written in a mixture of C, C++ and Flex (fast lexical analyser generator). The code itself is not really well documented and modular, but that can be attributed to the age of some of the components -- despite the whole system being developed in 2000, some files seem to be as old as 1991.

This system itself is not very extendable from Slovak to Russian, because -- as we can see on the examples in the section \ref{sec:experiments} -- the sentences are not really translatable word-by-word.

For that reason I decided not to further experiment with Česílko 1.0 for Czech-to-Russian machine translation.

\subsection{Česílko 2.0}
Česílko 2.0 is a different project with similar goals, but using different frameworks and adding more transfer rules\footnote{See \cite{cesilko2}}. However, it has its own shares of problems, that prevented us to use it.

The system works in following fashion:
\begin{itemize}
\item \textbf{non-deterministic} morphological analysis of source Czech language
\item translation of lemmas
\item applying transfer rules by changing syntactic tree
\item morphological synthesis
\item ranking of all the generated sentences
\end{itemize}

Unlike Česílko 1.0, Česílko 2.0 uses a non-deterministic parser and explores all the possibilities in parallel. 

The more advanced transfer would, by itself, make the system more modular and extensionable for our purposes. However, the implementation details prevented us from doing so.

To illuminate why, let me focus a little on the technical details.

\subsubsection{Objective-C}
Objective-C is a very simple and elegant extension of C language, developed by Brad Cox in 1980s by adding Smalltalk features to C\footnote{See \cite{cocoa4}}. 

Objective-C is, in my opinion, very easy to learn and understand, at least compared to C++, its more popular counterpart.

Objective-C is not a proprietary language and is possible to compile with either gcc or Clang/LLVM compilers. However, what is proprietary is its most used standard library, Cocoa.

\subsubsection{Cocoa}
When Steve Jobs left Apple, he made a smaller company called NeXT. Among other things, they produced a proprietary operating system called NeXTSTEP, based on Unix.\footnote{For a more detailed history, see \url{https://developer.apple.com/legacy/library/documentation/Cocoa/Conceptual/CocoaFundamentals/WhatIsCocoa/WhatIsCocoa.html\#//apple\_ref/doc/uid/TP40002974-CH3-SW12}.}

This operating system used Objective-C as its standard language, and proprietary libraries, called OpenStep.\footnote{Despite the name, OpenStep is not open source -- the Open allude to the fact that its API specification was open.}

Several years later, Apple (now merged with NeXT) made its new version of Mac OS, called Mac OS X; this operating system was partially based on NeXTSTEP and also used some of its proprietary libraries, now renamed Cocoa.\footnote{The kernel of Mac OS X is open source, as is its \uv{underlying} operating system called Darwin -- however, this system does not contain Cocoa libraries.}

Cocoa is not the only library for Objective-C, but because Apple is the main investor in Objective-C-based systems, it's a de-facto standard library.

\subsubsection{GNUstep}
GNUstep is a free re-implementation of OpenStep/Cocoa.\footnote{See \url{http://www.gnustep.org/}}.

Its development started in the NeXTSTEP days; however, it still hasn't met feature parity with Cocoa's OS X.

Aaron Hillegass in 2nd edition of his popular book \emph{Cocoa Programming on Mac OS X} discouraged people from using GNUStep. He redacted this note in later versions of the book, perhaps because of protests from GNUstep developers\footnote{\url{http://www.gnustep.org/resources/BookClarifications.html}}, but in my opinion, his notes are still valid.

GNUstep implementations are very often buggy and not feature-complete with Cocoa and, most unfortunately, unpredictable. This is what hurt us with Česílko 2.0.

\subsubsection{Cocoa and Česílko}
When Petr Homola was writing Česílko 2.0, he decided to use Cocoa and Objective-C for development.

On Mac OS X, this configuration is just fine; however, on Linux, where we wanted to run the MT systems, this creates unpredictable results.

In my experiments with Czech-to-Slovak translations, I noticed that on Mac OS X, there are about 5-times more sentences generated, than on Linux -- while the program was compiled from the same sources.

After thorough inspection, I found out the error was in GNUstep implementation of NSDictionary -- Cocoa's implementation of associative array\footnote{\url{https://developer.apple.com/library/mac/documentation/Cocoa/Reference/Foundation/Classes/NSDictionary\_Class/Reference/Reference.html}} -- that in some unpredictable cases for two equal NSStrings returns two different values. As a result, one of the modules returned wrong inflection patterns for a number of words and the morphological analyzer then returned only a fraction of the results.

After workaround for this issue, the system returned same results on both OS X and Linux. However, I am not at all confident there aren't more similar issues in GNUstep to further develop the system for Russian; when the very basic frameworks themselves are unstable and unreliable, the development ceases to make sense.

It's possible the same issue plagued the authors of the paper \cite{evalquality_cesilko} -- it's unprobable the BLEU of the correctly working system would be that low, when in \cite{cesilko2}, the results of Česílko 2.0 were slightly better, than in Česílko 1.0. 

\chapter{Experiment setting}
\section{Statistical vs. rule-based translation systems}

Machine translation (MT) systems has historically used many different approaches. One way of classifying the approaches is on the axis of rule-based vs. statistical.

In general, we can re-use the descriptions, used by \cite{bojar}, which is as follows:
\begin{itemize}
\item rule based MT systems:
\begin{itemize}
\item use analysis, transfer and synthesis steps
\item rule-based machine translation systems use formal grammars
\item rule-based machine translation systems use hand-made dictionaries
\item has linguistic information hard-coded and therefore aren't language-agnostic
\end{itemize}
\item statistical MT systems
\begin{itemize}
    \item use more variants of outputs, rank them with some score, and choose the best one
    \item train internal dictionaries from big parallel data
    \item the core of the system is compact and its inner working less obvious
    \item uses statistics instead of linguistic rules and therefore are language-agnostic
\end{itemize}
\end{itemize}

However, with some of the modern system, the distinction is not as clear-cut as we would like, for the purpose of our comparison. 
For example, statistical MT systems like Moses can get significantly better with some added linguistic information; 
on the other hand, systems like TectoMT, which can sometimes be classified as more rule-based, actually have big modules more or less based on statistics.

%In this thesis, I will present several existing solutions to Czech-Russian machine translation, describe their position on the aforementioned axis, present their history and discuss their results on a testing data.
\section{Development and testing data}

To compare several machine translation systems, we need some consistent data to show the deficiencies and/or improvements of different MT systems.

In general, we want to produce a system that's general enough and performs well on unseen sentences; for that reason, we should have -- in the case of statistical MT systems -- separated test set and training/development set.\footnote{As noted in for example \cite{koehn2010statistical}, p. 274, or \cite{bishop}, pg. 6}

We have tried to do that in our experiments with our data models. However, some systems evaluated in this thesis are only black-box systems (Google Translate, PC Translator). Especially with Google Translate, we cannot rule out the possibility that they already use some parts of the test/development data.




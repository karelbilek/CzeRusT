\chapter{Brief comparison of Czech and Russian}

This section is an overview of history of Slavic languages, and also of some characteristics of Slavic languages in general.

%If a reader is interested in a further study of Slavic languages as a whole, I can only recommend \cite{sussex2011slavic}, which covers both history and
%linguistic characteristics of Slavic languages on approximately 660 pages.
%Most of the factual information in this chapter is from the afforementioned book.

\section{History}

%Unless otherwise noted, information in this chapter are taken from this book.

\subsection{Common history}

\label{ch:common_history}

Using comparative linguistics, Slavic languages in general can be traced back to Indo-European language family, with Proto-Indo-European (PIE) as its reconstructed ancestral language. 

As greatly described in \cite{oxfordintro}, taking Proto-Indo-Europeans as unified people with unified history can be somehow controversial -- we are, as noted in that book, trying to \uv{put absolute dates on a hypothetical construct}. Still, based on the reconstructed Proto-Indo-European language having, for example, word for wheeled wehicles, we can date Indo-Europeans on Great Eurasian Plain to approximately 4000 BC, as noted by \cite{sussex2011slavic}.

\cite{sussex2011slavic} further notes that our knowledge of this part of European pre-history is \uv{sketchy and partly conjectural}. He further notes:

\begin{quotee}Although we can only guess how far their territory extended, it is possible that at least the European center of the Indo-European homeland -- if
not the original homeland itself (on one widely held view) -- was in what is now Western Ukraine, and that they spoke a fairly homogeneous language.\end{quotee}

\cite{oxfordintro} further describes the reconstruction of this PIE. In here, however, this excerpt from \cite{ringe2008proto} will suffice:

\begin{quotee}Though there continue to be gaps in our knowledge of PIE, an astonishing
proportion of its grammar and vocabulary are se\-cu\-re\-ly re\-con\-struct\-a\-ble by the
comparative method. As might be expected from the way the method works,
the phonology of the lan\-guage is re\-la\-ti\-ve\-ly cer\-tain. Though syntactic reconstruction is in its infancy, PIE syntax is also relatively uncontroversial because
the earliest-attested daughter languages agree so well.\end{quotee}

With similar logic, we can reconstruct an ancestral language to all Slavic languages, call it Proto-Slavic and put it on a place in space and time.

\cite{sussex2011slavic} puts the emergence of Proto-Slavic at around 2000-1500 BC. 


%on the other hand, names the language at the time of Slavic unity \uv{Early Slavic} and \uv{Middle Slavic} and places Proto-Slavic only at 600 AD.

\cite{schenker1993proto} adds, rather poetically:
\begin{quotee}The Slavs were the last Indo-Europeans to appear in the annals of
history. Slavonic texts were not recorded till the middle of the ninth
century and the first definite reference to the Slavs' arrival on the frontiers
of the civilized world dates from the sixth century AD, when the Slavs
struck out upon their conquest of central and south-eastern Europe. Before
that time the Slavs dwelled in the obscurity of their ancestral home, out of
the eye-reach of ancient historians. Their early fates are veiled by the
silence of their neighbours, by their own unrevealing oral tradition and by
the ambiguity of such non-verbal sources of information as archaeology,
anthropology or palaeobotany.\end{quotee}

\cite{kortlandt1982early} breaks this period into several sub-periods, like Balto-Slavic, Middle Slavic and Proto-Slavic; while Balto-Slavic, in his approximation, appears at 2000 BC, Late Proto-Slavic disappears and disintegrate in 900-1200 AD. This division is also slightly mentioned in \cite{schenker1993proto}.

\cite{sussex2011slavic} also sums up the spacial location of Slavic areas before breaking to individual languages:

\begin{quotee}By the fourth century AD the Slav area
stretched from the Oder (Pol \emph{Odra}) River in the west to the Dnieper (Rus \emph{Dnepr},
Ukr \emph{Dnipró}) in the east. In the north they had reached the Masurian Lakes in
central Poland, the Baltic Sea and the Pripet (Pol \emph{Prypeć}; also Eng \emph{Pripyat}, from
Ukr \emph{Prýp’jať}) Marshes. During this period the Slavs would have spoken a fairly
uniform language. Although dialect differences soon began to appear, resulting
\emph{inter alia} in the division into Baltic, Slavic or an intermediate Balto-Slavic, the pace
of linguistic change was relatively slow.\end{quotee}

According to \cite{curta2004slavic}, some form of Slavic was still used as a \emph{lingua franca} in Avar qaganate by about 700 AD, but no further than in 800 AD.

\subsection{Division of the languages}


Continuing its narrative, \cite{sussex2011slavic} notes about breaking of Proto-Slavic:
\begin{quotee}According to general consensus in what is still a controversial area, the real break-up
of Proto-Slavic unity began about the fifth century AD. There seems to have been a
steady expansion to the north and east by the Eastern Slavs. For the others there is
evidence that their migrations were related partly to the disintegration of the Roman
and Hun empires and the ensuing vacuum in Central Europe. 

One group of Slavs
moved westwards, reaching what is now western Poland and the Czech Republic, and
the eastern and north-eastern part of modern Germany. 

A second wave broke away
to the south towards the Balkan Peninsula, where they became the dominant ethnic
group in the seventh century, some (in the east) in turn being conquered by the
Bulgars, a non-Slavic people of Turkic Avar origin.\end{quotee}

\cite{oxfordintro} also puts the point of Slavic break-up at arount 500, while for example \cite{kortlandt1982early} puts it at 900-1200 AD, and \cite{schenker1993proto} puts it at about 900 AD.


While sources disagree on what exactly are all the Slavic languages in which group (partly because question of a language distinction is a political one), all sources\footnote{\cite{sussex2011slavic}, \cite{oxfordintro}, \cite{kortlandt1982early} and others already mentioned} agree on the division to South Slavic languages, East Slavic languages and West Slavic languages; West Slavic being the westwards moving group from the above narrative, South Slavic being the group moving to Balkan and Eastern Slavic being the group moving eastwards.

According to \cite{sussex2011slavic}, the West Slavic language group consists of Czech, Slovak, Polish, Kashubian and Sorbian. As mentioned, some sources divide the languages slightly differently; for example \cite{siewierska1998overview} breaks Sorbian into Upper and Lower Sorbian; other sources take Kashubian as only a variant of Polish.

South Slavic language group consists of Serbo-Croatian, Bulgarian, Slovenian and Macedonian. As before, this list is taken from \cite{sussex2011slavic} -- the question of Serbo-Croatian language unity, for example, is a highly politiced one, thanks to recent military conflicts in the region.

The Eastern Slavic language group consists of Russian, Ukrainian and Belarussian, again according to \cite{sussex2011slavic}.

%The differences between the languages themselves are less syntactical and are more phonological, morphonological and morphological, of course with some lexical changes. 

%In the section \ref{slavic_overview}, I am describing
% For that reason, I am not describing the linguistic differences of the languages any further, since it would be, again, not in a scope for this thesis. Again, I can recommend the book \cite{sussex2011slavic} for anybody interested in the deeper differences.


\section{Slavic languages overview}
\label{slavic_overview}
In this section, I am almost exclusively citing \cite{sussex2011slavic}. While there are books like \cite{comrie2003slavonic}, they describe the languages one-by-one, whereas \cite{sussex2011slavic} compares language properties across all Slavic languages in a concise and clear manner that I haven't been able to find anywhere else.

Therefore, even when the book is not quoted and cited \emph{directly}, the general information in this chapter is heavily informed by it.

\subsection{Morphology}
On the morphological typology, Slavic languages belong to synthetic inflectional languages. They are morphologically rich, with a sophisticated affix system and a little analytical approach to verb morphology (for forms like future tense).

Slavic word is composed of roots (which can be one or several) and affixes (prefixes and suffixes). Prefixes usually modify the word's meaning somehow (for example, \uv{ne-} for negative), while suffixes modify the word's class or one of its grammatical categories.
%are either derivational, inflectional or post-inflectional (appearing in this order).

Suffixes are of several types. The ones appearing first are the derivational suffixes, which can determine the word's class \uv{in familiar processes like abstract and agent nominalization, verbalization, adjectivalization}, as noted in \cite{sussex2011slavic}. Next type of suffix are endings\footnote{Called inflectional suffixes in \cite{sussex2011slavic}} that mark one of several grammatical categories -- like \uv{infinitive, person,
number, tense, case and gender}, as noted  in \cite{sussex2011slavic}.

\jednatabulkan{reflex} { |l |l | l | }
{
         \hline

to wash (someone) & mýt & мыть \\ \hline
to wash (self) & mýt se & мыться \\ \hline


} {Reflectives in Russian vs. Czech} 


In some Slavic languages -- like Russian -- reflexivity is expressed by special suffix, called \emph{post-inflectional suffix} by \cite{sussex2011slavic};
whereas in others -- like Czech -- a separate word is used. See for example Table~\ref{tabulka:reflex} (also compare to the voices in the section~\ref{ch:mediopassive}).

\jednatabulkan{categories} { |l |l | l | }
{
         \hline

\textbf{Verbs} & \textbf{Nominals} & \textbf{Both} \\ \hline
Tense & Case & Person \\ \hline
Aspect & Definiteness & Gender \\ \hline
Mood & Deixis & Number \\ \hline
Voice &  &  \\ \hline


} {Inflectional categories} 

Inflectional categories that exist in  Slavic languages are described in the Table~\ref{tabulka:categories} (definiteness and deixis as inflectional categories are, however, not relevant to either Czech or Russian, but are used as such in Macedonian or Bulgarian).\footnote{The table is, again, from \cite{sussex2011slavic}.}

Morphological process are quite similar across Slavic languages -- however, using either different affixes or using the same affixes, but with slightly shifted properties (diferent categories, etc.).

%Slavic languages are surprisingly uniform in the word formation system, and the operations and grammatical processes are simmilar -- however, the affixes themselves are usually either wholly different, or have vastly different semantic and stylistic properties -- either absolutely, or in combination with other words.

In general, it can be said that languages from the Slavic family are morphologically richer than other languages. The result of this richness is that the number of word forms arise significantly, especially when compared with more analytical languages (like English). As you can see in the section TODO, in the same corpus, Czech and Russian use \emph{significantly} more word forms, compared to English.
%That can hurt us 
\subsection{Word order}

Languages like English use word order for marking constituents
in a sentence. Slavic languages, on the other hand, use inflective processes, like agreement\footnote{\cite{sussex2011slavic} lists concord, agreement and government}, for the same type of information -- subject \emph{agrees} with predicate, and so on.

Because those relations are marked by inflexion, it \uv{allows} the Slavic languages to be of freer word order. The \emph{standard} order is SVO -- however, this is possible to change when different emphasis is needed.

In particular, this reffers to the so-called Functional Sentence Perspective. Very simply said, it describes the sentence as consisting of two parts -- Topic and Comment, appearing in this order and being separated by a verb. Topic is the part of sentence, \emph{about which we say some information} -- while Comment is the \emph{new information}. Most importantly, Topic doesn't have to be a grammatical subject of the sentence -- for example, the Czech sentence \uv{Ve městě bydlí strašidla} (\emph{Ghosts live in the city}, literally \emph{In city live ghosts}).

Slavic languages usually (with some exceptions like Bulgarian and Macedonian) don't have particles or any other means of marking definitiveness -- the only mean is the definite information being in the topic of the sentence. In other view, the functional sentence perspective can be viewed as \uv{replacing} the definitive particles, and is usually translated as such in translation to English.

In respect to the translation between English and Slavic languages, word order can be seen as something that's hard to translate correctly. 
With respect to translation between Slavic languages, it could possibly help us, since we don't have to, in theory, change the word order too much.
\subsection{Grammatical categories and their changes}
In this section, I will show several grammatical categories and their change from PIE to Slavic languages.

\subsubsection{Cases}

PIE had at least eight cases -- nominative, accusative, genitive, dative, instrumental, locative, ablative and vocative. \cite{sussex2011slavic} lists these eight cases; \cite{ringe2008proto} also adds allative (which, however, survived only in old Hittite).

In Slavic languages, ablative and genitive were conflated into just genitive.
We can demonstrate this conflation on of two phrases from New Testament.
%To illustrate, I have added a simple comparison of old latin and Czech. 

\jednatabulka{korintskym} { |l|X |X | X | }
{
         \hline
genitive &
in\-vo\-cant no\-men \textbf{Do\-mi\-ni} no\-stri Ie\-su Chris\-ti
&
   who call on the name \textbf{of} our \textbf{Lord} Jesus Christ         
&
všem, kdo na ja\-kém\-ko\-li mí\-stě vzý\-va\-jí jmé\-no na\-še\-ho spo\-leč\-né\-ho \textbf{Pá\-na} Je\-ží\-še Kris\-ta
\\

    \hline
ablative
&
gra\-tia vo\-bis et pax a Deo Pa\-tre no\-stro et \textbf{Do\-mi\-no} Ie\-su Chris\-to
&
Grace and pe\-ace to you \textbf{from} God our Fa\-ther and \textbf{the Lord} Je\-sus Christ
&
Mi\-lost vám a po\-koj od Bo\-ha, na\-še\-ho Ot\-ce, a od \textbf{Pá\-na} Je\-ží\-še Kris\-ta \\
    \hline
} {Demonstrating ablative and genitive conflation on 1 Corinthians} 


Old Latin retained both ablative and genitive. Latin genitive of \emph{dominus} (lord, master) is \emph{dominī}, latin ablative of the same word is  \emph{dominō}. Both these forms were used in the very beginning of 1~Corinthians in Latina Vulgata (latin version of The Bible).

In the Table~\ref{tabulka:korintskym}, you can see the translation to both Czech and English. Both cases are inflated in Czech as genitive \uv{(našeho) pána}.\footnote{Bible sources: \cite{latinavulgata}, \cite{bibleniv}, \cite{bible21}. Note that English and Czech translations are not actually translations from Latin, but from more primary sources, but it will suffice for this simple comparison.}

\subsubsection{Numbers}
\jednatabulkan{kravy} { |l|l |l | l | }
{
         \hline
singular &
one cow
&
ena krava
&
jedna kráva
\\
   \hline
dual (in Slovenian) &
two cows
&
dve kravi
&
dvě krávy
\\
   \hline
plural &
three cows
&
tri krave
&
tři krávy
\\


    \hline
} {Demonstrating duals on Slovenian} 

PIE had three numbers -- singular, plural and dual.\footnote{According to both \cite{sussex2011slavic} and \cite{ringe2008proto}.} 

In most of the Slavic languages (including Czech and Russian), dual disappeared, leaving only traces in the grammar. One of the languages where dual remained in full is Slovenian. To illustrate dual in Slovenian, I have added a comparison of \uv{one cow}, \uv{two cows} and \uv{three cows} in Slovenian and Czech in Table~\ref{tabulka:kravy}.

In Czech, dual was retained in declensions of several words, like \uv{hands}; in Russian, the dual \uv{рукама} survives in some dialects, but is generally incorrect.\footnote{See \cite{offord1996using}, page 18.}

\subsubsection{Genders}
PIE had three genders, masculine, feminine and neutral.\footnote{As noted in both \cite{sussex2011slavic} and \cite{ringe2008proto}.} Slavic languages retained these genders, refining them with added features Personal and Animate.

\subsubsection{Tenses}
According to \cite{sussex2011slavic}, late PIE had six tenses: present, future, aorist, imperfect, perfect and pluperfect. 

The tenses were somehow retained in Slavic languages; however, they are used more analytically and with the help of auxiliary verbs -- for example, \uv{budu zpívat} (I will sing) in Czech, or \uv{буду петь} in Russian.

\subsubsection{Moods}
PIE had four moods: indicative, subjunctive, optative and imperative.\footnote{According to both \cite{sussex2011slavic} and \cite{ringe2008proto}} 

In Slavic languages, imperative was replaced by the optatives, and subjunctive mood became conditional.

\subsubsection{Voices}
\label{ch:mediopassive}
\jednatabulkan{voices} { |l|l |l | l | }
{
         \hline

active & νιπτω & I wash (someone) & myji (někoho) \\ \hline
medium & νίπτομαι & I wash (myself) & myji se \\ \hline
passive & νίπτομαι & I am washed (by somebody) & jsem myt \\ \hline


} {Voices in Classic Greek vs. Czech} 

PIE had an active and a mediopassive voice.\footnote{According to both \cite{sussex2011slavic} (where the mediopassive voice is called \uv{middle voice}) and \cite{ringe2008proto}} 

In Slavic languages, this was refined as active and a passive voice, while reflexives, in a way, took the function of a mediopassive voice.

Since mediopassive voice will probably be unknown in general to the reader, I have added an example of Classic Greek that still retained it, in Table~\ref{tabulka:voices},\footnote{See for example \cite{greek1}, \cite{greek2}} with both English and Czech translation.


\subsubsection{Aspects}
\jednatabulkan{nosim} { |l|l |l |  }
{
         \hline
imperfective determinate &
нести́
&
nést
\\
   \hline
imperfective indeterminate &
носи́ть
&
nosit
\\
   \hline
perfective& 
понести 
&
ponést

\\


    \hline
} {Aspects in Czech and Russian} 

PIE had distinction between two aspects -- eventive and stative; eventive aspect being further divided into perfective and imperfective aspect.\footnote{See \cite{ringe2008proto}, page 24}

In Slavic, the sta\-tive as\-pect is de\-gra\-mma\-tized\footnote{\cite{andersen2013origin}}; ho\-we\-ver, the per\-fec\-tive / im\-per\-fe\-ctive di\-stin\-ction be\-came more important than in other Indo-European languages. \cite{sussex2011slavic} calls the growing distinction \uv{the most important development in Proto-Slavic}.

Imperfective motion verbs were also added determinate/non-determinate distinction.

This determinate / non-determinate and perfective / imperfective distinction is present in both Czech and Russian. For the demonstration on the two languages, see Table~\ref{tabulka:nosim} and note, how hard would be to correctly translate the distinction into English.


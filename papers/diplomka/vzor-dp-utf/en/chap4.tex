\chapter{Black-box systems}

\section{PC Translator}

\subsection{Description}

PC Translator is a commercial translation system from a Czech company LangSoft (\url{http://www.langsoft.cz/translator.htm}). PC Translator can translate several language pairs with Czech on either source or target side.

Authors of PC Translator don't publish any papers or other literature about the system -- what can we tell about its functionality is gathered only from its promotional website and from the experiments with the software itself.

PC translator seems to be purely rule-based. The system seems to work in following steps:

\begin{itemize}
\item some (probably rule-based) morphological analysis of the source language
\item translation of the lemmas from source language to target language by searching in a large dictionary
\item some synthesis of morphological information and the translated lemma
\end{itemize}

The system doesn't seem to do any kind of reordering. It also doesn't seem to do any analysis on a deeper level, like sentence constituents. Some of the phrases in the dictionary are longer than one word, but not too many of them.

One of the advantages of PC Translator is its large dictionary -- however, the dictionary is sometimes choosing very odd and inprobable choices when disambiguating between more possible translations. For example, the English sentence \uv{I like dogs} is translated as \uv{Mám rád kleště}, because the term \uv{dog} can be also translated as \uv{kleště}\footnote{from Collins' Dictionary: \uv{dog -- 5. a mechanical device for gripping or holding, esp one of the axial slots by which gear wheels or shafts are engaged to transmit torque}}.

According to its marketing materials, PC Translator v14 uses a Czech-Russian dictionary with above 650.000 words.

\subsection{Experiments}
We found out it's not easy to automatize translating with PC Translator. Its GUI is suited for translating by hand, sentence-by-sentence, but not for automated translation of thousands of sentences. Also, by definition, Windows GUI is harder to automate on Linux machine from a script.

However, we were able to do some work-around, with the help of VMWare Player virtualization software (\url{http://www.vmware.com/cz/products/player}) and AutoHotkey GUI scripting software (\url{http://www.autohotkey.com/}). Our workflow therefore is:

\begin{itemize}
\item on Linux machine, encode the source from UTF-8 to windows-friendly encoding
\item encode the source as HTML code
\item start a virtual machine with PC Translator pre-installed
\item on the start of the virtual machine, run AutoHotkey script from an outer-machine folder (thanks to VMWare shared folders and Windows Startup scripts)
\item via this AutoHotkey script, run PC Translator and click on \uv{translate file} feature 
\item translate the HTML file (also shared in the VMWare shared folder)
\item turn off the virtual machine
\item turn the file back from HTML and Windows encodings back to UTF-8
\end{itemize}

The HTML part is needed because PC Translator had some problems with translating ordinary text files, plus we can pair the translated sentences better thanks to \texttt{id} parameters in \texttt{div} tags.

We used the newest version of PC Translator available at the time, which is PC Translator v14.



\subsection{Results}
\subsubsection{Demonstration}
\priklady {
\priklad{ Je mi teprve čtyřiadvacet let a nemohu prožit celý svůj život s legitimací invalidy práce a potloukat se po nemocnicích , když vím , že je to marné~. }{ Есть мне только двадцать четыре рейс и не могу пережив весь свой жизнь с удостоверение личности инвалидами труд и болтаться по больницах , когда знаю , что это бесполезные~. }{Est` mne tol`ko dvadczat` chety're rejs i ne mogu perezhiv ves` svoj zhizn` s udostoverenie lichnosti invalidami trud i boltat`sya po bol`niczax , kogda znayu , chto e`to bespolezny'e~.}
\priklad{ Generál chodil po pokoji sem a tam , kouře svou pěnovku~. }{ Генерал ходил по комнате туда - сюда , дыма свою пенковая трубка~. }{General xodil po komnate tuda - syuda , dy'ma svoyu penkovaya trubka~.}
\priklad{ " Možná že žádné brilianty neexistují ? " }{ " возможно никакие бриллианты отсутствовать ?  }{" vozmozhno nikakie brillianty' otsutstvovat` ? }
\priklad{ Nesměl při tom udělat chybu , vyžadovalo to stejnou přesnost , jako když se zaměřuje dělo~. }{ Обмен причём ошибиться , требовало то такой же аккуратность , как когда специализируется пушка~. }{Obmen prichyom oshibit`sya , trebovalo to takoj zhe akkuratnost` , kak kogda specializiruetsya pushka~.}
\priklad{ S hrdostí vzpomněl , jak snadno dobyl kdysi srdce krásné Heleny Baurové~. }{ С гордостью вспомнил , как ходко захватил когда - то сердце красивое Елена Баурове~. }{S gordost`yu vspomnil , kak xodko zaxvatil kogda - to serdce krasivoe Elena Baurove~.}
\priklad{ Thiago Silva, který patří k nejlepším obráncům na světě, taky umožňuje ostatním vedle sebe růst. }{ Тhиаго Силва, какой принадлежать к лучшим защитник в мире, тоже даёт возможность остальным плечом к плечу рост. }{Thiago Silva, kakoj prinadlezhat` k luchshim zashhitnik v mire, tozhe dayot vozmozhnost` ostal`ny'm plechom k plechu rost.}
\priklad{ "Dávali mi pět let života a už je to sedm," říká bez emocí na svém lůžku v domě pro paliativní péči Victor-Gadbois v Beloeil, kam přijel předešlý den. }{ "давали мне пять лет жизни и уже это семь," говорит минус эмоций в своем гнезде в доме для башка опеку Victor-Гадбоис зажечься Бєлоєил, куда доехал предшествующий день. }{"davali mne pyat` let zhizni i uzhe e`to sem`," govorit minus e`mocij v svoem gnezde v dome dlya bashka opeku Victor-Gadbois zazhech`sya Bєloєil, kuda doexal predshestvuyushhij den`.}
\priklad{ Opatrnost je ovšem na místě například na některých přemostěních, kde může být povrch namrzlý a kluzký. }{ Осторожность есть конечно на месте например на некоторых перекрытие, где может быть покрытие замёрзший и сальный. }{Ostorozhnost` est` konechno na meste naprimer na nekotory'x perekry'tie, gde mozhet by't` pokry'tie zamyorzshij i sal`ny'j.}
\priklad{ Prostě je ignoruji. }{ Запросто есть игнорирую. }{Zaprosto est` ignoriruyu.}
\priklad{ Podle doktorky Christiane Martelové není quebecký zdravotnický systém dostatečně výkonný, aby zajistil přístup všech osob ke kvalitní paliativní péči, než bude možno souhlasit s provedením eutanazie. }{ По врачи Christiane Мартєлове нет qуєбєцкэ медицинский комплекс достаточно производительный, чтобы обеспечил подход всех» личностей к качественный башка опеку, нежели можно будет согласиться с проведением єутаназиє. }{Po vrachi Christiane Martєlove net quєbєczke` medicinskij kompleks dostatochno proizvoditel`ny'j, chtoby' obespechil podxod vsex» lichnostej k kachestvenny'j bashka opeku, nezheli mozhno budet soglasit`sya s provedeniem єutanaziє.}
}



\subsubsection{BLEU}

\jednatabulkan{bleutrans} { |r|r|r | }
{
\hline
&
intercorp&
WMT\\ \hline
BLEU & 5.02\% $\pm$ 0.01\%
&
6.74\% $\pm$ 0.32\%

\\ \hline
}{PC translator BLEU}

\subsubsection{Discussion}
We can see the results are not ideal. The sentences are evidently translated word-by-word, without any regard for the whole sentence, or the correct word order.

Because the languages are similar, the morphological information is sometimes transferred correctly -- for example, \uv{s hrdostí vzpomněl} -> \uv{с гордостью вспомнил} -- but sometimes, very incorrectly -- for example, genitive \uv{překrásné} is not with the correct ending as \uv{-ой}, but incorrectly as \uv{-ое}.

It's obvious the dictionary is quite big and complete, so a lot of the words were translated somehow correctly; however, some of the words had strange ?????????


\section{Google Translate}
\subsection{Description}
Google Translate is a popular free online translation service by Google, an American web search giant (\url{http://translate.google.com}). 
Although Google is producing many academic papers on machine translation, the whole system is still proprietary and we cannot fully inspect it, as in the case of PC Translator.

Google Translate uses mostly statistical approach to machine translation, see for example \cite{och}\footnote{F. J. Och is a head of Google Translate group in Google}. Its results are often seen as \uv{state-of-the-art} in machine translation.

However, thanks to its purely statistical approach, it either needs huge amounts of data for every language pair, or it needs to use so-called \uv{pivot languages}\footnote{See for example \cite{koehn2010statistical}} -- in the case of Google Translate, it's usually English; specific English word order and English idioms are then re-translated into the target language and sometimes introduce downright wrong translations.

\subsection{Experiments}
To automate Google Translate, we cannot use the website itself, simply because pasting tens of thousands of lines into a browser window usually crashes the browser itself.

There are some workarounds around this, such as \uv{faking} browser environment using some automation tools and/or libraries, but we used more stable option.

Google Translate, apart from being a website, has a paid translation API\footnote{\url{https://developers.google.com/translate/?hl=cs}}. We figured out it's not too expensive for our testing purposes, so we ended up paying for the API.\footnote{The cost is measured per character on the source side. We used about 3 million characters and paid about 60 dollars. This is rather high for any repeated experiments, but not that high for one-time translation.}

We used an unofficial Java library for Google Translate API, called prosaically \uv{google-api-translate-java}\footnote{https://code.google.com/p/google-api-translate-java/}.

The tests were done on 3rd May, 2014.\footnote{I think it's important to note the date of the tests, because the quality of online services might change overtime.}

\subsection{Results}

\subsubsection{Demonstration}
\priklady {
\priklad{ Je mi teprve čtyřiadvacet let a nemohu prožit celý svůj život s legitimací invalidy práce a potloukat se po nemocnicích , když vím , že je to marné~. }{ Я всего двадцать четыре года, и я живу свою жизнь с купоном недействительным работы и болтаться в больнице, зная, что это бесполезно. }{Ya vsego dvadczat` chety're goda, i ya zhivu svoyu zhizn` s kuponom nedejstvitel`ny'm raboty' i boltat`sya v bol`nice, znaya, chto e`to bespolezno.}
\priklad{ Generál chodil po pokoji sem a tam , kouře svou pěnovku~. }{ Генеральный ходил по комнате взад и вперед, его дым пены. }{General`ny'j xodil po komnate vzad i vpered, ego dy'm peny'.}
\priklad{ " Možná že žádné brilianty neexistují ? " }{ "Возможно, не алмазы там?" }{"Vozmozhno, ne almazy' tam?"}
\priklad{ Nesměl při tom udělat chybu , vyžadovalo to stejnou přesnost , jako když se zaměřuje dělo~. }{ Он не мог ошибиться в этом, это требуется такой же точности, как при фокусировке пушки. }{On ne mog oshibit`sya v e`tom, e`to trebuetsya takoj zhe tochnosti, kak pri fokusirovke pushki.}
\priklad{ S hrdostí vzpomněl , jak snadno dobyl kdysi srdce krásné Heleny Baurové~. }{ Мы с гордостью вспомнил, как легко покорил когда-то сердце прекрасной Елены Баур. }{My' s gordost`yu vspomnil, kak legko pokoril kogda-to serdce prekrasnoj Eleny' Baur.}
\priklad{ Thiago Silva, který patří k nejlepším obráncům na světě, taky umožňuje ostatním vedle sebe růst. }{ Тьяго Силва, один из лучших защитников в мире, также позволяет другой параллельный рост. }{T`yago Silva, odin iz luchshix zashhitnikov v mire, takzhe pozvolyaet drugoj parallel`ny'j rost.}
\priklad{ "Dávali mi pět let života a už je to sedm," říká bez emocí na svém lůžku v domě pro paliativní péči Victor-Gadbois v Beloeil, kam přijel předešlý den. }{ "Они дали мне пять лет жизни, и это семь", говорит он без эмоций на его постели у себя дома для паллиативной помощи Виктор-Гадбуа в Beloeil, куда они прибыли в предыдущий день. }{"Oni dali mne pyat` let zhizni, i e`to sem`", govorit on bez e`mocij na ego posteli u sebya doma dlya palliativnoj pomoshhi Viktor-Gadbua v Beloeil, kuda oni priby'li v predy'dushhij den`.}
\priklad{ Opatrnost je ovšem na místě například na některých přemostěních, kde může být povrch namrzlý a kluzký. }{ Внимание, однако, находится в месте, например, некоторые перемычки, где поверхность может быть ледяной и скользкий. }{Vnimanie, odnako, naxoditsya v meste, naprimer, nekotory'e peremy'chki, gde poverxnost` mozhet by't` ledyanoj i skol`zkij.}
\priklad{ Prostě je ignoruji. }{ Просто игнорируйте их. }{Prosto ignorirujte ix.}
\priklad{ Podle doktorky Christiane Martelové není quebecký zdravotnický systém dostatečně výkonný, aby zajistil přístup všech osob ke kvalitní paliativní péči, než bude možno souhlasit s provedením eutanazie. }{ По словам доктора Кристиана Martel Квебеке система здравоохранения не является достаточно мощным, чтобы обеспечить доступ для всех людей на высококачественной паллиативной помощи, прежде чем он может согласиться проводить эвтаназию. }{Po slovam doktora Kristiana Martel Kvebeke sistema zdravooxraneniya ne yavlyaetsya dostatochno moshhny'm, chtoby' obespechit` dostup dlya vsex lyudej na vy'sokokachestvennoj palliativnoj pomoshhi, prezhde chem on mozhet soglasit`sya provodit` e`vtanaziyu.}
}

\subsubsection{BLEU}
\jednatabulkan{bleugoog} { |r|r|r | }
{
\hline
&
intercorp&
WMT\\ \hline
BLEU & 8.79\% $\pm$ 0.15\%
&
17.96\% $\pm$ 0.58\%

\\ \hline
}{Google translate BLEU}

\subsubsection{Discussion}

The first thing to note is the radically different results for Intercorp and WMT corpus. This disprepancy is not repeated in our other experiments, where both sets have BLEU that's closer to each other.

This might be caused by two reasons. One reason might be that the complete WMT data are simply already included in Google's training data, while Intercorp is not. It's not that far-fetched theory: WMT is one year old already, and Google is one of WMT sponsors. On the other hand, in this case, the results should be even better than they actually are -- Google's results are, after all, worse, than our Moses experiments in chapter ??.

Other reason might be that Google is domain specificity of Google's training data. It's possible that Google is generally using far more newspaper data for training than just fiction, which might disadvantage Google's system in Intercorp case. Since we don't know Google's training data structure, we can't know for sure.



NĚJAKÉ HODNOCENÍ VÝSLEDKŮ - já osobně to nedokážu vůbec zhodnotit.

\section{Microsoft Bing Translator}
\subsection{Description}
Another online service that we decided to try is Microsoft Translator/Bing Translator. (In Microsoft's own materials, the system is usually called Bing Translator when referring to the website and Microsoft Translator when referring to the API.)

Microsoft Translator is very similar to Google Translate -- it is an online website with an easy GUI, and an additional paid API. Again, the team occasionally publishes some scientific papers, but is again otherwise proprietary.

In different experiment, we found out (non-scientifically), that for some language pairs, Microsoft Translator does more post-editation, that seemed a little rule-based (for example, better verb separation in English-to-German translation). Therefore, we decided to try it alongside Google Translate in our experiments.


\subsection{Experiments}
Again, we used Microsoft's Translator API (marketed as part of Windows Azure).

The API is slightly more complicated than Google's API because of the auto-expiring token, but we used the example PHP script from the API documentation\footnote{\url{http://msdn.microsoft.com/en-us/library/hh454950.aspx}}.

The pricing is slightly different in Microsoft Translator than in Google Translate, but in general is slightly cheaper. First 2 million letters are for free, next 2 million are for about 40 US dollars.

\subsection{Results}
\subsubsection{Demonstration}

\priklady {
\priklad{ Je mi teprve čtyřiadvacet let a nemohu prožit celý svůj život s legitimací invalidy práce a potloukat se po nemocnicích , když vím , že je to marné~. }{ Já jsem jen dvacet čtyři let, a já nemůžu žít jejich dní s relax pracovních Rove na nemocnic, protože věděl, že nemá cenu se. }{Já jsem jen dvacet čtyři let, a já nemůžu žít jejich dní s relax pracovních Rove na nemocnic, protože věděl, že nemá cenu se.}
\priklad{ Generál chodil po pokoji sem a tam , kouře svou pěnovku~. }{ Generál pohyboval sem a tam, kouření jeho potrubí penkovuû. }{Generál pohyboval sem a tam, kouření jeho potrubí penkovuû.}
\priklad{ " Možná že žádné brilianty neexistují ? " }{ -Možná, diamanty, ne? }{-Možná, diamanty, ne?}
\priklad{ Nesměl při tom udělat chybu , vyžadovalo to stejnou přesnost , jako když se zaměřuje dělo~. }{ Tato práce však neumožňuje Přepis chyb-stejně jako zrak. }{Tato práce však neumožňuje Přepis chyb-stejně jako zrak.}
\priklad{ S hrdostí vzpomněl , jak snadno dobyl kdysi srdce krásné Heleny Baurové~. }{ Hrdě připomněl, jak snadno dobyl srdce krásné Bouryho Helen. }{Hrdě připomněl, jak snadno dobyl srdce krásné Bouryho Helen.}
\priklad{ Thiago Silva, který patří k nejlepším obráncům na světě, taky umožňuje ostatním vedle sebe růst. }{ Thiago Silva, který je jedním z nejlepších na světě, také motivuje, vše se pohnout kupředu. }{Thiago Silva, který je jedním z nejlepších na světě, také motivuje, vše se pohnout kupředu.}
\priklad{ "Dávali mi pět let života a už je to sedm," říká bez emocí na svém lůžku v domě pro paliativní péči Victor-Gadbois v Beloeil, kam přijel předešlý den. }{ "Dostal jsem pět let, žil jsem sedm," říká, že mezi životem a smrtí, ležící v posteli a v útulku paliativní péče, Viktor-Gadbua v Belœil, den před příjezdem. }{"Dostal jsem pět let, žil jsem sedm," říká, že mezi životem a smrtí, ležící v posteli a v útulku paliativní péče, Viktor-Gadbua v Belœil, den před příjezdem.}
\priklad{ Opatrnost je ovšem na místě například na některých přemostěních, kde může být povrch namrzlý a kluzký. }{ Nicméně péče je zapotřebí, například jako na mosty, kde může být kluzký povrch a namerzšaâ. }{Nicméně péče je zapotřebí, například jako na mosty, kde může být kluzký povrch a namerzšaâ.}
\priklad{ Prostě je ignoruji. }{ Jen nevěnují velkou pozornost. }{Jen nevěnují velkou pozornost.}
\priklad{ Podle doktorky Christiane Martelové není quebecký zdravotnický systém dostatečně výkonný, aby zajistil přístup všech osob ke kvalitní paliativní péči, než bude možno souhlasit s provedením eutanazie. }{ Podle Dr. Christiane Martel Quebec systém zdravotní péče není účinně zajistit právo na kvalitní paliativní péče, než jim bude povoleno přechod k eutanazii. }{Podle Dr. Christiane Martel Quebec systém zdravotní péče není účinně zajistit právo na kvalitní paliativní péče, než jim bude povoleno přechod k eutanazii.}
}

\subsubsection{BLEU}

\jednatabulkan{bleubing} { |r|r|r | }
{
\hline
&
intercorp&
WMT\\ \hline
BLEU & 7.60\% $\pm$ 0.13\%
&
16.27\% $\pm$ 0.52\%

\\ \hline
}{Microsoft Translator BLEU}

\subsubsection{Discussion}

Já to celý budu muset nějak přepsat. Nevim co sem psát.


\chapter*{Introduction}
\addcontentsline{toc}{chapter}{Introduction}

TODO: předělat: intro by mělo víc popisovat celou práci.

Czech and Russian are languages that share a long history.

At approximately 2000 BC, Proto-Slavic started to break away from Proto-Indo-European
\footnote{See the chapter \ref{ch:common_history}.}, slowly creating a Slavic family of languages, nowadays including Czech and Russian.
%-- that nowadays includes Czech, Russian and other 8 langues \footnote{The final count depent on exact definition. See for example \cite{sussex2011slavic} or \cite{siewierska1998overview} --  the final language count depends on whether Upper and Lower Sorbian are taken as different languages or not; whether Kashubian is a separate language, or just a variety of Polish; whether Serbian and Croatian are treated as separate languages or not}.

If we consider more recent events, in the 20th century, Czechoslovakia was a part of so-called Eastern Bloc and Soviet Union was its dominant force. Without going too far into politics, we can see why was the translation between those two languages considered important.

As for the Slavic connection -- Slavic languages are usually divided into three sub-groups -- South Slavic, West Slavic and East Slavic. While languages inside each of those groups are in general mutually intelligible, intelligibility accross those groups is lower; still, the Slavic languages in general have very similar morphology, syntax and vocabulary. 

Czech language belongs to the West Slavic group, while Russian belongs to the East Slavic group. Therefore, the intelligibility is lower -- so doing machine translation makes sense -- but some aspects of the language are still similar. In the work described in this thesis, I am trying to ask if the similarities are somehow exploitable for the machine translation task.

%Therefore, they are not fully mutually intelligible -- but, as noted, as Slavic languages, they are in some respects still very similar. 
%This creates a great environment for building machine translations tools -- the languages are not too close to make the task meaningless, while we can still take an advantage of the closeness.

As for the political connection -- because of the 
political role of Russian,
 significant efforts were directed towards building an automatic translation system between Russian and Czech, mainly during mid-eighties with the RUSLAN system. The efforts were terminated in 1990 in the final phases of development.\footnote{See for example \cite{recycled}, \cite{hajic1987}, \cite{olivaruslan}} 

After the velvet revolution, political dependence on Soviet Union (and then just Russia) was weakened, while the economical and cultural dependence on English-speaking countries was strengthened. The efforts of automatic translation were then more focused on English. 

In the process, general direction of machine translation efforts shifted from rule-based systems to more statistical-based systems. In this thesis, I am trying to describe this historical system and compare it with some more modern approaches.

%This allows us to compare older, rule-based approaches to the more modern statistical approaches.

After a brief introduction to the two languages and its similarities and differences, I will describe all the available Czech-to-Russian translation systems, the amount of work needed to get them running and the work we spent on their additional development. Then I will describe a common set of test data and the exact testing procedure and lastly, I will describe the results of the described tests.


%In general, I don't have ambitions for any scientific breaktrough in the Czech-to-Russian translation; I just hope that I have successfully compiled and implemented all the existing approaches and compared their general translation quality.
